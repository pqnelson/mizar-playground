\documentclass[oneside]{book}% ===> this file was generated automatically by noweave --- better not edit it
\usepackage{mizar}
\usepackage{tutmac}

\title{Tutorial on Mizar}
\author{Alex Nelson}
\date{December 11, 2021}

\begin{document}

\chapter{Introduction}

\M
Every Mizar article looks like:

\nwfilename{tutorial.nw}\nwbegincode{1}\sublabel{NW1Vh4dG-479AAV-1}\nwmargintag{{\nwtagstyle{}\subpageref{NW1Vh4dG-479AAV-1}}}\moddef{text/hello.miz~{\nwtagstyle{}\subpageref{NW1Vh4dG-479AAV-1}}}\endmoddef\nwstartdeflinemarkup\nwenddeflinemarkup

\LA{}Environment for \code{}hello.miz\edoc{}~{\nwtagstyle{}\subpageref{NW1Vh4dG-2bJZfg-1}}\RA{}

\LA{}\code{}hello.miz\edoc{} article body~{\nwtagstyle{}\subpageref{NW1Vh4dG-1vhHfS-1}}\RA{}
\nwnotused{text/hello.miz}\nwendcode{}\nwbegindocs{2}\nwdocspar



\section{Header}

\M
The header, or ``environment part'', tells Mizar what mathematics needs
to be imported from existing Mizar articles found in the MML.

\nwenddocs{}\nwbegincode{3}\sublabel{NW1Vh4dG-2bJZfg-1}\nwmargintag{{\nwtagstyle{}\subpageref{NW1Vh4dG-2bJZfg-1}}}\moddef{Environment for \code{}hello.miz\edoc{}~{\nwtagstyle{}\subpageref{NW1Vh4dG-2bJZfg-1}}}\endmoddef\nwstartdeflinemarkup\nwusesondefline{\\{NW1Vh4dG-479AAV-1}}\nwenddeflinemarkup
environ
 \LA{}\code{}hello.miz\edoc{} vocabularies~{\nwtagstyle{}\subpageref{NW1Vh4dG-1Wxwer-1}}\RA{};
 \LA{}\code{}hello.miz\edoc{} constructors~{\nwtagstyle{}\subpageref{NW1Vh4dG-ucvcb-1}}\RA{};
 \LA{}\code{}hello.miz\edoc{} notations~{\nwtagstyle{}\subpageref{NW1Vh4dG-sSBDf-1}}\RA{};
\nwused{\\{NW1Vh4dG-479AAV-1}}\nwendcode{}\nwbegindocs{4}\nwdocspar

\N{Vocabularies}
The {\Tt{}vocabularies\nwendquote} refers to the identifiers defined. For example,
{\Tt{}Isomorphism\nwendquote} may be found in {\Tt{}RING{\_}3\nwendquote}. So even if I wanted to use
{\Tt{}Isomorphism\nwendquote} as defined in {\Tt{}GROUP{\_}66\nwendquote}, I need to add {\Tt{}RING{\_}3\nwendquote} to
the vocabularies list.

\nwenddocs{}\nwbegincode{5}\sublabel{NW1Vh4dG-1Wxwer-1}\nwmargintag{{\nwtagstyle{}\subpageref{NW1Vh4dG-1Wxwer-1}}}\moddef{\code{}hello.miz\edoc{} vocabularies~{\nwtagstyle{}\subpageref{NW1Vh4dG-1Wxwer-1}}}\endmoddef\nwstartdeflinemarkup\nwusesondefline{\\{NW1Vh4dG-2bJZfg-1}}\nwprevnextdefs{\relax}{NW1Vh4dG-1Wxwer-2}\nwenddeflinemarkup
vocabularies HELLO,
\nwalsodefined{\\{NW1Vh4dG-1Wxwer-2}}\nwused{\\{NW1Vh4dG-2bJZfg-1}}\nwendcode{}\nwbegindocs{6}\nwdocspar

\M
We usually want to use set theoretic notation, too, so we add

\nwenddocs{}\nwbegincode{7}\sublabel{NW1Vh4dG-1Wxwer-2}\nwmargintag{{\nwtagstyle{}\subpageref{NW1Vh4dG-1Wxwer-2}}}\moddef{\code{}hello.miz\edoc{} vocabularies~{\nwtagstyle{}\subpageref{NW1Vh4dG-1Wxwer-1}}}\plusendmoddef\nwstartdeflinemarkup\nwusesondefline{\\{NW1Vh4dG-2bJZfg-1}}\nwprevnextdefs{NW1Vh4dG-1Wxwer-1}{\relax}\nwenddeflinemarkup
  SUBSET_1, XBOOLE_0
\nwused{\\{NW1Vh4dG-2bJZfg-1}}\nwendcode{}\nwbegindocs{8}\nwdocspar

\N{Constructors}
But the vocabularies just permits Mizar's parser to \emph{recognize}
terms. For the \emph{meaning} of these terms, we need to import the
\emph{constructors}.

For now, we just need the powerset as defined in {\Tt{}XBOOLE{\_}0\nwendquote}, so we add
it to the constructors:

\nwenddocs{}\nwbegincode{9}\sublabel{NW1Vh4dG-ucvcb-1}\nwmargintag{{\nwtagstyle{}\subpageref{NW1Vh4dG-ucvcb-1}}}\moddef{\code{}hello.miz\edoc{} constructors~{\nwtagstyle{}\subpageref{NW1Vh4dG-ucvcb-1}}}\endmoddef\nwstartdeflinemarkup\nwusesondefline{\\{NW1Vh4dG-2bJZfg-1}}\nwenddeflinemarkup
constructors XBOOLE_0
\nwused{\\{NW1Vh4dG-2bJZfg-1}}\nwendcode{}\nwbegindocs{10}\nwdocspar

\N{Notations}
Now we need to import the functor patterns to ``couple'' the definitions
and notations. Usually this is just the constructor list.

\nwenddocs{}\nwbegincode{11}\sublabel{NW1Vh4dG-sSBDf-1}\nwmargintag{{\nwtagstyle{}\subpageref{NW1Vh4dG-sSBDf-1}}}\moddef{\code{}hello.miz\edoc{} notations~{\nwtagstyle{}\subpageref{NW1Vh4dG-sSBDf-1}}}\endmoddef\nwstartdeflinemarkup\nwusesondefline{\\{NW1Vh4dG-2bJZfg-1}}\nwenddeflinemarkup
notations XBOOLE_0
\nwused{\\{NW1Vh4dG-2bJZfg-1}}\nwendcode{}\nwbegindocs{12}\nwdocspar

\section{Article Body}

\M
The article body is where the magic happens.

\nwenddocs{}\nwbegincode{13}\sublabel{NW1Vh4dG-1vhHfS-1}\nwmargintag{{\nwtagstyle{}\subpageref{NW1Vh4dG-1vhHfS-1}}}\moddef{\code{}hello.miz\edoc{} article body~{\nwtagstyle{}\subpageref{NW1Vh4dG-1vhHfS-1}}}\endmoddef\nwstartdeflinemarkup\nwusesondefline{\\{NW1Vh4dG-479AAV-1}}\nwenddeflinemarkup
begin
\nwused{\\{NW1Vh4dG-479AAV-1}}\nwendcode{}

\nwixlogsorted{c}{{\code{}hello.miz\edoc{} article body}{NW1Vh4dG-1vhHfS-1}{\nwixu{NW1Vh4dG-479AAV-1}\nwixd{NW1Vh4dG-1vhHfS-1}}}%
\nwixlogsorted{c}{{\code{}hello.miz\edoc{} constructors}{NW1Vh4dG-ucvcb-1}{\nwixu{NW1Vh4dG-2bJZfg-1}\nwixd{NW1Vh4dG-ucvcb-1}}}%
\nwixlogsorted{c}{{\code{}hello.miz\edoc{} notations}{NW1Vh4dG-sSBDf-1}{\nwixu{NW1Vh4dG-2bJZfg-1}\nwixd{NW1Vh4dG-sSBDf-1}}}%
\nwixlogsorted{c}{{\code{}hello.miz\edoc{} vocabularies}{NW1Vh4dG-1Wxwer-1}{\nwixu{NW1Vh4dG-2bJZfg-1}\nwixd{NW1Vh4dG-1Wxwer-1}\nwixd{NW1Vh4dG-1Wxwer-2}}}%
\nwixlogsorted{c}{{Environment for \code{}hello.miz\edoc{}}{NW1Vh4dG-2bJZfg-1}{\nwixu{NW1Vh4dG-479AAV-1}\nwixd{NW1Vh4dG-2bJZfg-1}}}%
\nwixlogsorted{c}{{text/hello.miz}{NW1Vh4dG-479AAV-1}{\nwixd{NW1Vh4dG-479AAV-1}}}%
\nwbegindocs{14}\nwdocspar

\begin{definition}[{Dummit and Foote~\cite[\S4,4]{dummit-foote}}]
A subgroup $H$ of $G$ is called \define{Characteristic} in $G$, usually
denoted $H~\mathrm{char}~G$, if every Automorphism of $G$ maps $H$ to
itself; i.e., $\sigma(H)=H$ for all $\sigma\in\aut(G)$.
\end{definition}


\begin{thebibliography}{99}
\bibitem{dummit-foote}
  David Dummit and Richard Foote,
  \emph{Abstract Algebra}.
  Third ed., Wiley \& Sons, 2004.
\end{thebibliography}
\end{document}
\nwenddocs{}

\chapter{Group Theory in Mizar}

\section{First-Order Logic in Mizar}

\N{Propositional Logic}
We will briefly review Mizar's syntax for formulas.

\begin{equation}
  \neg\Phi\quad\equiv\quad\mathtt{not}~\Phi
\end{equation}
\begin{equation}
  \Phi\lor\Psi\quad\equiv\quad\Phi~\mathtt{or}~\Psi
\end{equation}
\begin{equation}
  \Phi\land\Psi\quad\equiv\quad\Phi~\mathtt{\&}~\Psi
\end{equation}
\begin{equation}
  \Phi\implies\Psi\quad\equiv\quad\Phi~\mathtt{implies}~\Psi
\end{equation}
\begin{equation}
  \Phi\iff\Psi\quad\equiv\quad\Phi~\mathtt{iff}~\Psi
\end{equation}

\N{Typing Statements as Propositions}
Unique to Mizar is that typing judgements are object language
propositions. When we write ``\texttt{x is natural number}'', that's a
proposition. Type theory would make it a judgement, a proposition
\emph{in the metalanguage}. Mizar makes it a proposition in the
\emph{object} language.

\N{Quantifiers}
Quantifiers are a bit trickier. We quantify a variable over a type.
\begin{equation}
\forall_{x:A}\Phi\quad\equiv\quad\mathtt{for}~x~\mathtt{being}~A~\mathtt{holds}~\Phi
\end{equation}
\begin{equation}
\exists_{x:A}\Phi\quad\equiv\quad\mathtt{ex}~x~\mathtt{being}~A~\mathtt{st}~\Phi
\end{equation}
We can also just ``stack'' multiple quantifiers next to each other, like
\begin{equation}
\forall_{x:A}\forall_{y:B}\Phi\quad\equiv\quad\mathtt{for}~x~\mathtt{being}~A~\mathtt{for}~y~\mathtt{being}~B~\mathtt{holds}~\Phi.
\end{equation}
We have syntactic sugar for universal quantification of an implication:
\begin{equation}
\forall_{x:A}\Phi\implies\Psi\quad\equiv\quad\mathtt{for}~x~\mathtt{being}~A~\mathtt{st}~\Phi~\mathtt{holds}~\Psi.
\end{equation}

\N{Final remark on notations used in Mizar}
There are a few quirky choices of notation which I wish to discuss, just
to let the reader be aware of it. I will also remind the reader when it
comes up in the future:

\begin{enumerate}
\item In mathematics, function application is written $f(x)$, but in
  Mizar it is ``\texttt{f.x}''
\item Applying a function to a set is denoted $f(A)$, but in Mizar it is
  denoted ``\texttt{f.:A}''
\item The inverse function relating $y=f(x)$ to $f^{-1}(y)=x$, in Mizar
  is denoted by ``\texttt{f".y}''. More generally, instead of using
  superscript $-1$, Mizar will use double quotes.
\item If $f\colon X\to Y$ is a function of sets, and $A\subset X$, then
  we write the restriction of $f$ to $A$ as $f|_{A}\colon A\to X$
  defined by $\forall a\in A, f(a)=f|_{A}(a)$. In Mizar, this
  restriction is denoted ``\verb#f|A#'' and function application with it
  is ``\verb#(f|A).x#'' for $f|_{A}(x)$.
\end{enumerate}

\section{Basic Definitions}

\M
Recall, as Baez~\cite[week one]{baez2004qg-lectures} notes, there are
two ways to define a group. These two different definitions give rise to
\emph{different} notions of ``group homomorphism'' (but identical
notions of ``group isomorphism''). I will make bold the differences:

\begin{definition}[Version 1]\label{defn:introduction:mizar-style-group}
  A \emph{group} consists of
  \begin{enumerate}
  \item a set $G$
  \end{enumerate}
  equipped with
  \begin{enumerate}
  \item a binary operator $\mu\colon G\times G\to G$
  \end{enumerate}
  such that
  \begin{enumerate}
  \item Associativity: for any $g_{1}$, $g_{2}$, $g_{3}\in G$, we have
    $\mu(g_{1}, \mu(g_{2},g_{3})) = \mu(\mu(g_{1},g_{2}),g_{3})$
  \item \textbf{Existence of Unit: there exists an element $1\in G$ such
    that for any $g\in G$ we have $\mu(g,1)=\mu(1,g)=g$}
  \item \textbf{Existence of inverses: for each $g\in G$, there is an
    $h\in G$ such that $\mu(g,h)=1$}
  \end{enumerate}
\end{definition}

\begin{definition}[Version 2]
  A \emph{group} consists of
  \begin{enumerate}
  \item a set $G$
  \end{enumerate}
  equipped with
  \begin{enumerate}
  \item a binary operator $\mu\colon G\times G\to G$
  \item \textbf{a unit $1\in G$}
  \item \textbf{an inverse operator $\iota\colon G\to G$}
  \end{enumerate}
  such that
  \begin{enumerate}
  \item Associativity: for any $g_{1}$, $g_{2}$, $g_{3}\in G$, we have
    $\mu(g_{1}, \mu(g_{2},g_{3})) = \mu(\mu(g_{1},g_{2}),g_{3})$
  \item \textbf{Unit laws: for any $g\in G$ we have $\mu(g,1)=\mu(1,g)=g$}
  \item \textbf{Inverse law: for each $g\in G$, $\mu(\iota(g),g)=1$}
  \end{enumerate}
\end{definition}

\begin{remark}
I will adopt modern terminology, and simply refer to a group
homomorphism as a ``\emph{group morphism}'', or more tersely as just a
morphism.
\end{remark}

\M Mizar takes the first version as its definition. Implicit in its
definition is a notion of a ``\verb#multMagma#'', a magma with a
multiplication operator (defined in \verb#ALGSTR_0#). As Nakasho and
friends explicate~\cite{nakasho2014formalization}, a ``\verb#multMagma#''
is a magma as could be found in, e.g., Bourbaki's
\emph{Algebra}~\cite[see Ch.I\S1.1]{bourbaki1974elements}: it
consists of a set (which Mizar calls its ``\verb#carrier#'') and a
binary operator ``\verb#multF#''.\index{multF@\texttt{multF}}\index{carrier@\texttt{carrier}}\index{multMagma@\texttt{multMagma}}

\begin{mizar}
:: algstr_0.miz
definition
  struct (1-sorted) multMagma (# carrier -> set,
  multF -> BinOp of the carrier
  #);
end;
\end{mizar}

\M
Let us unpack what is going on in this code. We are defining a
\verb#struct#, a mathematical structure which is usually presented in
ordinary mathematics as a tuple of some kind $\langle\dots\rangle$. For
example, a field might be presented as $\langle F, +, \cdot, 0, 1\rangle$
consisting of a set $F$, a couple binary operators, and a couple of
selected elements of $F$.

Mizar's grammar for defining a structure is:
\begin{equation}
  \langle\textit{structure-definition}\rangle ::=
  \mbox{\texttt{struct}}~(\langle\textit{ancestors}\rangle)~\langle\textit{symbol}\rangle~
  \mbox{\texttt{(\#}}\quad
  \langle\textit{fields}\rangle\quad
  \mbox{\texttt{\#)}}
  \texttt{;}
\end{equation}
The ancestors are optional. For ``\verb#multMagma#'',\index{multMagma@\texttt{multMagma}} it has one ancestor
(namely, ``\verb#1-sorted#''\index{1-sorted@\texttt{1-sorted}}). The fields for a structure look like a
comma-separated list whose entries are of the form:
\begin{equation}
  \langle\textit{field}\rangle ::= \langle\textit{identifier}\rangle\quad\mathtt{->}\quad\langle\textit{type}\rangle
\end{equation}
We see that ``\texttt{multMagma}'' has two fields:
\begin{enumerate}
\item ``\texttt{carrier}'' which is a set
\item ``\texttt{multF}'' which is a ``\texttt{BinOp}'' (binary operator)
  acting on the ``\texttt{carrier}''.
\end{enumerate}
This matches Bourbaki's definition of a magma.

\N{Mizar's definition of Group}
Mizar formalizes a group following the style of
Definition~\ref{defn:introduction:mizar-style-group}: it's a magma which
satisfies a bunch of extra properties (as opposed to a magma equipped
with more structure). Mizar's notion of ``extra properties'' is handled
by a mechanism called \define{Attributes}\index{Attribute!Mizar}, which
are adjectives we can tack onto types and they encode certain axioms or
formulas which must hold. There are three such attributes being defined
immediately in \texttt{group\_1.miz}:
\begin{enumerate}
\item ``\texttt{unital}''\index{unital@\texttt{unital}} which asserts
  $\exists_{e{:}\mathtt{IT}}\forall_{h{:}\mathtt{IT}} \mu(h,e)=h\land\mu(e,h)=h$
\item ``\texttt{Group-like}''\index{Group-like@\texttt{Group-like}},
  which extends the condition of ``\texttt{unital}'' with the assertion
  that for any element $h$ and inverse element $g$ exists
  $\exists_{e{:}\mathtt{IT}}\forall_{h{:}\mathtt{IT}} \mu(h,e)=h\land\mu(e,h)=h\land\exists_{g{:}\mathtt{IT}}\mu(h,g)=e\land\mu(g,h)=e$
\item ``\texttt{associative}''\index{associative@\texttt{associative}}
  which corresponds to our demand of associativity
  $\forall_{x,y,z{:}\mathtt{IT}}\mu(\mu(x,y),z)=\mu(x,\mu(y,z))$.
\end{enumerate}
With these attributes in hand, Mizar defines a group to be a
``\texttt{Group-like associative non empty multMagma}'', i.e., a magma such
that it is associative and group-like (has an identity element and for
each element, a corresponding inverse element exists). This is collected
in a new type, or \define{Mode}\index{Mode!in Mizar} (as Mizar calls them).

\begin{mizar}
:: group_1.miz
definition
  let IT be multMagma;

  attr IT is unital means
:: GROUP_1:def 1
  ex e being Element of IT st for h being
  Element of IT holds h * e = h & e * h = h;

  attr IT is Group-like means
:: GROUP_1:def 2
  ex e being Element of IT st for h being Element of IT
  holds h * e = h & e * h = h &
        ex g being Element of IT st h * g = e & g * h = e;

  attr IT is associative means
:: GROUP_1:def 3
  for x,y,z being Element of IT holds (x*y )*z = x*(y*z);
end;

definition
  mode Group is Group-like associative non empty multMagma;
end;
\end{mizar}

\M Observe that a more faithful translation of Mizar's definition would
be:

\begin{quote}
  \textbf{Definition.} A \emph{Group} $\langle G,\cdot\rangle$ is a set of
  elements with a binary operator $\cdot\colon G\times G\to G$
  satisfying the following conditions:
  \begin{enumerate}
  \item \textbf{Associativity:} for all $a,b,c\in G$, we have $(a\cdot b)\cdot c = a\cdot(b\cdot c)$
  \item \textbf{Identity element:} There exists an element $e$ such that
    $e\cdot a= a\cdot e = a$ for all $a\in G$
  \item \textbf{Inverse Element:} For all $a\in G$, there exists an
    element $b\in G$ with $a\cdot b=b\cdot a = e$.
  \end{enumerate}
\end{quote}

We often work with multiple groups, and if we want to refer to the
identity element of a particular group, then in ordinary mathematics we
write either $1_{G}$ (if $G$ uses multiplicative notation) or $0_{G}$
(if $G$ uses addition). Mizar has a way to encode this convention, which
is the next definition in \texttt{group\_1.miz}:

\begin{mizar}
definition
  let G be multMagma such that
A1: G is unital;
  func 1_G -> Element of G means
:: GROUP_1:def 4
  for h being Element of G holds h * it = h & it * h = h;
  existence by A1;
  uniqueness
  proof
    let e1,e2 be Element of G;
    assume that
A2: for h being Element of G holds h * e1 = h & e1 * h = h and
A3: for h being Element of G holds h * e2 = h & e2 * h = h;
    thus e1 = e2 * e1 by A3
           .= e2 by A2;
  end;
end;
\end{mizar}

\M What's going on here? Well, for any unital magma $G$, we have a
``functor'' (a logical function, i.e., parametrized term) denoted
``\verb#1_#'' which produces the unit element for the given magma.
Whereas (in print) mathematicians would write $1_{G}$, in Mizar we would
write ``\verb#1_G#''.

Mizar demands we prove such a thing exists, and then a proof of
uniqueness is supplied.

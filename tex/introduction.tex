\chapter{Group Theory in Mizar}

\section{Basic Definitions}

\M
Recall, there are two ways to define a group. I will make bold the
differences:

\begin{definition}[Version 1]\label{defn:introduction:mizar-style-group}
  A \emph{group} consists of
  \begin{enumerate}
  \item a set $G$
  \end{enumerate}
  equipped with
  \begin{enumerate}
  \item a binary operator $\mu\colon G\times G$
  \end{enumerate}
  such that
  \begin{enumerate}
  \item Associativity: for any $g_{1}$, $g_{2}$, $g_{3}\in G$, we have
    $\mu(g_{1}, \mu(g_{2},g_{3})) = \mu(\mu(g_{1},g_{2}),g_{3})$
  \item \textbf{Existence of Unit: there exists an element $1\in G$ such
    that for any $g\in G$ we have $\mu(g,1)=\mu(1,g)=g$}
  \item \textbf{Existence of inverses: for each $g\in G$, there is an
    $h\in G$ such that $\mu(g,h)=1$}
  \end{enumerate}
\end{definition}

\begin{definition}[Version 2]
  A \emph{group} consists of
  \begin{enumerate}
  \item a set $G$
  \end{enumerate}
  equipped with
  \begin{enumerate}
  \item a binary operator $\mu\colon G\times G$
  \item \textbf{a unit $1\in G$}
  \item \textbf{an inverse operator $\iota\colon G\to G$}
  \end{enumerate}
  such that
  \begin{enumerate}
  \item Associativity: for any $g_{1}$, $g_{2}$, $g_{3}\in G$, we have
    $\mu(g_{1}, \mu(g_{2},g_{3})) = \mu(\mu(g_{1},g_{2}),g_{3})$
  \item \textbf{Unit laws: for any $g\in G$ we have $\mu(g,1)=\mu(1,g)=g$}
  \item \textbf{Inverse law: for each $g\in G$, $\mu(\iota(g),g)=1$}
  \end{enumerate}
\end{definition}

\M
As Baez~\cite[week one]{baez2004qg-lectures} notes, these two different
definitions give rise to \emph{different} notions of ``morphism'' (but
identical notions of ``isomorphism'').

Mizar takes the first version as its definition. Implicit in its
definition is a notion of a ``\verb#multMagma#'', a magma with a
multiplication operator (defined in \verb#ALGSTR_0#). As Nakasho and
friends explicate~\cite{nakasho2014formalization}, a ``\verb#multMagma#''
is a magma as could be found in, e.g., Bourbaki's \emph{Algebra}: it
consists of a set (which Mizar calls its ``\verb#carrier#'') and a
binary operator ``\verb#multF#''.

\begin{mizar}
:: algstr_0.miz
definition
  struct (1-sorted) multMagma (# carrier -> set,
  multF -> BinOp of the carrier
  #);
end;
\end{mizar}

\M
Mizar formalizes a group following the style of
Definition~\ref{defn:introduction:mizar-style-group}. 

\begin{mizar}
:: group_1.miz
definition
  let IT be multMagma;

  attr IT is unital means
:: GROUP_1:def 1
  ex e being Element of IT st for h being
  Element of IT holds h * e = h & e * h = h;
  
  attr IT is Group-like means
:: GROUP_1:def 2
  ex e being Element of IT st for h being Element of IT
  holds h * e = h & e * h = h &
        ex g being Element of IT st h * g = e & g * h = e;

  attr IT is associative means
:: GROUP_1:def 3
  for x,y,z being Element of IT holds (x*y )*z = x*(y*z);
end;

definition
  mode Group is Group-like associative non empty multMagma;
end;
\end{mizar}
  
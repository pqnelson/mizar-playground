\section{Higher-Order Logic in Mizar}

\begin{ddanger}
The reader may skim or even skip this section, and return to it when
encountering a scheme either in the Mizar mathematical library, or later
in this text. 
\end{ddanger}

\bigbreak
Mizar provides a minimal facility for a fragment of higher-order
logic. They're called \emph{schemes} in Mizar. The intuition is that
when we ``abstract away'' predicates and functors in theorems, we get a
scheme. They look like:
\begin{quote}
\noindent\texttt{scheme} \textit{label} \verb#{# \emph{parameters} \verb#}# \texttt{:}

\noindent\qquad\emph{statement}

\noindent\texttt{provided}

\noindent\qquad\emph{statements}

\noindent\texttt{proof}

\noindent\qquad\dots

\noindent\texttt{end;}
\end{quote}
Just as theorems have premises or hypotheses, schemes have ``provisions'' ---
statements made in the ``\lstinline{provided}'' section. They are
statements (optionally labeled), separated by ``\lstinline{and}''. These
statements are referred to as the scheme's premises.

Theorems make claims (consequents), which correspond to the scheme's
conclusion (i.e., the very first statement made in a scheme, on line 2
of our schematic scheme). The proof associated to the scheme establishes
the conclusion deduced from given the premises.

The \emph{parameters} of a scheme are the functors and predicates we
abstract away. Functor parameters look like
``\emph{identifier}\verb#(#\emph{params}\verb#) -> #\emph{type}''. For
the special case when we want to abstract away a constant, we would use
a functor parameter with an empty parameter list, e.g.,
``\lstinline{G() -> Group}'' for ``some group'' \lstinline{G()}.
Predicate parameters look like ``\emph{parameter}\verb#[#\emph{params}\verb#]#''.
``Wildcard arguments'' of a predicate parameter would be just
``\lstinline{set}'', since everything is a set in Mizar.

We use schemes as we use theorems: to justify a step in a proof. Unlike
theorems, we ``cite'' a scheme using different notation. For a scheme
named ``\lstinline{Sch}'' with several provisions, we use it to justify
a proof step by writing ``\lstinline{from Sch(P1,P2,...,Pn)}'' where
\verb#P1# through \verb#Pn# refers to local labels in the proof from
which we appeal to the scheme. They correspond to the scheme's premises,
establishing the applicability of the scheme.

Note: it is necessary to use local functor and predicate definitions
when using a scheme. That is to say, we need to use
``\lstinline{defpred P[object] means ...}'' and
``\lstinline{deffunc F(...) = ...}''. An additional caveat, in these
local definitions, we use ``\lstinline{$1}'' to refer to the first
argument, ``\lstinline{$2}'' to refer to the second argument, and so on.

An example of schemes appearing ``in the wild'' would be when we
have a class of subgroups $\mathcal{F}$ satisfying some property $P[-]$
and are closed under automorphism, then $\bigcap\mathcal{F}$ is a
characteristic subgroup. Here, there are two parameters to the scheme:
one is the obvious unary predicate $P[-]$, the other is the group object
which would be \verb#G() -> Group#.
Another example would be the $\mathcal{X}$-residual of a group (c.f.,
Definition~\ref{defn:pure-math:X-residual}).

In my experience, schemes are only an ``obvious'' solution when I
realize I'm using ``the same argument'' with different
predicates. Usually, after I've written a theorem and its proof, and
I've just formulated a similar theorem with an analogous proof. That's
when I say, ``Aha, now I can go back and abstract away my previous
theorem as a scheme!'' A good heuristic, I think, is to generically rely
on the programmer's idiom YAGNI when it comes to schemes: you ain't
gonna need it.
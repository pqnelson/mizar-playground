\section{First-Order Logic in Mizar}

\subsection{Propositional Logic}
We will briefly review Mizar's syntax for formulas.

\begin{equation}
  \neg\Phi\quad\equiv\quad\mathtt{not}~\Phi
\end{equation}
\begin{equation}
  \Phi\lor\Psi\quad\equiv\quad\Phi~\mathtt{or}~\Psi
\end{equation}
\begin{equation}
  \Phi\land\Psi\quad\equiv\quad\Phi~\mathtt{\&}~\Psi
\end{equation}
\begin{equation}
  \Phi\implies\Psi\quad\equiv\quad\Phi~\mathtt{implies}~\Psi
\end{equation}
\begin{equation}
  \Phi\iff\Psi\quad\equiv\quad\Phi~\mathtt{iff}~\Psi
\end{equation}

\subsection{Typing Statements as Propositions}
Unique to Mizar is that typing judgements are object language
propositions. When we write ``\texttt{x is natural number}'', that's a
proposition. Type theory would make it a judgement, a proposition
\emph{in the metalanguage}. Mizar makes it a proposition in the
\emph{object} language. That is to say, if $t$ is a term in Mizar and
$\mathcal{T}$ is a Mizar type, then
\begin{equation}
  t~\mathtt{is}~\mathcal{T}
\end{equation}
is a proposition in Mizar.

\subsection{Quantifiers}
Quantifiers are a bit trickier. We quantify a variable over a type.
\begin{equation}
\forall_{x\esti A}\Phi\quad\equiv\quad\mathtt{for}~x~\mathtt{being}~A~\mathtt{holds}~\Phi
\end{equation}
\begin{equation}
\exists_{x\esti A}\Phi\quad\equiv\quad\mathtt{ex}~x~\mathtt{being}~A~\mathtt{st}~\Phi
\end{equation}
We can also just ``stack'' multiple quantifiers next to each other, like
\begin{equation}
\forall_{x\esti A}\forall_{y\esti B}\Phi\quad\equiv\quad\mathtt{for}~x~\mathtt{being}~A~\mathtt{for}~y~\mathtt{being}~B~\mathtt{holds}~\Phi.
\end{equation}
We have syntactic sugar for universal quantification of an implication:
\begin{equation}
\forall_{x\esti A}\Phi\implies\Psi\quad\equiv\quad\mathtt{for}~x~\mathtt{being}~A~\mathtt{st}~\Phi~\mathtt{holds}~\Psi.
\end{equation}

\begin{exercise}
  Translate into Mizar ``Every element $x\in X$ corresponds to a unique
  $y\in Y$ such that $\Phi$ holds.''
\end{exercise}

\subsection{Referring to Labels}\label{par:introduction:referring-to-labels}
Mizar can refer to theorems from another article, say
``\verb#ARTICLE#'', by writing ``\verb#ARTICLE:56#'' to refer to theorem 56.
Definitions are referred to using ``\verb#ARTICLE:def xx#'' (where
``\verb#xx#'' is a number). \emph{It is a common beginner's mistake to
try to refer to theorem $X$ by writing } ``\verb#ARTICLE:ThX#''.

One quirk when referring to multiple theorems from the same article, we
can simply separate them by commas. For example ``\verb#ARTICLE:x,y,z#''
refers to theorems {\tt x}, {\tt y}, and {\tt z} from ``\verb#ARTICLE#''.
Similarly, we can do likewise for definitions. For example,
``\verb#ARTICLE:def a,b,c#'' refers to definitions {\tt a}, {\tt b},
{\tt c} from ``\verb#ARTICLE#''.

Informally, outside of Mizar code, I may write ``\verb#ARTICLE:ThX#'' to
refer to Theorem {\tt X} from ``\verb#ARTICLE#''. When citing the Mizar
mathematical library, I follow the Mizar citation conventions. For
example, a citation like \mml[def10]{xboole0} refers to definition 10
from \verb#XBOOLE_0#. The only differences are that I use small-caps (for
visual ease), I sometimes prepend ``Th'' before the theorem number, and
I sometimes omit a space between ``def'' and the number.

\subsection{Final remark on notations used in Mizar}
There are a few quirky choices of notation which I wish to discuss, just
to let the reader be aware of it. I will also remind the reader when it
comes up in the future:\footnote{Shinshu University has a webpage giving
a ``rosetta stone'' for Mizar notation and the usual mathematical
notation, see: \url{http://markun.cs.shinshu-u.ac.jp/kiso/projects/proofchecker/mizar/mizardictionary1.htm}}

\begin{enumerate}
\item\index{\texttt{f.x}!Mizar for $f(x)$} In mathematics, function application is written $f(x)$, but in
  Mizar it is ``\texttt{f.x}''
\item\index{\texttt{f.:A}!Mizar for $f(A)$} Applying a function to a set is denoted $f(A)$, but in Mizar it is
  denoted ``\texttt{f.:A}''
\item\index{\texttt{f''}@{\texttt{f"\""}}!Mizar for $f^{-1}$} The inverse function relating $y=f(x)$ to $f^{-1}(y)=x$, in Mizar
  is denoted by ``\texttt{f".y}''. More generally, instead of using
  superscript $-1$, Mizar will use double quotes.
\item\index{\texttt{f \mid A}@{\texttt{f"|"A}}!Mizar for $f\vert_{A}$} If $f\colon X\to Y$ is a function of sets, and $A\subset X$, then
  we write the restriction of $f$ to $A$ as $f|_{A}\colon A\to X$
  defined by $\forall a\in A, f(a)=f|_{A}(a)$. In Mizar, this
  restriction is denoted ``\lstinline{f|A}'' and function application with it
  is ``\lstinline{(f|A).x}'' for $f|_{A}(x)$.
\item\index{\texttt{h in H}!Mizar for $h\in H$} Set membership $x\in X$ is
  denoted in Mizar by the formula ``\verb#x in X#''.
\end{enumerate}

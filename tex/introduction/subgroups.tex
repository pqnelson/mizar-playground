\section{Subgroups}

\subsection{Helper notation}
We have $A,B\subset G$ be subsets of the carrier set underlying a
non-empty magma $G$. Then we can define ``\lstinline{A*B}'' as the set of
elements of the form $ab$ for $a\in A$ and $b\in B$. This was done in
\mml[def 2]{group2}\MizDef{GROUP\_2}[02]{2}. Similarly (in definitions
3 and 4\MizDef{GROUP\_2}[03]{3}\MizDef{GROUP\_2}[04]{4}) we can define
for any subset $A\subset G$ and any $g\in G$ the product $Ag$ and $gA$
in the obvious way.

For any subgroup $H\subgroup G$ and subset $A\subset G$,
\mml[def 11]{group2}\MizDef{GROUP\_2}{11} defines \lstinline{A*H} and
\lstinline{H*A} \emph{as subsets of the carrier of $G$} using the set
underlying $H$, then using the earlier definitions (from
\mml[def 2]{group2}).

Similarly, for any subgroup $H\subgroup G$ and $g\in G$, we have
\lstinline{g*H} and \lstinline{H*g} be defined using the underlying set of $H$.

\subsection{Subgroup}
For a group $G$, \mml[def 5]{group2}\MizDef{GROUP\_2}[05]{5} defines a
``\lstinline{Subgroup of G}'' to be a Group-like non-empty \lstinline{multMagma}
such that its carrier is a subset of the underlying set of $G$, and the
Subgroup's multiplication operator is the multiplication from $G$
restricted to the Subgroup.

\begin{notation}
We denote $H\subgroup G$ for the statement ``$H$ is a subgroup of $G$''.
\end{notation}

\subsection{Equality of Subgroups}\label{par:introduction:subgroup-equality}
For reasons which are not well documented, \mml[def 6]{group2}
redefines equality for \emph{strict} Subgroups. Non-strict subgroups
cannot be equal (not as subgroups).

There are a few useful theorems
concerning equality of subgroups, but the idiomatic Mizar code seems to
be to assert ``\lstinline[breaklines=true]{the multMagma of H1 = the multMagma of H2}''.
When the user knows both subgroups are strict, they can set ``\lstinline{H1 = H2}''.



\begin{aside}[On ``strict'' mathematical gadgets]\index{strict@\lstinline{strict}|textbf}
Mizar has a notion of a ``\lstinline{strict}'' attribute. What does it mean?
Well, sometimes a mathematical gadget is ``composite''; like a vector
space over a field is an Abelian group with vector addition, but it also
has other stuff (like scalar multiplication). This ``other stuff''
prevents it from being a ``bald-faced Abelian group''. Just as black
holes have no hair, strict groups have no hair: it's just a
``\lstinline{multMagma}'' satisfying the group properties (in Mizar, at
least).

We can always coerce a gadget into a strict version by manually
constructing it from the relevant fields. For example, given any group
$G$, we can construct a strict group as follows:
\begin{mizar}
:: given some non-strict Group G
reconsider strict_G = multMagma (# the carrier of G,
  the multF of G #) as strict Group;
:: now use strict_G to prove results
\end{mizar}
\end{aside}

\subsection{Trivial Subgroups}\index{$\trivialSubgroup_{G}$}
For any group $G$, $\trivialSubgroup_{G}$ is the trivial subgroup
consisting of the identity element \lstinline{1_G} and nothing else. This is
defined in \mml[def 7]{group2}\MizDef{GROUP\_2}[07]{7} and denoted \lstinline{(1).G}\mizindex{1.G@$\mathtt{(1).G}$}. The other
trivial subgroup is denoted by mathematicians as $G$ itself, but Mizar
distinguishes this trivial improper subgroup as $\Omega_{G}$\index{$\Omega_{G}$} in
\mml[def 8]{group2}\MizDef{GROUP\_2}[08]{8}.\mizindex{Omega.G@$\mathtt{(Omega).G}$}

\subsection{Intersection of Subgroups}
If $H_{1}\subgroup G$ and $H_{2}\subgroup G$, then $H_{1}\cap H_{2}$ is
a subgroup of $G$, too. Mizar denotes this by
\lstinline{H1 /\ H2}\index{\lstinline{H1 /\ H2}}. Its
defined in \mml[def 10]{group2}\MizDef{GROUP\_2}{10} as the group whose
underlying set is the intersection of the sets underlying $H_{1}$ and $H_{2}$.

\subsection{Cosets}
For any subgroup $H\subgroup G$, we can consider the left (or right)
cosets of $H$ as the family of subsets of $G$ of the form
$\{gH|g\in G\}$ (resp., $\{Hg|g\in G\}$). This is defined in
\mml[def 15,16]{group2}.\MizDef{GROUP\_2}{15}\MizDef{GROUP\_2}{16}\mizindex{Right Cosets@$\mathtt{Right\_Cosets}$}\mizindex{Left Cosets@$\mathtt{Left\_Cosets}$}

This naturally leads to a notion of the index of a subgroup $H\subgroup G$,
which Mizar defines as \lstinline{Index H}\mizindex{Index@\lstinline{Index}} in
\mml[def 17]{group2}\MizDef{GROUP\_2}{17} as the
cardinality of the collection of left cosets of $H$ in $G$. This is
proven to be the same as the same definition using right cosets in
\mml[Th 145]{group2}.\MizThm{GROUP\_2}{145}

More generally, the collection of all possible subgroups of $G$ is given
in \mml[def 1]{group3}\MizDef{GROUP\_3}[01]{1} as
``\lstinline{Subgroups G}''\mizindex{Subgroups@\lstinline{Subgroups}}.

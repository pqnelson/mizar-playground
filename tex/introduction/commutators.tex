\section{Commutators}

\begin{definition}[{\mml[def 2,3]{group5}}]\index{Commutator}
  Let $G$ be a group and let $x$, $y\in G$ be arbitrary.
  Then the \define{Commutator} of $x$ with $y$ is
  \begin{equation*}
    [x,y] = x^{-1}y^{-1}xy.
  \end{equation*}
  We abuse notation and write $[x,y,z] = [{[x,y]},z]$,
  $[w,x,y,z] = [{[w,x,y]},z]$, and so on.
\end{definition}

\begin{def-remark}
Mizar defines its commutators, $[x,y]$ in \mml[def 2]{group5}\MizDef{GROUP\_5}[002]{2}, and
$[x,y,z]$ in \mml[def 3]{group5}.\MizDef{GROUP\_5}[003]{3}
\end{def-remark}

\begin{def-remark}
This convention, while used in Mizar (see Definition~\mml[def 2]{group5})
and finite group theory, clashes with the convention with Lie groups
where they have $[X,Y]_{\text{Lie}}=XYX^{-1}Y^{-1}$ to have the Lie
bracket for the corresponding Lie algebra coincide with the commutator
$[x,y] = xy - yx$. However, one advantage of the finite group version is
that $[g,h] = g^{-1}g^{h}$ where $g^{h}=h^{-1}gh$ is conjugation of $g$
by $h$. \emph{Extreme care must be taken when working with finite groups
of Lie type!}
\end{def-remark}

\begin{notation}
  Mizar denotes the commutator $[x,y]$ as \lstinline![.x,y.]! where
  ``\verb#[.#'' and ``\verb#.]#'' are left and right functor brackets
  introduced in the \mml{xxreal1} vocabulary.
\end{notation}

\begin{definition}\index{$G'$}\index{Subgroup!Derived}\index{$[G,G]$}
Let $G$ be a group. The \define{Derived Subgroup} of $G$ is the subgroup
denoteed $G'$ or $[G,G]$ generated by commutators of elements of $G$, where
for generic subsets $X,Y\subset G$ we denote
\begin{equation}
  [X,Y] = \langle [x,y] : x\in X,y\in Y\rangle.
\end{equation}
We also use the notation $[X,Y,Z] = [\,{[X,Y]},Z]$, $[W,X,Y,Z] = [\,{[W,X,Y]},Z]$,
and so on. In particular, $[X,Y,Z] = \langle [a,z] : a\in[X,Y], z\in Z\rangle$.
\end{definition}

\begin{def-remark}
This is defined for arbitrary subsets $A\subset G$ and $B\subset G$ in
\mml[def 7]{group5},\MizDef{GROUP\_5}[07]{7} and for subgroups $H_{1}\subgroup G$ and
$H_{2}\subgroup G$ in \mml[def 8]{group5}.\MizDef{GROUP\_5}[08]{8} The derived subgroup is
denoted ``\lstinline{G `}'' (that is the letter G followed by a
backtick/back-quote [grave accent]) and defined in \mml[def 9]{group5}.\MizDef{GROUP\_5}[09]{9}\mizindex{G'@\lstinline{G`}}
\end{def-remark}

\begin{def-remark}
  Mizar does not define $[X,Y,Z]$ for subgroups $X,Y,Z\subgroup G$.
\end{def-remark}

\begin{theorem}[{\mml[Th23]{group5}}]
For any $x,y,z\in G$, we have $[x,y]^{z} = [x^{z},y^{z}]$.
\end{theorem}

\begin{theorem}[{\mml[Th25,26]{group5}}]
  For any $x,y,z\in G$, we have $[xy,z] = [x,z]^{y}[y,z]$ and
  $[x,yz] = [x,z] [x,y]^{z}$.
\end{theorem}

\begin{theorem}[{Hall--Witt identity, \mml[Th46]{group5}}]
  For any $x,y,z\in G$ we have
  \begin{equation}
    [x, y^{-1}, z]^y\cdot[y, z^{-1}, x]^z\cdot[z, x^{-1}, y]^x = 1.
  \end{equation}
\end{theorem}

\begin{theorem}[Hall's Three Subgroup Lemma]
  Let $H$, $K$, $L$ be subgroups of $G$.
  If $[H,K,L]=\trivialSubgroup$ and $[K,L,H]=\trivialSubgroup$,
  then $[L,H,K]=\trivialSubgroup$.
\end{theorem}

This was not proven in Mizar, but it seems do-able.


\begin{definition}[{\mml[def 10]{group5}}]\mizindex{Center@\lstinline{center}|see{$\mathtt{GROUP\_5:def 10}$}}\index{$Z(G)$}\index{Center}\index{Group!Center}
Let $G$ be a group. Its \define{Center} is the Subgroup $Z(G)\subgroup G$
consisting of all elements which commute with all of $G$:
\begin{equation*}
Z(G) = \{z\in Z(G) | \forall g\in G, zg=gz\}
\end{equation*}
\end{definition}

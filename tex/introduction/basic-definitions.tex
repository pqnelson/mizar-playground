\section{Basic Definitions}

We will review the definitions germane to group theory, as inspired
by Nakasho and friends~\cite{nakasho2014formalization}. This entire
section is inspired by their paper.

Recall, as Baez~\cite[week one]{baez2004qg-lectures}\index{Stuff, Structure, Properties} notes, there are
two ways to define a group. These two different definitions give rise to
\emph{different} notions of ``group homomorphism'' (but identical
notions of ``group isomorphism''). I will make bold the differences:

\begin{definition}[Version 1]\label{defn:introduction:mizar-style-group}
  A \emph{group} consists of
  \begin{enumerate}
  \item a set $G$
  \end{enumerate}
  equipped with
  \begin{enumerate}
  \item a binary operator $\mu\colon G\times G\to G$
  \end{enumerate}
  such that
  \begin{enumerate}
  \item Associativity: for any $g_{1}$, $g_{2}$, $g_{3}\in G$, we have
    $\mu(g_{1}, \mu(g_{2},g_{3})) = \mu(\mu(g_{1},g_{2}),g_{3})$
  \item \textbf{Existence of Unit: there exists an element $1\in G$ such
    that for any $g\in G$ we have $\mu(g,1)=\mu(1,g)=g$}
  \item \textbf{Existence of inverses: for each $g\in G$, there is an
    $h\in G$ such that $\mu(g,h)=1$}
  \end{enumerate}
\end{definition}

\begin{definition}[Version 2]
  A \emph{group} consists of
  \begin{enumerate}
  \item a set $G$
  \end{enumerate}
  equipped with
  \begin{enumerate}
  \item a binary operator $\mu\colon G\times G\to G$
  \item \textbf{a unit $1\in G$}
  \item \textbf{an inverse operator $\iota\colon G\to G$}
  \end{enumerate}
  such that
  \begin{enumerate}
  \item Associativity: for any $g_{1}$, $g_{2}$, $g_{3}\in G$, we have
    $\mu(g_{1}, \mu(g_{2},g_{3})) = \mu(\mu(g_{1},g_{2}),g_{3})$
  \item \textbf{Unit laws: for any $g\in G$ we have $\mu(g,1)=\mu(1,g)=g$}
  \item \textbf{Inverse law: for each $g\in G$, $\mu(\iota(g),g)=1$}
  \end{enumerate}
\end{definition}

\begin{def-remark}[Terminology: ``Morphisms'']
I will adopt modern terminology, and simply refer to a group
homomorphism as a ``\emph{group morphism}'', or more tersely as just a
morphism.
\end{def-remark}

\index{Magma|(} Mizar takes the first version as its definition.
\emph{In fact, a Group in Mizar is literally ``just a Magma'' satisfying some extra properties.} Implicit in its
definition is a notion of a ``\lstinline{multMagma}'', a magma with a
multiplication operator (defined in \mml{algstr0}). As Nakasho and
friends explicate~\cite{nakasho2014formalization}, a ``\lstinline{multMagma}''
is a magma as could be found in, e.g., Bourbaki's\index{Bourbaki, Nicholas}
\emph{Algebra}~\cite[see Ch.I\S1.1]{bourbaki1974elements}: it
consists of a set (which Mizar calls its ``\lstinline{carrier}'') and a
binary operator ``\lstinline{multF}''.\index{multF@\texttt{multF}}\index{carrier@\texttt{carrier}}\index{multMagma@\texttt{multMagma}}

Note: comments in Mizar are started by ``\lstinline{::}'' and continue until
the end of the line.

\begin{mizar}
:: algstr_0.miz
definition
  struct (1-sorted) multMagma (# carrier -> set,
  multF -> BinOp of the carrier
  #);
end;
\end{mizar}

Let us unpack what is going on in this code. We are defining a
\lstinline{struct},\index{Struct@\texttt{struct}} a mathematical structure
which is usually presented in ordinary mathematics as a tuple of some
kind $\struct{\dots}$. For example, a field might be presented as
$\struct{F, +, \cdot, 0, 1}$ consisting of a set $F$, a couple binary
operators, and a couple of selected elements of $F$.

Mizar's grammar for defining a structure is:
\begin{equation}
  \langle\textit{structure-definition}\rangle ::=
  \mbox{\texttt{struct}}~(\langle\textit{ancestors}\rangle)~\langle\textit{symbol}\rangle~
  \mbox{\texttt{(\#}}\quad
  \langle\textit{fields}\rangle\quad
  \mbox{\texttt{\#)}}
  \texttt{;}
\end{equation}
The ancestors are optional. For ``\lstinline{multMagma}'',\index{multMagma@\texttt{multMagma}} it has one ancestor
(namely, ``\lstinline{1-sorted}''\index{1-sorted@\texttt{1-sorted}}). Recall
from first-order logic that a one-sorted
structure\index{Structure!One-Sorted}\mizindex{One-Sorted Structure@\texttt{1-sorted}} means we have
\emph{one set} equipped with extra gadgetry (as opposed to a two-sorted
structure\index{Structure!Two-Sorted} which consists of \emph{two sets} equipped with gadgetry, like
a topological space consisting of a set of points and a set of neighborhoods).
The fields for a structure look like a
comma-separated list whose entries are of the form:
\begin{equation}
  \langle\textit{field}\rangle ::= \langle\textit{identifier}\rangle\quad\mbox{\texttt{->}}\quad\langle\textit{type}\rangle
\end{equation}
We see that ``\texttt{multMagma}'' has two fields:
\begin{enumerate}
\item ``\texttt{carrier}'' which is a set
\item ``\texttt{multF}'' which is a ``\texttt{BinOp}'' (binary operator)
  acting on the ``\texttt{carrier}''.
\end{enumerate}
This matches Bourbaki's definition of a magma.\index{Magma|)}

\begin{notation}[Accessing fields of a structure]
We can refer to, e.g., the ``\lstinline{multF}'' of a given instance of a
``\lstinline{multMagma}'' by writing
``\lstinline{the multF of <instance>}''. These are called accessor
functors. Sometimes we want to access ``the underlying magma'' of an
algebraic gadget, and a horrible pun used to describe this is the
\emph{forgetful functor}\index{Functor!Forgetful}. We discuss more about
functors later on (\S\ref{par:introduction:functor-one-g}).
\end{notation}

\begin{aside}[Multiplication operator]\label{rmk:introduction:multiplication-operator}\index{\texttt{x * y}}
Before continuing on too far, I should note that for a
``\texttt{M being multMagma}'' we have an infix binary operator
``\texttt{x*y = (the multF of M).(x,y)}''. This is defined in
Definition~\mml[def19]{algstr0}.\MizDef{ALGSTR\_0}{19}
\end{aside}

\subsection{Equality of Instances of Structures}
If we a structure $\struct[Foo]{\mathsf{F}_{1},\dots,\mathsf{F}_{n}}$
with $n$ fields $\mathsf{F}_{j}$, then we can make an instance of it
using notation ``\verb$Foo(# F1, ..., Fn #)$'' where ``\verb$(#$'' is an
ASCII approximation to ``$\langle\!\langle$'', and ``\verb$#)$'' approximates
``$\rangle\!\rangle$''. Equality of two
instances follows the same axiomatic behavior as ordered $n$-tuples:
\verb$Foo(# E1, ..., En #) = Foo(# F1, ..., Fn #)$ if and only if each
pair of fields are equal
\lstinline{E1} = \lstinline{F1}, \dots, and \lstinline{En} = \lstinline{Fn}.
If we tried to assert two \emph{different} structures are equal --- e.g.,
\verb$Foo(# E1, ..., En #) = Bar(# F1, ..., Fm #)$ --- well, this is not
allowed, Mizar will reject it.

One caveat to this is if we have, for example, a topological group
$G_{1}$ (which has all the fields as a group \emph{and} a topological
space \emph{combined}) and, say, $S_{n}$ (which has all the fields of a
group, but nothing else), then we cannot meaningfully compare these
two. We will need to do something like write ``\lstinline{the Group of G1 = Sn}''
to compare the \emph{group structure} of $G_{1}$ to the \emph{Group structure}
of $S_{n}$.

\subsection{Mizar's definition of Group}
Mizar formalizes a group following the style of
Definition~\ref{defn:introduction:mizar-style-group}: it's a magma which
satisfies a bunch of extra properties (as opposed to a magma equipped
with more structure). Mizar's notion of ``extra properties'' is handled
by a mechanism called \define{Attributes}\index{Attribute!Mizar}, which
are adjectives we can tack onto types and they encode certain axioms or
formulas which must hold. There are three such attributes being defined
immediately in \mml{group1}:
\begin{enumerate}
\item ``\lstinline{unital}''\index{unital@\texttt{unital}} which asserts
  $\exists_{e\esti \mathtt{IT}}\forall_{h\esti \mathtt{IT}} \mu(h,e)=h\land\mu(e,h)=h$
\item ``\lstinline{Group-like}''\index{Group-like@\texttt{Group-like}},
  which extends the condition of ``\lstinline{unital}'' with the assertion
  that for any element $h$ and inverse element $g$ exists
  $\exists_{e\esti \mathtt{IT}}\forall_{h\esti \mathtt{IT}} \mu(h,e)=h\land\mu(e,h)=h\land\exists_{g\esti \mathtt{IT}}\mu(h,g)=e\land\mu(g,h)=e$
\item ``\lstinline{associative}''\index{associative@\texttt{associative}}
  which corresponds to our demand of associativity
  $\forall_{x,y,z\esti \mathtt{IT}}\mu(\mu(x,y),z)=\mu(x,\mu(y,z))$.
\end{enumerate}
With these attributes in hand, Mizar defines a group to be a
``\lstinline{Group-like associative non empty multMagma}'', i.e., a magma such
that it is associative and group-like (has an identity element and for
each element, a corresponding inverse element exists). This is collected
in a new type, or \define{Mode}\index{Mode!in Mizar} (as Mizar calls them).

\index{Unital@\texttt{unital}|(}\index{Group-like@\texttt{Group-like}|(}\index{Associative@\texttt{associative}|(}
\begin{mizar}
:: group_1.miz
definition
  let IT be multMagma;

  attr IT is unital means
:: GROUP_1:def 1
  ex e being Element of IT st for h being
  Element of IT holds h * e = h & e * h = h;

  attr IT is Group-like means
:: GROUP_1:def 2
  ex e being Element of IT st for h being Element of IT
  holds h * e = h & e * h = h &
        ex g being Element of IT st h * g = e & g * h = e;

  attr IT is associative means
:: GROUP_1:def 3
  for x,y,z being Element of IT holds (x*y )*z = x*(y*z);
end;

definition
  mode Group is Group-like associative non empty multMagma;
end;
\end{mizar}
\index{Associative@\texttt{associative}|)}\index{Group-like@\texttt{Group-like}|)}\index{Unital@\texttt{unital}|)}\index{Group@\texttt{Group}|textbf}

\begin{idiom}\label{rmk:introduction:idiom-it-in-attributes}\index{Idiom!IT in Attributes}\index{IT@\texttt{IT}}
Whenever defining an attribute in Mizar, we have an ambient
``\lstinline{IT}'' object for which the attribute will be defined upon. For
example, these attributes appearing in \mml[def 1]{group1} are defined
on a \lstinline{multMagma}.
\end{idiom}

\begin{remark}[Registering attributes as adjectives]\index{Registration}
Once we have defined attributes, we cannot use them as adjectives for
types until we ``\lstinline{register}'' them. Without registration, writing
something like ``\lstinline{let x be associative multMagma}'' would confuse
Mizar, and it would throw an error. Until registered, we would have to
write something like ``\lstinline{let x be multMagma; assume x is associative}''
or ``\lstinline{let x be multMagma such that x is associative}''.
\end{remark}

\begin{idiom}\index{Idiom!\lstinline{-like} Attributes}
In the early 1990s, it was idiomatic to introduce an attribute
schematically of the form ``$X$-like'' when defining a new mode $X$. We
see this with the definition of \lstinline{Group-like} attribute for a
\lstinline{multMagma}. This idiom can be found in about 147 occurrences
in the Mizar mathematical library.
\end{idiom}

Observe that a more faithful translation of Mizar's definition would
be:

\begin{quote}
  \textbf{Definition.} A \emph{Group} $\struct{G,\cdot}$ is a set of
  elements with a binary operator $\cdot\colon G\times G\to G$
  satisfying the following conditions:
  \begin{enumerate}
  \item \textbf{Associativity:} for all $a,b,c\in G$, we have $(a\cdot b)\cdot c = a\cdot(b\cdot c)$
  \item \textbf{Identity element:} There exists an element $e$ such that
    $e\cdot a= a\cdot e = a$ for all $a\in G$
  \item \textbf{Inverse Element:} For all $a\in G$, there exists an
    element $b\in G$ with $a\cdot b=b\cdot a = e$.
  \end{enumerate}
\end{quote}
We stress \emph{a Group in Mizar is just a Magma} which just so happens
to satisfy some extra properties.

\begin{definition}
Let $G$ be a group. We define its \define{Identity Element} to be the
unique element $e_{G}\in G$
\end{definition}
\begin{notation}
We denote the identity element sometimes by $1_{G}$ if $G$'s binary
operator is denoted by multiplication, and $0_{G}$ if $G$'s binary
operator is denoted by addition. Since generically the binary operator
is denoted by multiplication, we usually use $1_{G}$ as the identity element.
Mizar has a way to encode this convention, which
is the next definition in~\mml{group1}:\MizDef{GROUP\_1}[04]{4}
\mizindex{1G@$\mathtt{1\_G}$}
\end{notation}


\begin{mizar}
definition
  let G be multMagma such that
A1: G is unital;
  func 1_G -> Element of G means
:: GROUP_1:def 4
  for h being Element of G holds h * it = h & it * h = h;
  existence by A1;
  uniqueness
  :: proof omitted
end;
\end{mizar}

\subsubsection{Functor Definitions}\label{par:introduction:functor-one-g} What's going on here? Well, for any unital magma $G$, we have a
``functor'' (a logical function, i.e., parametrized term) denoted
``\lstinline{1_}'' which produces the unit element for the given magma.
Whereas (in print) mathematicians would write $1_{G}$, in Mizar we would
write ``\lstinline{1_G}''.

Such a term is defined as $\forall h\in G, h\cdot\mathtt{it}=h\land\mathtt{it}\cdot{h}=h$.
Here ``\lstinline{it}''\index{It@\texttt{it}} anaphorically\footnote{We have
Lisp programmers to thank for this lovely term.} refers to the
definiendum (the term being defined, i.e., ``\lstinline{1_G}''). Whenever
functors are defined, their defining properties will involve
``\lstinline{it}'' satisfying some formula.\footnote{The ``\lstinline{IT}''
appearing in definitions of attributes play an analogous role, hence the
idiom noted in Idiom~\ref{rmk:introduction:idiom-it-in-attributes}.}

Mizar demands we prove such a thing exists, and then a proof of
uniqueness is supplied. This coincides with the usual practice in
mathematical logic, when considering a ``function term'' in first-order
logic, which has [existence] exactly one [uniqueness] output for each
possible argument configuration.

\begin{aside}[Functors in Mizar]\index{Functor!In Mizar|(}
A functor in Mizar is the only way to construct a term. Rudnicki and
friends~\cite{rudnicki2001commutative} cite the term originating from
Rasiowa and Sikorski~\cite[see p.148]{rasiowa1963mathematics},
quoting: ``\dots some signs in the formalized language should correspond
to the mappings and functions being examined. These signs are called
\emph{functors}, or---more precisely---\emph{$m$-argument functors}
provided they correspond to $m$-argument mappings from objects to
objects ($m = 1, 2, \dots$)''.

\emph{\textbf{This has absolutely no relation to functors in category theory},
and in my opinion is an extremely poor
choice of terminology due entirely to the ironies of history.}
\index{Functor!In Mizar|)}%
\end{aside}

\subsubsection{Stuff, Structure, Properties versus Mizar Definitions}\index{Stuff, Structure, Properties|textbf}
I grew up learning abstract algebra from the perspective of Baez--Dolan
``Stuff, Structure, and Properties''~\cite{baez2001finite,baez2004qg-lectures,baez2006quantum,baez2010lectures}, thinking in terms of ``group objects''
in some ambient category. Defining new gadgets could then be framed in
terms of consisting of objects equipped with morphisms satisfying some
commutative diagrams. As an added bonus, we get morphisms ``for free''.

Mizar takes a more ``traditional'' approach, defining a
\lstinline{struct} which loosely lists the stuff and minimul structure
underlying a gadget. The properties are then defined as attributes. Even
some structure are introduced using attributes (like the inverse
operator for groups). And then the gadget is bundled together as a
mode. Unlike the Baez--Dolan approach, we do not get morphisms for free.

\subsection{Examples of Groups}
There are not many examples of groups in Mizar, only four nontrivial
groups (that I could find).

\begin{example}
  The real numbers equipped with addition are proven to be a group in
  Theorem~\mml[Th3]{group1}.
\end{example}

\begin{example}\index{$\ZZ$!Group}
The integers, as an additive group, are defined as \lstinline|INT.Group|\mizindex{Integers@\texttt{INT.Group}}
in Definition~\mml[def 1]{gr_cy_1}.
\end{example}

\begin{example}[Cyclic group $\cyclicGroup{n}$]\label{example:introduction:cyclic-group}\index{Group!Cyclic}
The finite cyclic group of order $n$ is defined as \lstinline|INT.Group(n)|\mizindex{Integers@\texttt{INT.Group(n)}}
in Definition~\mml[def 5]{gr_cy_1}. The notion of a cyclic group is
introduced in Definition~\mml[def 7]{gr_cy_1}\MizDef{gr\_cy\_1}{7}\mizindex{Cyclic@\texttt{cyclic}}, and the finite cyclic
group is registered as cyclic.
\end{example}

\begin{example}[Symmetric Group]\index{Group!Symmetric}
Given a non-empty set $X$, we can consider the group of bijections on
$X$, popularly known as the \define{Symmetric Group} and denoted
$\Sym(X)$. It's introduced in Definition~\mml[def 2]{cayley} as
``\lstinline{SymGroup(X)}''\mizindex{SymGroup@\texttt{SymGroup(X)}}.
This Mizar implementation is quite faithful to our mathematical
intuition, using the definition of a \lstinline{Permutation of X} being
a bijection $f\colon X\to X$ as defined in Definition~\mml[def 5]{funct2}. It
does not appear to be developed anywhere further, for example: the
alternating group is not defined anywhere.
\end{example}


\subsection{Inverse operator for group elements}
When we have a group $\struct{G,\cdot}$, we also have the ability
to invert any given element $g\in G$. Mizar denotes this by
``\lstinline{g"}'', which is what the ``Rosette stone'' of notation suggests
(since mathematicians would write $g^{-1}$). This is handled in~\mml[def 5]{group1}.\MizDef{GROUP\_1}[05]{5}
A formal ``\lstinline{inverse_op(G)}''\mizindex{Inverse Op(G)@$\mathtt{inverse\_op(G)}$} is given in~\mml[def 6]{group1}.\MizDef{GROUP\_1}[06]{6}

\begin{notation}[Inverse operator for subsets of $G$]
We generalize this notation in group theory to write, if $A\subset G$ is
some subset, $A^{-1}=\{a^{-1}\in G\mid a\in A\}$. Mizar defines this
notation in \mml[def 1]{group2}.\MizDef{GROUP\_2}[01]{1}
\end{notation}

\subsection{Power operator}
For any $g\in G$, and $n\in\NN_{0}$, we often write $g^{n}$ with the
understanding this stands for $g^{0}=1_{G}$ and $g^{n+1}=g^{n}\cdot g$.
This is defined in~\mml[def 7]{group1}\MizDef{GROUP\_1}[07]{7}. But we generalize this in
mathematics to let $n\in\ZZ$, allowing the case when $n<0$ to be defined
by $g^{n} = (g^{|n|})^{-1}$.

Mizar does this in~\mml[def 8]{group1},\MizDef{GROUP\_1}[08]{8}
following the conventions taken by earlier articles in Mizar that
$x^{y}$ is denoted by ``\lstinline{x |^ y}''.\index{\texttt{x \mid\Caret y}@{\texttt{x "|"\Caret\ y}}!Mizar for $x^{y}$}
For the sake of consistency, Mizar explicitly states in~\mml[def 8]{group1}
when $g\in G$ and $n\esti \NN_{0}$, it follows
that ``\lstinline{g |^ n}'' coincides with what we expect (i.e., $g^{n}$).

\subsection{Order of Group Element (and Group Order)}
If $h\in G$, then we say $h$ is of order 0 if the only $n\in\NN_{0}$
satisfying $h^{n}=1_{G}$ is $n=0$. Mizar defines this as an attribute
in~\mml[def 10]{group1}\MizDef{GROUP\_1}{10} as ``\lstinline{being_of_order_0}''\index{Being of Order 0@$\mathtt{being\_of\_order\_0}$}\mizindex{Being of Order 0@$\mathtt{being\_of\_order\_0}$}.
This segues into defining the notion of the order for an element $g\in G$
as the smallest natural number $|g|=n\in\NN_{0}$ for which $g^{n}=1_{G}$
(provided $g$ is not of order zero, in which case we specially define
its order to be $|g|=0$). Mizar defines this as
``\lstinline{ord g}''\index{Order!\texttt{ord g}}\mizindex{ord g@$\mathtt{ord~g}$}
in~\mml[def 11]{group1}.\MizDef{GROUP\_1}{11}

The order of the group, Mizar overloads the cardinality operator (again,
matching mathematical practice). Specifically,
for $G$ being any finite 1-sorted typed object (like a group), we have
``\lstinline{card G}''\mizindex{Card G@\texttt{card G}}\index{Cardinality@\texttt{card G}} be the cardinality
of the underlying set.

\subsection{Abelian Groups}
Sadly, Mizar departs from mathematical practice, and defines an
attribute in~\mml[def 12]{group1}\MizDef{GROUP\_1}{12} for a
\lstinline{multMagma} to be ``\lstinline{commutative}''.\index{Commutative@\texttt{commutative}}\index{Abelian|see{\texttt{commutative}}}\mizindex{Commutative@\texttt{commutative}}
This demands the expected property
\begin{equation*}
  \forall x,y\in G, x\cdot y=y\cdot x.
\end{equation*}

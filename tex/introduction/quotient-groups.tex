\section{Quotient Groups}

\begin{definition}[{\mml[def 1]{group6}}]
Let $B\subgroup G$ be a subgroup, $M\normalSubgroup G$, and $M\subgroup B$.
Then $(M)_{B}$ is the normal subgroup of $B$ coinciding with $M$.
\end{definition}

\begin{theorem}[{\mml[Th9]{group6}}]
If $N\normalSubgroup G$ and $B\subgroup G$, then $B\cap N$ and $N\cap B$
are both normal subgroups of $B$.
\end{theorem}

\begin{definition}[{\mml[def 2]{group6}}]
We call a group \define{Trivial} if its underlying set is a singleton.
\end{definition}

\begin{definition}[{\mml[def 15]{group2}}]
Let $H\subgroup G$. The \define{Left Cosets} of $H$ is the collection of
subsets $A\subset G$ which look like $A = aH$ for $a\in G$.
\end{definition}

\begin{def-remark}
Without loss of generality, we will simply refer to \define{Cosets} of
$H$ for its left cosets.
\end{def-remark}

\begin{theorem}[{\mml[Th16]{group6}}]
If $N\normalSubgroup G$, then for any cosets $A_{1}$, $A_{2}$ of $H$ in
$G$ we have $A_{1}A_{2}$ be another coset.
\end{theorem}

\begin{definition}[{\mml[def 3]{group6}}]
Let $N\normalSubgroup G$ be a normal subgroup. We can define the
\define{Coset Operator} to be the induced binary operator on cosets
$W_{1}$, $W_{2}$ of $N$ in $G$ by $W_{1}W_{2}$.
\end{definition}

\begin{def-remark}
This makes sense since $W_{1}=aN$ and $W_{2}=bN$ for some $a,b\in G$.
Then $W_{1}W_{2}=(ab)N$.
\end{def-remark}

\begin{definition}[{\mml[def 4]{group6}}]
Let $N\normalSubgroup G$ be a normal subgroup. We can define the
\define{Quotient Group} to be the group $G/N$ whose underlying set is
the cosets of $N$ in $G$ and whose binary operator is
the coset operator.
\end{definition}

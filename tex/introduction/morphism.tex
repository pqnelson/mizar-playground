\section{Group Morphisms}

\begin{definition}[{\mml[def 5, def 6]{group6}}]
Let $G_{1}$, $G_{2}$ be groups.
A \define{Homomorphism} of $G_{1}$, $G_{2}$ is a function
$f\colon G_{1}\to G_{2}$ such that it is multiplicative
\begin{equation}
\forall a,b\in G_{1}, f(ab) = f(a)f(b).
\end{equation}
\end{definition}

\begin{def-remark}
I will use the term ``morphism'' instead of ``homomorphism''.
Frequently, I will specify ``group morphism''.
\end{def-remark}

\begin{theorem}[{\mml[Th31]{group6}}]
If $\varphi\colon G\to H$ is a group morphism, $1_{G}\in G$ and
$1_{H}\in H$ are the identity elements, then $\varphi(1_{G})=1_{H}$.
\end{theorem}

\begin{theorem}[{\mml[Th32]{group6}}]
If $\varphi\colon G\to H$ is a group morphism, and $g\in G$ is
arbitrary, then $\varphi(g^{-1})=\varphi(g)^{-1}$.
\end{theorem}

\begin{theorem}[{\mml[Th33]{group6}}]
If $\varphi\colon G\to H$ is a group morphism, and $a,b\in G$ is
arbitrary, then conjugation is preserved $\varphi(a^{b})=\varphi(a)^{\varphi(b)}$.
\end{theorem}

\begin{theorem}[{\mml[Th34]{group6}}]
If $\varphi\colon G\to H$ is a group morphism, and $a,b\in G$ is
arbitrary, then commutators are preserved $\varphi([a,b])=[\varphi(a), \varphi(b)]$.
\end{theorem}

\begin{theorem}[{\mml[Th37]{group6}}]
If $\varphi\colon G\to H$ is a group morphism, and $g\in G$ and
$n\in\ZZ$ both arbitrary, then $\varphi(g^{n})=\varphi(g)^{n}$.
\end{theorem}

\begin{theorem}[{\mml[Th38]{group6}}]
For any group $G$, the identity function on the underlying set is a
group morphism.
\end{theorem}

\begin{definition}[{\mml[def 7]{group6}}]
For any groups $G_{1}$, $G_{2}$, there is the constant group morphism
$G_{1}\to G_{2}$ sending every $x\in G_{1}$ to
$1_{G_{2}}\in G_{2}$.
\end{definition}

\begin{definition}[{\mml[def 8]{group6}}]
Let $G$ be a group, $N\normalSubgroup G$ a normal subgroup.
The \define{Natural Homomorphism} $G\to G/N$ sends every $g\in G$ to the
coset $gN$.
\end{definition}

\begin{definition}[{\mml[def 9]{group6}}]
Let $\varphi\colon G_{1}\to G_{2}$ be a group morphism. We define the
\define{Kernel} of $\varphi$ to be the subgroup
$\ker(\varphi)=\{g\in G_{1}\mid\varphi(g)=1_{G}\}\subgroup G_{1}$.
\end{definition}

\begin{theorem}
  If $\varphi\colon G_{1}\to G_{2}$ is a group morphism,
  then $\ker(\varphi)\normalSubgroup G_{1}$ is a normal subgroup.
\end{theorem}

\begin{theorem}[{\mml[Th41]{group6}}]
Let $\varphi\colon G_{1}\to G_{2}$ be a group morphism, $a\in G_{1}$ be
an arbitrary group element.
Then $a\in\ker(\varphi)$ if and only if $\varphi(a)=1_{G_{2}}$.
\end{theorem}

\begin{theorem}[{\mml[Th43]{group6}}]
If $N\normalSubgroup G$ and $\varphi\colon G\to G/N$ is the natural
homomorphism, then $\ker(\varphi)=N$.
\end{theorem}

\begin{definition}[{\mml[def10]{group6}}]
  Let $\varphi\colon G_{1}\to G_{2}$ be a group morphism.
  Then the \define{Image} of $\varphi$ is the subgroup
  $\Im(\varphi)\subgroup G_{2}$ such that its underlying set is equal to
  the image $\varphi(U(G_{1}))$ of applying $\varphi$ to the underlying
  set of its domain.
\end{definition}

\begin{theorem}[{\mml[Th45]{group6}}]
If $\varphi\colon G_{1}\to G_{2}$ is a group morphism, then
$x\in\Im(\varphi)$ if and only if there exists $a\in G_{1}$ such that $x=\varphi(a)$.
\end{theorem}

\begin{theorem}[{\mml[Th49]{group6}}]
If $\varphi\colon G_{1}\to G_{2}$ is a group morphism,
then $\varphi$ is also a morphism $G_{1}\to\Im(\varphi)$.
\end{theorem}

\begin{theorem}[{\mml[Th50]{group6}}]
If $G_{1}$ is a finite group and $\varphi\colon G_{1}\to G_{2}$ is a
group morphism, then $\Im(\varphi)$ is a finite subgroup of $G_{2}$.
\end{theorem}

\begin{theorem}[{\mml[Th51]{group6}}]
If $G_{1}$ is an Abelian group and $\varphi\colon G_{1}\to G_{2}$ is a
group morphism, then $\Im(\varphi)$ is an Abelian subgroup of $G_{2}$.
\end{theorem}

\begin{theorem}[{\mml[Th56]{group6}}]
Let $\varphi\colon G_{1}\to G_{2}$ be a group morphism. Then
$\varphi$ is injective if and only if $\ker(\varphi)=\trivialSubgroup_{G_{1}}$.
\end{theorem}

\begin{theorem}[{\mml[Th57]{group6}}]
Let $\varphi\colon G_{1}\to G_{2}$ be a group morphism. Then
$\varphi$ is surjective if and only if $\Im(\varphi)=G_{2}$.
\end{theorem}

\begin{theorem}[{\mml[Th58]{group6}}]
Let $\varphi\colon G_{1}\to G_{2}$ be a group morphism. Then
$\varphi$ is surjective if and only if for each $c\in G_{2}$ there
exists an $a\in G_{1}$ such that $c=\varphi(a)$.
\end{theorem}

\begin{theorem}[{\mml[Th59]{group6}}]
If $N\normalSubgroup G$ is a normal subgroup and $\varphi\colon G\to G/N$
is the natural morphism, then the natural morphism is surjective.
\end{theorem}

\begin{theorem}[{\mml[Th62]{group6}}]
If $\varphi\colon G_{1}\to G_{2}$ is a bijective group morphism,
then the inverse function $\varphi^{-1}\colon G_{2}\to G_{1}$ is a group
morphism. 
\end{theorem}

\begin{definition}[{\mml[def11]{group6}}]
Let $G_{1}$, $G_{2}$ be groups. We say $G_{1}$, $G_{2}$ \define{are Isomorphic} 
if there exists a bijective homomorphism $\varphi\colon G_{1}\to G_{2}$.
We'll denote $G_{1}\iso G_{2}$ to indicate they are isomorphic.
\end{definition}

\begin{theorem}[{\mml[Th66]{group6}}]
If $G_{1}\iso G_{2}$ are isomorphic groups, then $G_{2}\iso G_{1}$.
\end{theorem}

\begin{theorem}[{\mml[Th67]{group6}}]
If $G_{1}\iso G_{2}$ are isomorphic groups, and if $G_{2}\iso G_{3}$ are
isomorphic groups, then $G_{1}\iso G_{3}$.
\end{theorem}

\begin{theorem}[{\mml[Th67]{group6}}]
If $\varphi\colon G_{1}\into G_{2}$ is an injective group morphism,
then $G_{1}\iso\varphi(G_{1})$ are isomorphic groups.
\end{theorem}

\begin{theorem}[{\mml[Th68]{group6}}]
If $G_{1}$ and $G_{2}$ are trivial groups, then $G_{1}\iso G_{2}$.
\end{theorem}

\begin{theorem}[{\mml[Th71]{group6}}]
For any group $G$, $G\iso G/\trivialSubgroup_{G}$.
\end{theorem}

\begin{theorem}[{\mml[Th72]{group6}}]
For any group $G$, $\trivialSubgroup_{G}\iso G/G$.
\end{theorem}

\begin{theorem}[{\mml[Th73]{group6}}]
Isomorphic groups have the same cardinality; i.e., for any $G_{1}\iso G_{2}$,
we have $|G_{1}|=|G_{2}|$.
\end{theorem}

\begin{theorem}[{\mml[Th76]{group6}}]
If $G_{1}$ is trivial and $G_{1}\iso G_{2}$, then $G_{2}$ is trivial.
\end{theorem}

\begin{theorem}[{\mml[Th78]{group6}}]
For any group morphism $\varphi\colon G_{1}\to G_{2}$, we have 
$\Im(\varphi)\iso G_{1}/\ker(\varphi)$.
\end{theorem}

\begin{theorem}[First Isomorphism Theorem, {\mml[Th79]{group6}}]
Let $\varphi\colon G_{1}\to G_{2}$ be a group morphism,
$\theta\colon G\to G/\ker(\varphi)$ be the natural morphism.
There exists a group morphism $\psi\colon G_{1}/\ker(\varphi)\to\Im(\varphi)$
such that $\psi$ is bijective and $\varphi = \psi\circ\theta$.
\end{theorem}

\begin{theorem}[Second Isomorphism Theorem, {\mml[Th80]{group6}}]
Let $M\normalSubgroup G$ and $N\normalSubgroup G$ such that $M\subgroup N$.
For any normal subgroup $J\normalSubgroup G/M$ such that $J=N/M$,
we have $(G/M)/J\iso G/N$ be isomorphic groups.
\end{theorem}

\begin{thm-remark}
Mizar uses the notation $(M)_{N}$ for $M$ as a group with normal
subgroup $N$. This is written ``\lstinline{N./.(N,M)`*`}'' in a Mizar script.
\end{thm-remark}

\begin{theorem}[Third Isomorphism Theorem, {\mml[Th81]{group6}}]
Let $B\subgroup G$ and $N\normalSubgroup G$.
Then $BN/N\iso B/(B\cap N)$.
\end{theorem}


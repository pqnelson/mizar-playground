\chapter{Pure Math}\label{chapter:pure-math}

This is just a grab bag of random results from the literature. A good
introduction to finite group theory would be I.~Martin
Isaacs~\cite{isaacs2008finite} for readers who have taken introductory
abstract algebra, Aschbacher~\cite{aschbacher2000finite} for a summary
of the tools necessary for proving the classification of quasithin groups~\cite{aschbacher2004classification1,aschbacher2004classification2}.
For a summary of finite simple groups with pointers to the literature,
Wilson~\cite{wilson2009finite} does a great job.


\begin{notation}
Most notation is following the literature, but just to
review them here:
\begin{itemize}%\setlength\itemsep{0em}
\item The trivial group is denoted $\trivialGroup$.
\item If $H$ is a subgroup of $G$, we write $H\subgroup G$; and if
  further $H$ is a proper subgroup of $G$, we write $H\properSubgroup G$.
\item For any group $G$, its proper trivial group is denoted
  $\trivialSubgroup_{G}\subgroup G$ or (if there is no confusion) just
  $\trivialSubgroup$.
\item Normal subgroups are denoted by $N\normalSubgroup G$ and proper
  normal subgroups are $N\properNormalSubgroup G$.
\item If $g$, $h\in G$ are group elements, we denote the conjugate of $g$ by
$h$ as $g^{h}=h^{-1}gh$.
\item Like any self-respecting American, I do not consider zero to be a
natural number. Thus $\NN=\{1,2,3,\dots\}$. When I need to refer to the
non-negative integers, I write $\NN_{0}=\{0\}\cup\NN$.
\item The other number systems are with the usual notation ($\ZZ$ for
  integers, $\QQ$ for rationals, $\RR$ for reals, $\CC$ for complex numbers).
\end{itemize}
\end{notation}

\begin{theorem}[{\mml[Th41]{group6}}]\MizThm{GROUP\_6}{41}
  Let $\varphi\colon G_{1}\to G_{2}$ is a group morphism, and
  $e_{G_{2}}\in G_{2}$ be the identity element.
  Then $g\in\ker(h)$ iff $\varphi(g)=e_{G_{2}}$.
\end{theorem}

\begin{theorem}[{\mml[Th56]{group6}}]\MizThm{GROUP\_6}{56}
  Let $\varphi\colon G_{1}\to G_{2}$ is a group morphism.
  Then $\varphi$ is injective if and only if $\ker(\varphi)=\trivialSubgroup$.
\end{theorem}


\begin{theorem}\label{thm:pure-math:iso-subgroups-have-same-order}
  Let $\varphi\in\Aut(G)$ be any automorphism of a group $G$.
  If $H\subgroup G$ is a subgroup, then the order of $H$ and
  $\varphi(H)$ are the same.
\end{theorem}

\begin{proof}[Proof sketch]
This follows from $\varphi$ being a bijection.
\end{proof}

\begin{thm-remark}
Mizar as a stronger result, since
\mml[Th73]{group6}\MizTh{GROUP\_6}{73} proves isomorphic groups have the same order.
\end{thm-remark}

\begin{theorem}[{Gorenstein~\cite[{Th.2.1.5}]{gorenstein1980finite}}]
  If $H$ is a minimal normal subgroup of $G$ (i.e., $H\normalSubgroup G$
  and there is no nontrivial proper subgroup of $H$ which is normal in
  $G$),
  then \emph{either} $H$ is an elementary Abelian $p$-group (for some
  prime $p\in\NN$) \emph{or} $H$ is the direct product of isomorphic
  non-Abelian simple groups.
\end{theorem}

\begin{proposition}
The class formula\index{Class Formula} for groups is proven in~\mml{weddwitt}.
\end{proposition}

\begin{definition}
Let $\pi$ be a set of prime numbers. We call a group $G$ a
\define{$\pi$-Group} if the set of the prime divisors of the order of
$G$, denoted $\pi(G)$, is a subset of $\pi\supset\pi(G)$.
\end{definition}

Observe that if $\pi=\{p\}$ is a single prime number, then we recover
the notion of a $p$-group.

\begin{definition}[Dummit and Foote \S3.2]
Let $A\subset G$ be some subset of the group $G$, let $H\subgroup G$ 
be some subgroup of $G$.
We say \define{$A$ Normalizes $H$} if $A\subset N_{G}(H)$ it is a subset
of the normalizer of $H$. Similarly, we say \define{$A$ Centralizes $H$}
if $A\subset C_{G}(H)$ is a subset of the centralizer.
\end{definition}

\begin{definition}[{\cite[def~23.1]{cfsg2}}]
Let $B$ and $Y$ be subgroups of the group $X$, let $\pi$ be a set of primes.
Then $\signalizer_{Y}(B;\pi)$ is the set of all $B$-invariant [i.e., $B$
normalizes it] $\pi$-subgroups of $Y$.
\end{definition}

\section{Characteristic Subgroups}

\begin{definition}[{Dummit and Foote~\cite[\S4.4]{dummit-foote}}]\index{Subgroup!Characteristic}\index{Characteristic!Subgroup}
A subgroup $H$ of $G$ is called \define{Characteristic} in $G$, usually
denoted $H~\mathrm{char}~G$, if every Automorphism of $G$ maps $H$ to
itself; i.e., $\sigma(H)=H$ for all $\sigma\in\aut(G)$.
\end{definition}

\begin{theorem}[{Gorenstein~\cite[Th.2.1.3]{gorenstein1980finite}}]
Let $G$ be a group. If a normal subgroup $H\normalSubgroup G$ whose
order and index are coprime
\begin{equation*}
\gcd(|H|, [G:H])=1,
\end{equation*}
then $H$ is a characteristic subgroup of $G$.
\end{theorem}

\begin{theorem}[{Gorenstein~\cite[Th.2.1.3(iv)]{gorenstein1980finite}}]
If  $H\subgroup K$ are both subgroups of $G$ such that $H$ is
a characteristic subgroup of $G$, and if $K/H$ is characteristic in
$G/H$, \emph{then} $K$ is characteristic in $G$.
\end{theorem}

\begin{lemma}
Let $G$ be a finite group, $H$ be a Sylow $p$-subgroup of $G$. Then for
any automorphism $\varphi\in\Aut(G)$, $\varphi(H)$ is a Sylow
$p$-subgroup of $G$.
\end{lemma}

\begin{theorem}
The $p$-core for any group $G$ is a characteristic subgroup.
\end{theorem}

\begin{definition}\label{defn:pure-math:X-residual}
Let $G$ be a finite group, let $\mathcal{X}$ be a class of finite
subgroups of $G$ which is closed under isomorphisms (in particular,
automorphisms of $G$), quotients, subgroups, and finite direct products.
Then the \define{$\mathcal{X}$-Residual} of $G$ is the subgroup
\begin{equation}
  O^{\mathcal{X}}(G) := \bigcap\{N\normalSubgroup G\mid G/N\in\mathcal{X}\}.
\end{equation}
\end{definition}

\begin{def-remark}
This definition seems to be folklore. Indeed, I only discovered it by
accident from the internet.\footnote{I am indebted to Jack Schmidt's
post about it here: \url{https://math.stackexchange.com/a/216961/31693}}
\end{def-remark}

\begin{theorem}
The $\mathcal{X}$-Residual is the unique normal subgroup of $G$ such
that $G/N\in\mathcal{X}$ if and only if $O^{\mathcal{X}}(G)\subgroup N$.
\end{theorem}

\begin{theorem}
For any group $G$ and family of subgroups $\mathcal{X}\neq\emptyset$,
the $\mathcal{X}$-residual of $G$ is a characteristic subgroup of $G$.
\end{theorem}

\section{Permutation Groups}

\begin{theorem}[Orre]
The derived subgroup of $S_{n}$ is the alternating group $A_{n}$.
\end{theorem}

\section{Subgroup Series}

\begin{definition}
Let $G$ be a group. A [finite] \define{Subgroup Series} of $G$ is a
chain, either ascending $\trivialSubgroup=A_{0}\subgroup A_{1}\subgroup\dots\subgroup A_{n}=G$
or descending $G=B_{0}\supgroup B_{1}\supgroup\dots\supgroup B_{n}=\trivialSubgroup$.
\end{definition}

\begin{def-remark}
Mizar seems to take the convention of using descending series of subgroups
$G=B_{1}\supgroup B_{2}\supgroup\dots\supgroup B_{n}=\trivialSubgroup$.
\end{def-remark}

\begin{example}[Central Series]
The definition of a nilpotent group uses a central series\index{Central Series}
$G=Z_{1}\supgroup Z_{2}\supgroup\dots\supgroup Z_{n}=\trivialSubgroup$ where
each $Z_{k}\normalSubgroup G$ and the quotient satisfies
$Z_{k}/Z_{k+1}\subgroup Z(G/Z_{k+1})$.
\end{example}

\begin{mizar}
definition
  let IT be Group;
  attr IT is nilpotent means :Def2: :: GRNILP_1:def 2
  ex F being FinSequence of the_normal_subgroups_of IT st
  (len F > 0 & F.1 = (Omega).IT & F.(len F) = (1).IT &
   (for i being Element of NAT st i in dom F & i+1 in dom F
    for G1, G2 being strict normal Subgroup of IT
    st G1 = F.i & G2 = F.(i + 1)
    holds (G2 is Subgroup of G1 &
           G1./.((G1,G2)`*`) is Subgroup of
             center (IT./.G2))));
end; 
\end{mizar}

\begin{example}[Subnormal Series]
A solvable group is defined by having a subnormal series\index{Subnormal!Series}
%$\trivialSubgroup = N_{n}\normalSubgroup\dots\normalSubgroup N_{2}\normalSubgroup N_{1}=G$
$G=N_{1}\normalSupgroup N_{2}\normalSupgroup\dots\normalSupgroup N_{n}=\trivialSubgroup$
--- that is, $N_{j+1}\normalSubgroup N_{j}$ for $j=0,\dots,n-1$ ---
whose factor groups $N_{k}/N_{k+1}$ are Abelian.
\end{example}

\begin{mizar}
definition
  let IT be Group;
  attr IT is solvable means :Def1: :: GRSOLV_1:def 1
  ex F being FinSequence of Subgroups IT st
  (len F > 0 & F.1 = (Omega).IT & F.(len F) = (1).IT &
   (for i being Element of NAT st i in dom F & i+1 in dom F
    for G1, G2 being strict Subgroup of IT
    st G1 = F.i & G2 = F.(i+1)
    holds (G2 is strict normal Subgroup of G1 &
           (for N being normal Subgroup of G1 st N = G2
            holds G1 ./. N is commutative))));
end; 
\end{mizar}

\begin{example}[Composition Series]
A famous example of a subgroup series is the composition
series\index{Composition Series} for a
group. Mizar defines it as a subgroup series
$\trivialSubgroup=A_{n}\normalSubgroup\dots\normalSubgroup A_{2}\normalSubgroup A_{1}=G$.
Although this seems to be using a strange convention, Mizar is following
Bourbaki's \emph{Algbra}. The more familiar notion of a composition
series (as a ``maximal'' subnormal series) requires the
\lstinline{jordan_holder} attribute (again, this follows Bourbaki's
\emph{Algebra} Definition~10 in I.4.7).
\end{example}

\begin{mizar}
:: ALG I.4.7 Definition 9
definition
  let O be set;
  let G be GroupWithOperators of O;
  let IT be FinSequence of the_stable_subgroups_of G;
  attr IT is composition_series means
  :: GROUP_9:def 28
  IT.1=(Omega).G & IT.(len IT)= (1).G &
  for i being Nat st i in dom IT & i+1 in dom IT
  for H1,H2 being StableSubgroup of G
  st H1=IT.i & H2=IT.(i+1)
  holds H2 is normal StableSubgroup of H1;
end;
\end{mizar}

\begin{example}[Lower Central Series]
The lower central series\index{Central Series!Lower} may be defined as
$G=A_{1}\normalSupgroup A_{2}\normalSupgroup\dots\normalSupgroup A_{n}=\trivialSubgroup$,
where $A_{k+1}=[A_{k},G]=[G,A_{k}]$.
Mizar implicitly defines this in \mml[Th27]{grnilp1}.
\end{example}

\begin{mizar}
theorem :: GRNILP_1:27
  for G being Group
  st ex F being FinSequence of the_normal_subgroups_of G
     st len F > 0 & F.1 = (Omega).G & F.(len F) = (1).G &
     for i st i in dom F & i+1 in dom F
     for G1 being strict normal Subgroup of G st G1 = F.i
     holds [.G1, (Omega).G.] = F.(i+1)
  holds G is nilpotent
proof end;
\end{mizar}


\section{Solvable Groups}

This is the formal proof sketches of \S9 of Aschbacher~\cite{aschbacher2000finite}.

\begin{definition}[{\cite[\S8]{aschbacher2000finite}}]
We define the $n^{\text{th}}$-derived subgroup (for any $n\in\NN$) as:
\begin{enumerate}
\item $G^{(0)} = G$
\item $G^{(1)} = [G, G]$ 
\item $G^{(n+1)} = [G^{(n)}, G^{(n)}]$.
\end{enumerate}
\end{definition}

\begin{theorem}[{\cite[(9.1)]{aschbacher2000finite}}]
A group $G$ is solvable if and only if there exists some positive
$n\in\NN$ such that $G^{(n)}=\trivialGroup$.
\end{theorem}

\begin{theorem}[{\cite[(9.2)]{aschbacher2000finite}}]
A finite group is solvable if and only if all its composition factors
are of prime order.
\end{theorem}

\begin{theorem}
  \begin{enumerate}
  \item If $G$ is a solvable group and $\varphi\colon G\to H$ is a group
morphism, then $\varphi(G)$ is a solvable group.
\item If $G$ is solvable and $H\subgroup G$, then $H$ is solvable.
\item Let $H\normalSubgroup G$. If $H$ is solvable and $G/H$ is
  solvable, then $G$ is solvable.
  \end{enumerate}
\end{theorem}

\begin{thm-remark}
This first claim is proven in Theorem~\mml[Th16]{grsolv1},
the second claim is proven in Theorem~\mml[Th5]{grsolv1}.
\end{thm-remark}

\begin{definition}
Let $G$ be a group. We call a subgroup $H$ a \define{Minimal Normal}
subgroup if $H\normalSubgroup G$ and for any $N\normalSubgroup G$
such that $N\subgroup H$ we necessarily have $N=\trivialGroup$.
\end{definition}

\begin{mizar}
definition
  let G be Group;
  let IT be Subgroup of G;
  attr IT is minimal_normal means
  IT is normal & for N being strict normal Subgroup of G
  st N is Subgroup of IT holds N=(1).G;
end;

registration
  let G be Group;
  cluster minimal_normal for Subgroup of G;
  existence; :: take (1).G;
end;

registration
  let G be Group;
  cluster minimal_normal -> normal for Subgroup of G;
  correctness;
end;
\end{mizar}

\begin{definition}
Let $p\in\NN$ be prime.
A \define{Elementary Abelian} $p$-group is a group whose non-identity
elements are order $p$.
\end{definition}

\begin{mizar}
definition
  let p be Nat;
  let IT be p-group Group;
  attr IT is elementary_abelian means
  for g being Element of IT st g <> 1_IT holds ord g = p;
end;

registration
  let p be Prime;
  cluster elementary_abelian for p-group Group;
  existence; :: take INT.Group(p)
end;

registration
  let p be Prime;
  cluster elementary_abelian for solvable p-group Group;
  existence;
end;
\end{mizar}

\begin{theorem}[{\cite[(9.4)]{aschbacher2000finite}}]
Solvable minimal normal subgroups of finite groups are elementary
abelian $p$-groups.
\end{theorem}

\begin{mizar}
theorem
  for G being finite Group
  for H being solvable minimal_normal Subgroup of G
  ex p being Prime st H is elementary_abelian p-group Group;
\end{mizar}

\begin{definition}
Let $G$ be a group. The \define{Lower Central Series} is a family of
groups defined recursively as:
\begin{enumerate}
\item $L_{1}(G) = G$
\item $L_{n+1}(G) = [L_{n}(G), G]$.
\end{enumerate}
\end{definition}

\begin{mizar}
definition
  let G be Group;
  func lower_central_series G -> Function means
  dom it = pos Nat & it.1 = G
  & for n being pos Nat holds it.(n+1) = [. (it.n), (Omega).G .];
  existence;
  uniqueness;
end;

definition
  let G be Group;
  synonym L(G) for lower_centra_series G;
end;
\end{mizar}


\begin{theorem} Let $G$ be a group.
\begin{enumerate}
\item $L_{n}(G)$ is a characteristic subgroup of $G$ for each $n\in\NN$
\item $L_{n+1}(G)\subgroup L_{n}(G)$ for each $n\in\NN$
\item $L_{n}(G)/L_{n+1}(G)\subgroup Z(G/L_{n+1}(G))$ for each $n\in\NN$.
\end{enumerate}
\end{theorem}

\begin{mizar}
reserve G for Group;

theorem
  for n being pos Nat
  holds L(G).n is strict characteristic Subgroup of G;

definition
  let G be Group;
  let n be positive Nat;
  redefine (lower_central_series G).n -> strict
  characteristic Subgroup of G;
  correctness;
end;

theorem
  for n being positive Nat
  holds L(G).(n+1) is Subgroup of L(G).n;

theorem
  for n being positive Nat
  for Gn being strict normal Subgroup of L(G).(n+1)
  st Gn = (lower_central_series G).n
  holds ((lower_central_series G).(n+1)) ./. Gn
         = center (G ./. (lower_central_series G).n);
\end{mizar}

\begin{theorem}
Let $G_{1}$, $G_{2}$ be groups, let $\varphi\colon G_{1}\to G_{2}$ be a
group morphism, and let $H_{2}\subgroup G_{2}$ be a subgroup.
Then the pre-image $\varphi^{-1}(H_{2})$ forms a subgroup of $G_{1}$.
\end{theorem}

\begin{mizar}
theorem
  for G1,G2 being Group
  for H2 being Subgroup of G2
  for phi being Homomorphism of G1,G2
  ex H1 being strict Subgroup of G1
  st carr H1 = phi " carr H2;

definition
  let G1,G2 be Group;
  let H2 be Subgroup of G2;
  let phi be Homomorphism of G1,G2;
  func Preimage(phi,H) -> strict Subgroup of G1 means
  carr it = phi " carr H2;
  existence;
  uniqueness;
end;

registration
  let G1,G2 be Group;
  let N be normal Subgroup of G2;
  let phi be Homomorphism of G1,G2;
  cluster Preimage(phi, N) -> normal;
  correctness;
end;
\end{mizar}

\begin{definition}
Let $G$ be a group. We define the \define{Upper Central Series} for $G$
to be a function on $\NN$ defined recursively by:
\begin{enumerate}
\item $Z_{0}(G)=\trivialSubgroup$
\item $Z_{n}$ is the pre-image in $G$ of $Z(G/Z_{n-1}(G))$.
\end{enumerate}
Observe $Z_{1}=Z(G)$ and that each $Z_{n}(G)$ is a characteristic
Subgroup of $G$.
\end{definition}

\begin{def-remark}
Compare this to Definition~\mml[def2]{grnilp1}
\end{def-remark}

\begin{mizar}
definition
  let G be Group;
  func upper_central_series G -> Function of Nat, the_normal_subgroups_of G
  means
  it.0 = (1).G
  & for n being pos Nat
    holds it.(n+1) = Preimage(nat_hom it.n, G./.(it/.n));
  existence;
  uniqueness;
end;
\end{mizar}

\begin{proposition}
For each $n\in\NN$, we see $Z_{n}(G)$ is a characteristic Subgroup of $G$.
\end{proposition}

\begin{mizar}
definition
  let G be Group;
  let n be Nat;
  redefine (upper_central_series G).n -> strict characteristic
  Subgroup of G;
  
  correctness;
end;
\end{mizar}

\begin{definition}
Let $G$ be a nilpotent group. We define the \define{Nilpotence Class}
of $G$ to be the smallest $n$ for which there exists a defining subgroup
series for it.
\end{definition}

\begin{def-remark}
I am trying to be consistent with how Mizar defines a nilpotent group,
which is in terms of the upper central series (but without using that
name). This means we have to define it in terms of the upper central
series.
\end{def-remark}

\begin{mizar}
definition
  let G be nilpotent Group;
  func nilpotence_class G -> pos Nat means
  for F being FinSequence of the_normal_subgroups_of G st
  (len F > 0 & F.1 = (Omega).G & F.(len F) = (1).G &
   (for i being Element of NAT st i in dom F & i+1 in dom F
    for G1, G2 being strict normal Subgroup of G
    st G1 = F.i & G2 = F.(i + 1)
    holds (G2 is Subgroup of G1 &
           G1./.((G1,G2)`*`) is Subgroup of
             center (G./.G2))))
  for n being Nat st n <= len F
  holds n <= it & it <= len F;
end; 
\end{mizar}


\begin{theorem}[{\cite[(9.7)]{aschbacher2000finite}}]
Let $G$ be a nontrivial group.
Then $G$ is nilpotent of class $m$ if and only if $G/Z(G)$ is nilpotent
of class $m-1$.
\end{theorem}

\begin{mizar}
theorem ThNilpotentIffQuotientByCenter:
  G is nilpotent iff G./.(center G) is nilpotent;

registration
  let G be nilpotent Group;
  cluster G./.(center G) -> nilpotent;
  correctness;
end;

theorem
  for G being nilpotent Group
  holds nilpotence_class (G./.(center G)) = nilpotence_class G - 1;
\end{mizar}

\begin{theorem}[{\cite[(5.16)]{aschbacher2000finite}}]
If $G$ is a nontrivial $p$-group, then $Z(G)\neq\trivialSubgroup$.
\end{theorem}

\begin{mizar}
theorem
  for p being Prime
  for G being non trivial p-group Group
  holds center G is non trivial;
\end{mizar}

\begin{theorem}[{\cite[(9.8)]{aschbacher2000finite}}]
Let $p\in\NN$ be prime. Then $p$-groups are nilpotent.
\end{theorem}

\begin{mizar}
theorem
  for p being Prime
  for G being p-group Group
  holds G is nilpotent
proof
  let p be Prime;
  let G be p-group Group;
  per cases;
  suppose G is trivial;
    hence thesis;
  end;
  suppose A1: G is non trivial;
    defpred P[Group] means $1 is p-group;
    defpred Q[Group] means $1 is nilpotent;
    (not Q[G] &
    for H being Group st P[H] & card H < card G holds Q[H])
    implies contradiction
    proof
      assume A2: not Q[G];
      assume A3: for H being Group
                 st P[H] & card H < card G
                 holds Q[H];
      P[center G];
      P[G./.(center G)] & card (G./.(center G)) < card G;
      then Q[G./.(center G)] by A3;
      then Q[G] by ThNilpotentIffQuotientByCenter;
      hence contradiction by A2;
    end;
    hence thesis; :: by minimal counter-example
  end;
end;
  
registration
  let p be Prime;
  cluster p-group -> nilpotent for Group;
  correctness;
end;
\end{mizar}

\begin{thm-remark}
We should take a moment to think about the structure of the proof here,
the ``argument by minimal counter-example'' appears to be a regular
induction proof. Recall well-founded induction looks like: if $x$ is an
element of $X$ and $P[y]$ is true for all $y$ such that $y~R~x$, then
$P[x]$ must also be true.

We are trying to prove for any group $G$ that $P[G]\implies Q[G]$.
We start by proving for any group $H$ such that $H~R~G$ (i.e., $|H|<|G|$)
that $P[H]\implies Q[H]$ holds must imply $P[G]\implies Q[G]$ holds.
This is logically equivalent to $P[G]\land(P[H]\implies Q[H])\implies Q[G]$ 
by Currying. We have assumed $P[G]$. We can Curry again, writing
$Q[G]=(\neg Q[G])\implies\bot$ to make the proof obligation become
$P[G]\land(P[H]\implies Q[H])\land\neg Q[G]\implies\bot$ by Currying,
which is what we have proven.

For this to work in Mizar, I believe we need to define a lattice of
finite groups. We then need to use the scheme ``\lstinline{WFInduction}''
from~\mml{wellfnd1}.
\end{thm-remark}

\begin{mizar}
scheme :: WELLFND1:sch 3
WFInduction{ F1() -> non empty RelStr , P1[ set ] } :
  for x being Element of F1() holds P1[x]
provided
  A1: for x being Element of F1()
      st (for y being Element of F1()
          st y <> x & [y,x] in the InternalRel of F1()
          holds P1[y])
      holds P1[x] and
  A2: F1() is well_founded;
\end{mizar}

\begin{theorem}
If $G$ is a nilpotent group of class $m$, then subgroups and homomorphic
images of $G$ are nilpotent of class at most $m$.
\end{theorem}

\begin{theorem}
If $G$ is nilpotent and $H\subgroup G$ is a subgroup,
then $H\properSubgroup N_{G}(H)$ is a proper subgroup of its normalizer
in $G$.
\end{theorem}

\begin{mizar}
theorem
  for G being nilpotent Group
  for H being Subgroup of G
  holds H is proper Subgroup of Normalizer H;
\end{mizar}

\begin{theorem}
A finite group is nilpotent if and only if it is the [internal] direct
product of its Sylow subgroups.
\end{theorem}

\begin{mizar}
definition
  let G be Group;
  func Sylow_subgroups G -> Subset of Subgroups G means
  for P being strict Subgroup of G
  holds P in it
        iff ex p being Prime
            st P is_Sylow_p-subgroup_of_prime p;
  existence;
  uniqueness;
end;

theorem ThProdOfNilpotents:
  for G being Group
  for I being set
  for F being normal Subgroup-Family of G,I
  st (for i being Element of I holds F.i is nilpotent)
   & G is_internal_product_of F
  holds G is nilpotent;

theorem ThUniqueSylowSubgroups:
  for G being finite Group
  st (for p being Prime st p divides card G
      ex P being strict Subgroup of G
      st the_sylow_p-subgroups_of_prime (p,G) = {P})
  holds G is_internal_product_of Sylow_subgroups G;
  
theorem
  for G being finite Group
  holds G is_internal_product_of Sylow_subgroups G
        iff G is nilpotent
proof
  thus C1: G is_internal_product_of Sylow_subgroups G
           implies G is nilpotent
  proof
    assume A1: G is_internal_product_of Sylow_subgroups G;
    reconsider I=Sylow_subgroups of G as set,
    F=id (Sylow_subgroups of G) as Subgroup-Family of I,G;
    A2: G is_internal_product_of F by A1;
    for i being Element of I holds F.i is nilpotent;
    hence thesis by A2, ThProdOfNilpotents;
  end;
  thus G is nilpotent
       implies G is_internal_product_of Sylow_subgroups G
  proof
    assume A1: G is nilpotent;
    for p being Prime st p divides card G
    for P being Subgroup of G
    st P is_Sylow_p-subgroup_of_prime p
    holds the_sylow_p-subgroups_of_prime (p,G) = {P};
    then G is_internal_product_of Sylow_subgroups G
    by ThUniqueSylowSubgroups;
    hence thesis by C1;
  end;
end;
\end{mizar}

\subsection{More about ``Proof by Minimal Counter-example''}
We want to be able to perform proofs by minimal counter-example
easily. Towards that end, we need to construct the infrastructure to
facilitate such proofs. We need something like

\begin{mizar}
definition
  func finite_groups -> RelStr means
  (for G being object
   holds G is strict finite Group
         iff G in the carrier of it)
  & (for x,y being strict finite Group
     holds [x,y] in the InternalRel of it
           iff card x < card y);
  existence;
  uniqueness;
end;

registration
  cluster finite_groups -> non empty;
  correctness; :: the trivial finite Group in finite_groups
end;
\end{mizar}

We need to prove that ``\lstinline{finite_groups}'' is well-founded,
then we can construct a scheme which basically amounts to a wrapper
around the well-founded induction scheme.

However, it is unclear to me whether we should restrict ourselves to
\emph{strict} finite groups, or whether we should permit \emph{all}
finite groups. I think all finite groups would no longer be
well-founded, which could cause problems: our induction scheme would
only work for proving properties about \emph{strict} finite groups.
\section{Local Analysis of Finite Groups}

``Local analysis of a finite group $G$ means the study of the structure
of, and the interaction betewen, the centralizers and normalizers of
nonidentity $p$-subgroups of $G$.'' Bender and Glauberman note in the
preface to their book, \emph{Local Analysis for the Odd Order Theorem}.
This is carefully introduced in I.~Martin Isaacs' \emph{Finite Group Theory},
chapter 2C (pp.58 \emph{et seq}.).
\section{Finite Fields}

It seems that Galois Fields\index{Field!Galois} are not defined in
general, not anywhere I could find. Well, true to form, Mizar defines
the finite rings $\ZZ/n\ZZ$ as \lstinline{INT.Ring(n)}. And as far as
Mizar cares, a field is just a ring satisfying extra properties. So the
finite ring \lstinline{INT.Ring(p)} is provably a field, for example.

Unfolding the definitions, we have Definition~\mml[def10,11,12]{vectsp1}
first define \lstinline{Skew-Field} as a ring satisfying extra properties
(i.e., \lstinline{non degenerated} \lstinline{almost_left_invertible} \lstinline{Ring}).
Then a \lstinline{Field} is \lstinline{commutative Skew-Field}.

\begin{example}[{$\FF_{3}$}]
The field with three elements $\FF_{3}$ is defined as \lstinline!Z_3!
in \mml[def 20]{mod2}\MizDef{MOD\_2}{20}.
\end{example}

\begin{example}[{$\ZZ/n\ZZ$}]
The integers modulo $n$ form a ring in Definition~\mml[def12]{int3} as
\lstinline{INT.Ring(n)}. I think Theorem~\mml[Th12]{int3} suffices to
prove \lstinline{INT.Ring(p)} is a field. Also Registration~\mml[reg14]{int3}
associated \lstinline{INT.Ring(p)} with being\dots well, many things,
but they combine together to make it a finite field.
\end{example}

\begin{exercise}
Prove \lstinline{INT.Ring(p |^ n)} is a field where ``\lstinline{p |^ n}''
is Mizar for $p^{n}$ (where $p,n\in\NN$ and $p$ is prime).
\end{exercise}

\section{Finite Groups of Lie Type}

\begin{ddanger}
The notation for finite groups of Lie type is bewildering to people who
know Lie groups. This is because Artin~\cite{artin1955order} introduced
idiosyncratic notation for finite groups of Lie type, which diverges
quite badly from the notational choices made in the development of Lie
groups. 
\end{ddanger}

\section{Finite Simple Groups}

\begin{proposition}\label{prop:pure-math:cfsg}
There are four types of finite simple groups:
\begin{enumerate}
\item the finite cyclic groups of prime order;
\item the alternating groups $A_{n}$ for $n\geq 5$;
\item the finite groups of Lie type; and
\item the Sporadic groups (i.e., ``the 26 whacky groups'')
\end{enumerate}
Moreover, any finite simple group must be one of these four types, with
the only repetitions being:
\begin{subequations}
  \begin{align}
    \PSL_{2}(4)&\iso A_{5}\\
    \PSL_{2}(5)&\iso A_{5}\\
    \PSL_{2}(7)&\iso\PSL_{3}(2)\\
    \PSL_{2}(9)&\iso A_{6}
  \end{align}
\end{subequations}\vskip-2\belowdisplayskip\vskip-2\abovedisplayskip
  \begin{align}
    \PSL_{4}(2)&\iso A_{8}\\
    \PSU_{4}(2)&\iso\PSp_{4}(3).
  \end{align}
\end{proposition}

\begin{definition}
We say a group $G$ is a \define{$\mathcal{K}$-Group} if it is one of the
simple groups as listed in Proposition~\ref{prop:pure-math:cfsg}.
\end{definition}

\subsection{Sporadic Groups}
These are the simple groups which are not finite groups of Lie type,
cyclic, or alternating. Their construction is non-uniform, since they
are symmetries of random exotic mathematical objects.

\begin{definition}
  We call a finite group $G$ \define{Sporadic} if
  \begin{enumerate}
  \item $G$ is a simple group,
  \item $G$ is not cyclic,
  \item $G$ is not of Lie type,
  \item $G$ is not an alternating group.
  \end{enumerate}
\end{definition}

\begin{def-remark}[References]
For actually \emph{constructing} sporadic groups,
Aschbacher~\cite{aschbacher1994sporadic} provides a concise series of
constructions, Wilson~\cite{wilson2009finite} provides elegant
constructions. Griess~\cite{griess1998twelve} reviews 12 sporadic
groups.
\end{def-remark}

\begin{def-remark}[Organization of Sporadics]
There are three ``generations'' used to organize sporadic groups plus
the six ``pariahs'':
\begin{enumerate}
\item First Generation (the Mathieu groups): $\Mathieu{11}$,
  $\Mathieu{12}$, $\Mathieu{22}$, $\Mathieu{23}$, $\Mathieu{24}$;
\item Second Generation (the Leech Lattice groups): the Conway groups
  $\Conway{1}$, $\Conway{2}$, $\Conway{3}$; the McLaughlin group
  $\McLaughlin$; the Higman--Sims group $\HigmanSims$; and the Janko
  group $\Janko{2}$;
\item Third Generation (other Subgroups of the Monster $\Monster$): the most
  relevant one for quasithin classification being the Held group
  $\Held$;
\item Pariahs: three Janko groups $\Janko{1}$, $\Janko{3}$, $\Janko{4}$;
  the O'Nan group $\ONan$; the Rudvalis group $\Rudvalis$; and Lyons
  group $\Lyons$.
\end{enumerate}
They are very exciting and beautiful, but I am not going to investigate
\emph{all} of them further.
\end{def-remark}

\begin{def-remark}[Quasithin Sporadics]
According to Aschbacher and Smith~\cite{aschbacher2004classification1,aschbacher2004classification2}, the
sporadic groups which are quasithin include:
$\Mathieu{11}$, $\Mathieu{12}$, $\Mathieu{22}$,
$\Mathieu{23}$, $\Mathieu{24}$, $\Janko{2}$, $\Janko{3}$, $\Janko{4}$,
$\HigmanSims$, $\Held$, $\Rudvalis$. Clustered by generation:
\begin{enumerate}
\item all five Mathieu groups ($\Mathieu{11}$, $\Mathieu{12}$, $\Mathieu{22}$, $\Mathieu{23}$,
  $\Mathieu{24}$);
\item a couple Leech Lattice groups ($\HigmanSims$, $\Janko{2}$);
\item from the Monster-related sporadics, only the Held group $\Held$; and
\item $\Janko{3}$, $\Janko{4}$, $\Rudvalis$.
\end{enumerate}
Also note: unlike the Mathieu groups, the Conway groups, or the Fischer
groups, the Janko groups do not form a series or ``family'' --- they are
just a few ``random'' groups constructed by the same mathematician.
\end{def-remark}

\begin{proposition}
Finite non-Abelian simple quasithin $\mathcal{K}$-groups of even
characteristic consist of:
\begin{enumerate}
\item (Generic case) Groups of Lie type of characteristic 2 and Lie rank at most 2, but
  $U_{5}(q)$ only for $q=4$
\item (Certain groups of rank 3 or 4) $L_{4}(2)$, $L_{5}(2)$, $\Sp_{6}(2)$
\item (Alternating groups) $A_{5}$, $A_{6}$, $A_{8}$, $A_{9}$
\item (Lie type of odd characteristic) $L_{2}(p)$ when $p$ is a Mersenne
  or Fermat prime, $L_{3}^{\epsilon}(3)$, $L_{4}^{\epsilon}(3)$, $G_{2}(3)$
\item (Sporadic) $\Mathieu{11}$, $\Mathieu{12}$, $\Mathieu{22}$,
$\Mathieu{23}$, $\Mathieu{24}$, $\Janko{2}$, $\Janko{3}$, $\Janko{4}$,
$\HigmanSims$, $\Held$, $\Rudvalis$
\end{enumerate}
\end{proposition}


\chapter{Pure Math}\label{chapter:pure-math}

This is just a grab bag of random results from the literature. A good
introduction to finite group theory would be I.~Martin
Isaacs~\cite{isaacs2008finite} for readers who have taken introductory
abstract algebra, Aschbacher~\cite{aschbacher2000finite} for a summary
of the tools necessary for proving the classification of quasithin groups~\cite{aschbacher2004classification1,aschbacher2004classification2}.
For a summary of finite simple groups with pointers to the literature,
Wilson~\cite{wilson2009finite} does a great job.


\begin{notation}
Most notation is following the literature, but just to
review them here:
\begin{itemize}%\setlength\itemsep{0em}
\item The trivial group is denoted $\trivialGroup$.
\item If $H$ is a subgroup of $G$, we write $H\subgroup G$; and if
  further $H$ is a proper subgroup of $G$, we write $H\properSubgroup G$.
\item For any group $G$, its proper trivial group is denoted
  $\trivialSubgroup_{G}\subgroup G$ or (if there is no confusion) just
  $\trivialSubgroup$.
\item Normal subgroups are denoted by $N\normalSubgroup G$ and proper
  normal subgroups are $N\properNormalSubgroup G$.
\item If $g$, $h\in G$ are group elements, we denote the conjugate of $g$ by
$h$ as $g^{h}=h^{-1}gh$.
\item Like any self-respecting American, I do not consider zero to be a
natural number. Thus $\NN=\{1,2,3,\dots\}$. When I need to refer to the
non-negative integers, I write $\NN_{0}=\{0\}\cup\NN$.
\item The other number systems are with the usual notation ($\ZZ$ for
  integers, $\QQ$ for rationals, $\RR$ for reals, $\CC$ for complex numbers).
\end{itemize}
\end{notation}

\begin{definition}[{\texttt{GROUP\_5:10}}]\mizindex{Center@\texttt{center}|see{$\mathtt{GROUP\_5:def 10}$}}\index{$Z(G)$}\index{Center}\index{Group!Center}
Let $G$ be a group. Its \define{Center} is the Subgroup $Z(G)\subgroup G$
consisting of all elements which commute with all of $G$:
\begin{equation*}
Z(G) = \{z\in Z(G) | \forall g\in G, zg=gz\}
\end{equation*}
\end{definition}

\begin{theorem}[{\texttt{GROUP\_6:Th41}}]\MizThm{GROUP\_6}{41}
  Let $\varphi\colon G_{1}\to G_{2}$ is a group morphism, and
  $e_{G_{2}}\in G_{2}$ be the identity element.
  Then $g\in\ker(h)$ iff $\varphi(g)=e_{G_{2}}$.
\end{theorem}

\begin{theorem}[{\texttt{GROUP\_6:Th56}}]\MizThm{GROUP\_6}{56}
  Let $\varphi\colon G_{1}\to G_{2}$ is a group morphism.
  Then $\varphi$ is injective if and only if $\ker(\varphi)=\trivialSubgroup$.
\end{theorem}

\begin{theorem}[{Dummit and Foote~\cite[Cor.~4.14]{dummit-foote}}]
  If $H\subgroup G$ is any subgroup and $g\in G$, then
  $H$ and $gHg^{-1}$ are isomorphic. Conjugate elements have the same
  order. Conjugate subgroups have the same order.
\end{theorem}

\begin{thm-remark}
Mizar has the following results:
\begin{itemize}
\item \verb#GROUP_3:Th28#\MizTh{GROUP\_3}{28} state, for any group elements $a,b\in G$ and
  any integer $k\in\ZZ$, we have $(b^{-1}ab)^{k} = b^{-1}(a^{k})b$;
\item \verb#GROUP_3:Th64#\MizTh{GROUP\_3}{64} proves conjugate subgroups have the same order;
\item \verb#GROUP_3:Th71#\MizTh{GROUP\_3}{71} proves conjugate subgroups in $G$ have the
  same index;
\item \verb#GROUP_8:Th16#\MizTh{GROUP\_8}{16} proves, for any nonempty subset $A\subset G$,
  its conjugates $g^{-1}Ag$ (for any $g\in G$) has the same cardinality
  as $A$\dots which should imply conjugate subgroups have the same
  order, too.
\end{itemize}
\end{thm-remark}

\begin{theorem}\label{thm:pure-math:iso-subgroups-have-same-order}
  Let $\varphi\in\Aut(G)$ be any automorphism of a group $G$.
  If $H\subgroup G$ is a subgroup, then the order of $H$ and
  $\varphi(H)$ are the same.
\end{theorem}

\begin{proof}[Proof sketch]
This follows from $\varphi$ being a bijection.
\end{proof}

\begin{thm-remark}
Mizar as a stronger result, since \verb#GROUP_6:73#\MizTh{GROUP\_6}{73} proves isomorphic
groups have the same order.
\end{thm-remark}

\begin{theorem}[{Gorenstein~\cite[{Th.2.1.5}]{gorenstein1980finite}}]
  If $H$ is a minimal normal subgroup of $G$ (i.e., $H\normalSubgroup G$
  and there is no nontrivial proper subgroup of $H$ which is normal in
  $G$),
  then \emph{either} $H$ is an elementary Abelian $p$-group (for some
  prime $p\in\NN$) \emph{or} $H$ is the direct product of isomorphic
  non-Abelian simple groups.
\end{theorem}

The class formula\index{Class Formula} for groups is proven in~\citemml{arneson2003witt}.

\section{Commutators}

\begin{definition}[{\texttt{GROUP\_5:def 2,3}}]\index{Commutator}
  Let $G$ be a group and let $x$, $y\in G$ be arbitrary.
  Then the \define{Commutator} of $x$ with $y$ is
  \begin{equation*}
    [x,y] = x^{-1}y^{-1}xy.
  \end{equation*}
  We abuse notation and write $[x,y,z] = [{[x,y]},z]$,
  $[w,x,y,z] = [{[w,x,y]},z]$, and so on.
\end{definition}

\begin{def-remark}
Mizar defines its commutators, $[x,y]$ in \verb#GROUP_5:def 2#\MizDef{GROUP\_5}[002]{2}, and
$[x,y,z]$ in \verb#GROUP_5:def 3#.\MizDef{GROUP\_5}[003]{3}
\end{def-remark}

\begin{def-remark}
This convention, while used in Mizar (see definition 2 in \verb#GROUP_5#,
lines 329--334) and finite group theory, clashes with the convention
with Lie groups where they have $[X,Y]_{\text{Lie}}=XYX^{-1}Y^{-1}$ to
have the Lie bracket for the corresponding Lie algebra coincide with the
commutator $[x,y] = xy - yx$. However, one advantage of the finite group
version is that $[g,h] = g^{-1}g^{h}$ where $g^{h}=h^{-1}gh$ is
conjugation of $g$ by $h$. \emph{Extreme care must be taken when working
with finite groups of Lie type!}
\end{def-remark}

\begin{notation}
  Mizar denotes the commutator $[x,y]$ as \lstinline![.x,y.]! where
  ``\verb#[.#'' and ``\verb#.]#'' are left and right functor brackets
  introduced in the ``\verb#XXREAL_1#'' vocabulary.
\end{notation}

\begin{definition}\index{$G'$}\index{Subgroup!Derived}\index{$[G,G]$}
Let $G$ be a group. The \define{Derived Subgroup} of $G$ is the subgroup
denoteed $G'$ or $[G,G]$ generated by commutators of elements of $G$, where
for generic subsets $X,Y\subset G$ we denote
\begin{equation}
  [X,Y] = \langle [x,y] : x\in X,y\in Y\rangle.
\end{equation}
We also use the notation $[X,Y,Z] = [\,{[X,Y]},Z]$, $[W,X,Y,Z] = [\,{[W,X,Y]},Z]$,
and so on. In particular, $[X,Y,Z] = \langle [a,z] : a\in[X,Y], z\in Z\rangle$.
\end{definition}

\begin{def-remark}
This is defined for arbitrary subsets $A\subset G$ and $B\subset G$ in
\verb#GROUP_5:def 7#,\MizDef{GROUP\_5}[07]{7} and for subgroups $H_{1}\subgroup G$ and
$H_{2}\subgroup G$ in \verb#GROUP_5:def 8#.\MizDef{GROUP\_5}[08]{8} The derived subgroup is
denoted ``\verb#G `#'' (that is the letter G followed by a
backtick/back-quote [grave accent]) and defined in \verb#GROUP_5:def 9#.\MizDef{GROUP\_5}[09]{9}\mizindex{G'@\texttt{G`}}
\end{def-remark}


\begin{theorem}[Hall's Three Subgroup Lemma]
  Let $H$, $K$, $L$ be subgroups of $G$.
  If $[H,K,L]=\trivialSubgroup$ and $[K,L,H]=\trivialSubgroup$,
  then $[L,H,K]=\trivialSubgroup$.
\end{theorem}

\section{Characteristic Subgroups}

\begin{definition}[{Dummit and Foote~\cite[\S4.4]{dummit-foote}}]\index{Subgroup!Characteristic}\index{Characteristic!Subgroup}
A subgroup $H$ of $G$ is called \define{Characteristic} in $G$, usually
denoted $H~\mathrm{char}~G$, if every Automorphism of $G$ maps $H$ to
itself; i.e., $\sigma(H)=H$ for all $\sigma\in\aut(G)$.
\end{definition}

\begin{theorem}
For any group $G$, its center $Z(G)$ is a characteristic subgroup.
\end{theorem}

\begin{theorem}
If $K\subgroup H$ is characteristic, and if $H\normalSubgroup G$ is normal,
then $K\normalSubgroup G$ is normal.
\end{theorem}

\begin{theorem}
  If $H$ is the unique subgroup of a given order in a group $G$,
  then $H$ is characteristic in $G$.
\end{theorem}

\begin{proof}
Let $H$ be a subgroup of $G$. Assume there are no other subgroups of
order $|H|$. Then for any $\varphi\in\Aut(G)$, we'd have $\varphi(H)=H$
since $\varphi(H)$ has the same order as $H$ by
Theorem~\ref{thm:pure-math:iso-subgroups-have-same-order}, but we
assumed there is only one (namely, $H$).
\end{proof}

\begin{theorem}[{Gorenstein~\cite[Th.2.1.3]{gorenstein1980finite}}]
Let $G$ be a group. If a normal subgroup $H\normalSubgroup G$ whose
order and index are coprime
\begin{equation*}
\gcd(|H|, [G:H])=1,
\end{equation*}
then $H$ is a characteristic subgroup of $G$.
\end{theorem}

\begin{theorem}[{Gorenstein~\cite[Th.2.1.3(iv)]{gorenstein1980finite}}]
If  $H\subgroup K$ are both subgroups of $G$ such that $H$ is
a characteristic subgroup of $G$, and if $K/H$ is characteristic in
$G/H$, \emph{then} $K$ is characteristic in $G$.
\end{theorem}

\begin{definition}\index{$\SylowSubgroups{p}{G}$}\index{p-Core@$p$-Core}\index{$\pCore{p}{G}$}\index{$\pCore{p}{G}$|see{$p$-Core}}
Let $G$ be a group, $p\in\NN$ a prime, and $\SylowSubgroups{p}{G}$ the
collection of Sylow $p$-subgroups of $G$. The \define{$p$-Core} of $G$
is the Subgroup given by the intersection of all the Sylow $p$-subgroups
\begin{equation*}
\pCore{p}{G} = \bigcap \SylowSubgroups{p}{G}.
\end{equation*}
\end{definition}

\begin{theorem}
The $p$-core for any group $G$ is a characteristic subgroup.
\end{theorem}

\section{Finite Fields}

It seems that Galois Fields\index{Field!Galois} are not defined in
general, not anywhere I could find.

\begin{example}[{$\FF_{3}$, \texttt{MOD\_2:def 20}\MizDef{MOD\_2}{20}}]
The field with three elements $\FF_{3}$ is defined as \lstinline!Z_3!
in \texttt{MOD\_2:def 20}\MizDef{MOD\_2}{20}.
\end{example}

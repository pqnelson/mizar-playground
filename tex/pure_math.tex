\chapter{Pure Math}\label{chapter:pure-math}

This is just a grab bag of random results from the literature. A good
introduction to finite group theory would be I.~Martin
Isaacs~\cite{isaacs2008finite} for readers who have taken introductory
abstract algebra, Aschbacher~\cite{aschbacher2000finite} for a summary
of the tools necessary for proving the classification of quasithin groups~\cite{aschbacher2004classification1,aschbacher2004classification2}.
For a summary of finite simple groups with pointers to the literature,
Wilson~\cite{wilson2009finite} does a great job.


\begin{notation}
Most notation is following the literature, but just to
review them here:
\begin{itemize}%\setlength\itemsep{0em}
\item The trivial group is denoted $\trivialGroup$.
\item If $H$ is a subgroup of $G$, we write $H\subgroup G$; and if
  further $H$ is a proper subgroup of $G$, we write $H\properSubgroup G$.
\item For any group $G$, its proper trivial group is denoted
  $\trivialSubgroup_{G}\subgroup G$ or (if there is no confusion) just
  $\trivialSubgroup$.
\item Normal subgroups are denoted by $N\normalSubgroup G$ and proper
  normal subgroups are $N\properNormalSubgroup G$.
\item If $g$, $h\in G$ are group elements, we denote the conjugate of $g$ by
$h$ as $g^{h}=h^{-1}gh$.
\item Like any self-respecting American, I do not consider zero to be a
natural number. Thus $\NN=\{1,2,3,\dots\}$. When I need to refer to the
non-negative integers, I write $\NN_{0}=\{0\}\cup\NN$.
\item The other number systems are with the usual notation ($\ZZ$ for
  integers, $\QQ$ for rationals, $\RR$ for reals, $\CC$ for complex numbers).
\end{itemize}
\end{notation}

\begin{theorem}[{\mml[Th41]{group6}}]\MizThm{GROUP\_6}{41}
  Let $\varphi\colon G_{1}\to G_{2}$ is a group morphism, and
  $e_{G_{2}}\in G_{2}$ be the identity element.
  Then $g\in\ker(h)$ iff $\varphi(g)=e_{G_{2}}$.
\end{theorem}

\begin{theorem}[{\mml[Th56]{group6}}]\MizThm{GROUP\_6}{56}
  Let $\varphi\colon G_{1}\to G_{2}$ is a group morphism.
  Then $\varphi$ is injective if and only if $\ker(\varphi)=\trivialSubgroup$.
\end{theorem}


\begin{theorem}\label{thm:pure-math:iso-subgroups-have-same-order}
  Let $\varphi\in\Aut(G)$ be any automorphism of a group $G$.
  If $H\subgroup G$ is a subgroup, then the order of $H$ and
  $\varphi(H)$ are the same.
\end{theorem}

\begin{proof}[Proof sketch]
This follows from $\varphi$ being a bijection.
\end{proof}

\begin{thm-remark}
Mizar as a stronger result, since
\cite[Th73]{group6}\MizTh{GROUP\_6}{73} proves isomorphic groups have the same order.
\end{thm-remark}

\begin{theorem}[{Gorenstein~\cite[{Th.2.1.5}]{gorenstein1980finite}}]
  If $H$ is a minimal normal subgroup of $G$ (i.e., $H\normalSubgroup G$
  and there is no nontrivial proper subgroup of $H$ which is normal in
  $G$),
  then \emph{either} $H$ is an elementary Abelian $p$-group (for some
  prime $p\in\NN$) \emph{or} $H$ is the direct product of isomorphic
  non-Abelian simple groups.
\end{theorem}

The class formula\index{Class Formula} for groups is proven in~\mml{weddwitt}.

\section{Characteristic Subgroups}

\begin{definition}[{Dummit and Foote~\cite[\S4.4]{dummit-foote}}]\index{Subgroup!Characteristic}\index{Characteristic!Subgroup}
A subgroup $H$ of $G$ is called \define{Characteristic} in $G$, usually
denoted $H~\mathrm{char}~G$, if every Automorphism of $G$ maps $H$ to
itself; i.e., $\sigma(H)=H$ for all $\sigma\in\aut(G)$.
\end{definition}

\begin{theorem}[{Gorenstein~\cite[Th.2.1.3]{gorenstein1980finite}}]
Let $G$ be a group. If a normal subgroup $H\normalSubgroup G$ whose
order and index are coprime
\begin{equation*}
\gcd(|H|, [G:H])=1,
\end{equation*}
then $H$ is a characteristic subgroup of $G$.
\end{theorem}

\begin{theorem}[{Gorenstein~\cite[Th.2.1.3(iv)]{gorenstein1980finite}}]
If  $H\subgroup K$ are both subgroups of $G$ such that $H$ is
a characteristic subgroup of $G$, and if $K/H$ is characteristic in
$G/H$, \emph{then} $K$ is characteristic in $G$.
\end{theorem}

\begin{definition}\index{$\SylowSubgroups{p}{G}$}\index{p-Core@$p$-Core}\index{$\pCore{p}{G}$}\index{$\pCore{p}{G}$|see{$p$-Core}}
Let $G$ be a group, $p\in\NN$ a prime, and $\SylowSubgroups{p}{G}$ the
collection of Sylow $p$-subgroups of $G$. The \define{$p$-Core} of $G$
is the Subgroup given by the intersection of all the Sylow $p$-subgroups
\begin{equation*}
\pCore{p}{G} = \bigcap \SylowSubgroups{p}{G}.
\end{equation*}
\end{definition}

\begin{theorem}
The $p$-core for any group $G$ is a characteristic subgroup.
\end{theorem}

\section{Finite Fields}

It seems that Galois Fields\index{Field!Galois} are not defined in
general, not anywhere I could find. Unfolding the definitions, we have
Definition~\mml[def10,11,12]{vectsp1} defined \lstinline{Skew-Field} as
\lstinline{non degenerated almost_left_invertible Ring}, and
\lstinline{Field} is \lstinline{commutative Skew-Field}.

\begin{example}[{$\FF_{3}$}]
The field with three elements $\FF_{3}$ is defined as \lstinline!Z_3!
in \mml[def 20]{mod2}\MizDef{MOD\_2}{20}.
\end{example}

\begin{example}[{$\ZZ/n\ZZ$}]
The integers modulo $n$ form a ring in Definition~\mml[def12]{int3} as
\lstinline{INT.Ring(n)}. I think Theorem~\mml[Th12]{int3} suffices to
prove \lstinline{INT.Ring(p)} is a field.
\end{example}

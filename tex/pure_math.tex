\chapter{Pure Math}\label{chapter:pure-math}

This is just a grab bag of random results from the literature. A good
introduction to finite group theory would be I.~Martin
Isaacs~\cite{isaacs2008finite} for readers who have taken introductory
abstract algebra, Aschbacher~\cite{aschbacher2000finite} for a summary
of the tools necessary for proving the classification of quasithin groups~\cite{aschbacher2004classification1,aschbacher2004classification2}.
For a summary of finite simple groups with pointers to the literature,
Wilson~\cite{wilson2009finite} does a great job.


\begin{notation}
Most notation is following the literature, but just to
review them here:
\begin{itemize}%\setlength\itemsep{0em}
\item The trivial group is denoted $\trivialGroup$.
\item If $H$ is a subgroup of $G$, we write $H\subgroup G$; and if
  further $H$ is a proper subgroup of $G$, we write $H\properSubgroup G$.
\item For any group $G$, its proper trivial group is denoted
  $\trivialSubgroup_{G}\subgroup G$ or (if there is no confusion) just
  $\trivialSubgroup$.
\item Normal subgroups are denoted by $N\normalSubgroup G$ and proper
  normal subgroups are $N\properNormalSubgroup G$.
\item If $g$, $h\in G$ are group elements, we denote the conjugate of $g$ by
$h$ as $g^{h}=h^{-1}gh$.
\item Like any self-respecting American, I do not consider zero to be a
natural number. Thus $\NN=\{1,2,3,\dots\}$. When I need to refer to the
non-negative integers, I write $\NN_{0}=\{0\}\cup\NN$.
\item The other number systems are with the usual notation ($\ZZ$ for
  integers, $\QQ$ for rationals, $\RR$ for reals, $\CC$ for complex numbers).
\end{itemize}
\end{notation}

\begin{theorem}[{\mml[Th41]{group6}}]\MizThm{GROUP\_6}{41}
  Let $\varphi\colon G_{1}\to G_{2}$ is a group morphism, and
  $e_{G_{2}}\in G_{2}$ be the identity element.
  Then $g\in\ker(h)$ iff $\varphi(g)=e_{G_{2}}$.
\end{theorem}

\begin{theorem}[{\mml[Th56]{group6}}]\MizThm{GROUP\_6}{56}
  Let $\varphi\colon G_{1}\to G_{2}$ is a group morphism.
  Then $\varphi$ is injective if and only if $\ker(\varphi)=\trivialSubgroup$.
\end{theorem}


\begin{theorem}\label{thm:pure-math:iso-subgroups-have-same-order}
  Let $\varphi\in\Aut(G)$ be any automorphism of a group $G$.
  If $H\subgroup G$ is a subgroup, then the order of $H$ and
  $\varphi(H)$ are the same.
\end{theorem}

\begin{proof}[Proof sketch]
This follows from $\varphi$ being a bijection.
\end{proof}

\begin{thm-remark}
Mizar as a stronger result, since
\cite[Th73]{group6}\MizTh{GROUP\_6}{73} proves isomorphic groups have the same order.
\end{thm-remark}

\begin{theorem}[{Gorenstein~\cite[{Th.2.1.5}]{gorenstein1980finite}}]
  If $H$ is a minimal normal subgroup of $G$ (i.e., $H\normalSubgroup G$
  and there is no nontrivial proper subgroup of $H$ which is normal in
  $G$),
  then \emph{either} $H$ is an elementary Abelian $p$-group (for some
  prime $p\in\NN$) \emph{or} $H$ is the direct product of isomorphic
  non-Abelian simple groups.
\end{theorem}

\begin{proposition}
The class formula\index{Class Formula} for groups is proven in~\mml{weddwitt}.
\end{proposition}

\begin{definition}
Let $\pi$ be a set of prime numbers. We call a group $G$ a
\define{$\pi$-Group} if the set of the prime divisors of the order of
$G$, denoted $\pi(G)$, is a subset of $\pi\supset\pi(G)$.
\end{definition}

Observe that if $\pi=\{p\}$ is a single prime number, then we recover
the notion of a $p$-group.

\section{Characteristic Subgroups}

\begin{definition}[{Dummit and Foote~\cite[\S4.4]{dummit-foote}}]\index{Subgroup!Characteristic}\index{Characteristic!Subgroup}
A subgroup $H$ of $G$ is called \define{Characteristic} in $G$, usually
denoted $H~\mathrm{char}~G$, if every Automorphism of $G$ maps $H$ to
itself; i.e., $\sigma(H)=H$ for all $\sigma\in\aut(G)$.
\end{definition}

\begin{theorem}[{Gorenstein~\cite[Th.2.1.3]{gorenstein1980finite}}]
Let $G$ be a group. If a normal subgroup $H\normalSubgroup G$ whose
order and index are coprime
\begin{equation*}
\gcd(|H|, [G:H])=1,
\end{equation*}
then $H$ is a characteristic subgroup of $G$.
\end{theorem}

\begin{theorem}[{Gorenstein~\cite[Th.2.1.3(iv)]{gorenstein1980finite}}]
If  $H\subgroup K$ are both subgroups of $G$ such that $H$ is
a characteristic subgroup of $G$, and if $K/H$ is characteristic in
$G/H$, \emph{then} $K$ is characteristic in $G$.
\end{theorem}

\begin{lemma}
Let $G$ be a finite group, $H$ be a Sylow $p$-subgroup of $G$. Then for
any automorphism $\varphi\in\Aut(G)$, $\varphi(H)$ is a Sylow
$p$-subgroup of $G$.
\end{lemma}

\begin{theorem}
The $p$-core for any group $G$ is a characteristic subgroup.
\end{theorem}

\begin{definition}\label{defn:pure-math:X-residual}
Let $G$ be a finite group, let $\mathcal{X}$ be a class of finite
subgroups of $G$ which is closed under isomorphisms (in particular,
automorphisms of $G$), quotients, subgroups, and finite direct products.
Then the \define{$\mathcal{X}$-Residual} of $G$ is the subgroup
\begin{equation}
  O^{\mathcal{X}}(G) := \bigcap\{N\normalSubgroup G\mid G/N\in\mathcal{X}\}.
\end{equation}
\end{definition}

\begin{def-remark}
This definition seems to be folklore. Indeed, I only discovered it by
accident from the internet.\footnote{I am indebted to Jack Schmidt's
post about it here: \url{https://math.stackexchange.com/a/216961/31693}}
\end{def-remark}

\begin{theorem}
The $\mathcal{X}$-Residual is the unique normal subgroup of $G$ such
that $G/N\in\mathcal{X}$ if and only if $O^{\mathcal{X}}(G)\subgroup N$.
\end{theorem}

\begin{theorem}
For any group $G$ and family of subgroups $\mathcal{X}\neq\emptyset$,
the $\mathcal{X}$-residual of $G$ is a characteristic subgroup of $G$.
\end{theorem}

\section{Permutation Groups}

\begin{theorem}[Orre]
The derived subgroup of $S_{n}$ is the alternating group $A_{n}$.
\end{theorem}

\section{Finite Fields}

It seems that Galois Fields\index{Field!Galois} are not defined in
general, not anywhere I could find. Well, true to form, Mizar defines
the finite rings $\ZZ/n\ZZ$ as \lstinline{INT.Ring(n)}. And as far as
Mizar cares, a field is just a ring satisfying extra properties. So the
finite ring \lstinline{INT.Ring(p)} is provably a field, for example.

Unfolding the definitions, we have Definition~\mml[def10,11,12]{vectsp1}
first define \lstinline{Skew-Field} as a ring satisfying extra properties
(i.e., \lstinline{non degenerated} \lstinline{almost_left_invertible} \lstinline{Ring}).
Then a \lstinline{Field} is \lstinline{commutative Skew-Field}.

\begin{example}[{$\FF_{3}$}]
The field with three elements $\FF_{3}$ is defined as \lstinline!Z_3!
in \mml[def 20]{mod2}\MizDef{MOD\_2}{20}.
\end{example}

\begin{example}[{$\ZZ/n\ZZ$}]
The integers modulo $n$ form a ring in Definition~\mml[def12]{int3} as
\lstinline{INT.Ring(n)}. I think Theorem~\mml[Th12]{int3} suffices to
prove \lstinline{INT.Ring(p)} is a field. Also Registration~\mml[reg14]{int3}
associated \lstinline{INT.Ring(p)} with being\dots well, many things,
but they combine together to make it a finite field.
\end{example}

\begin{exercise}
Prove \lstinline{INT.Ring(p |^ n)} is a field where ``\lstinline{p |^ n}''
is Mizar for $p^{n}$ (where $p,n\in\NN$ and $p$ is prime).
\end{exercise}

\section{Finite Groups of Lie Type}

\begin{ddanger}
The notation for finite groups of Lie type is bewildering to people who
know Lie groups. This is because Artin~\cite{artin1955order} introduced
idiosyncratic notation for finite groups of Lie type, which diverges
quite badly from the notational choices made in the development of Lie
groups. 
\end{ddanger}

\section{Finite Simple Groups}

\begin{proposition}\label{prop:pure-math:cfsg}
There are four types of finite simple groups:
\begin{enumerate}
\item the finite cyclic groups of prime order;
\item the alternating groups $A_{n}$ for $n\geq 5$;
\item the finite groups of Lie type; and
\item the Sporadic groups (i.e., ``the 26 whacky groups'')
\end{enumerate}
Moreover, any finite simple group must be one of these four types, with
the only repetitions being:
\begin{subequations}
  \begin{align}
    \PSL_{2}(4)&\iso A_{5}\\
    \PSL_{2}(5)&\iso A_{5}\\
    \PSL_{2}(7)&\iso\PSL_{3}(2)\\
    \PSL_{2}(9)&\iso A_{6}
  \end{align}
\end{subequations}\vskip-2\belowdisplayskip\vskip-2\abovedisplayskip
  \begin{align}
    \PSL_{4}(2)&\iso A_{8}\\
    \PSU_{4}(2)&\iso\PSp_{4}(3).
  \end{align}
\end{proposition}

\begin{definition}
We say a group $G$ is a \define{$\mathcal{K}$-Group} if it is one of the
simple groups as listed in Proposition~\ref{prop:pure-math:cfsg}.
\end{definition}

\subsection{Sporadic Groups}
These are the simple groups which are not finite groups of Lie type,
cyclic, or alternating. Their construction is non-uniform, since they
are symmetries of random exotic mathematical objects.

\begin{definition}
  We call a finite group $G$ \define{Sporadic} if
  \begin{enumerate}
  \item $G$ is a simple group,
  \item $G$ is not cyclic,
  \item $G$ is not of Lie type,
  \item $G$ is not an alternating group.
  \end{enumerate}
\end{definition}

\begin{def-remark}[References]
For actually \emph{constructing} sporadic groups,
Aschbacher~\cite{aschbacher1994sporadic} provides a concise series of
constructions, Wilson~\cite{wilson2009finite} provides elegant
constructions. Griess~\cite{griess1998twelve} reviews 12 sporadic
groups.
\end{def-remark}

\begin{def-remark}[Organization of Sporadics]
There are three ``generations'' used to organize sporadic groups plus
the six ``pariahs'':
\begin{enumerate}
\item First Generation (the Mathieu groups): $\Mathieu{11}$,
  $\Mathieu{12}$, $\Mathieu{22}$, $\Mathieu{23}$, $\Mathieu{24}$;
\item Second Generation (the Leech Lattice groups): the Conway groups
  $\Conway{1}$, $\Conway{2}$, $\Conway{3}$; the McLaughlin group
  $\McLaughlin$; the Higman--Sims group $\HigmanSims$; and the Janko
  group $\Janko{2}$;
\item Third Generation (other Subgroups of the Monster $\Monster$): the most
  relevant one for quasithin classification being the Held group
  $\Held$;
\item Pariahs: three Janko groups $\Janko{1}$, $\Janko{3}$, $\Janko{4}$;
  the O'Nan group $\ONan$; the Rudvalis group $\Rudvalis$; and Lyons
  group $\Lyons$.
\end{enumerate}
They are very exciting and beautiful, but I am not going to investigate
\emph{all} of them further.
\end{def-remark}

\begin{def-remark}[Quasithin Sporadics]
According to Aschbacher and Smith~\cite{aschbacher2004classification1,aschbacher2004classification2}, the
sporadic groups which are quasithin include:
$\Mathieu{11}$, $\Mathieu{12}$, $\Mathieu{22}$,
$\Mathieu{23}$, $\Mathieu{24}$, $\Janko{2}$, $\Janko{3}$, $\Janko{4}$,
$\HigmanSims$, $\Held$, $\Rudvalis$. Clustered by generation:
\begin{enumerate}
\item all five Mathieu groups ($\Mathieu{11}$, $\Mathieu{12}$, $\Mathieu{22}$, $\Mathieu{23}$,
  $\Mathieu{24}$);
\item a couple Leech Lattice groups ($\HigmanSims$, $\Janko{2}$);
\item from the Monster-related sporadics, only the Held group $\Held$; and
\item $\Janko{3}$, $\Janko{4}$, $\Rudvalis$.
\end{enumerate}
Also note: unlike the Mathieu groups, the Conway groups, or the Fischer
groups, the Janko groups do not form a series or ``family'' --- they are
just a few ``random'' groups constructed by the same mathematician.
\end{def-remark}

\begin{proposition}
Finite non-Abelian simple quasithin $\mathcal{K}$-groups of even
characteristic consist of:
\begin{enumerate}
\item (Generic case) Groups of Lie type of characteristic 2 and Lie rank at most 2, but
  $U_{5}(q)$ only for $q=4$
\item (Certain groups of rank 3 or 4) $L_{4}(2)$, $L_{5}(2)$, $\Sp_{6}(2)$
\item (Alternating groups) $A_{5}$, $A_{6}$, $A_{8}$, $A_{9}$
\item (Lie type of odd characteristic) $L_{2}(p)$ when $p$ is a Mersenne
  or Fermat prime, $L_{3}^{\epsilon}(3)$, $L_{4}^{\epsilon}(3)$, $G_{2}(3)$
\item (Sporadic) $\Mathieu{11}$, $\Mathieu{12}$, $\Mathieu{22}$,
$\Mathieu{23}$, $\Mathieu{24}$, $\Janko{2}$, $\Janko{3}$, $\Janko{4}$,
$\HigmanSims$, $\Held$, $\Rudvalis$
\end{enumerate}
\end{proposition}



\chapter{Pure Math}

\M
This is just a grab bag of random results from the literature.

\N{Notation} Most notation is following the literature, but just to
review them here:
\begin{itemize}%\setlength\itemsep{0em}
\item The trivial group is denoted $\trivialGroup$.
\item If $H$ is a subgroup of $G$, we write $H\subgroup G$; and if
  further $H$ is a proper subgroup of $G$, we write $H\properSubgroup G$.
\item For any group $G$, its proper trivial group is denoted
  $\trivialSubgroup_{G}\subgroup G$ or (if there is no confusion) just
  $\trivialSubgroup$.
\item Normal subgroups are denoted by $N\normalSubgroup G$ and proper
  normal subgroups are $N\properNormalSubgroup G$.
\item If $g$, $h\in G$ are group elements, we denote the conjugate of $g$ by
$h$ as $g^{h}=h^{-1}gh$.
\item Like any self-respecting American, I do not consider zero to be a
natural number. Thus $\NN=\{1,2,3,\dots\}$. When I need to refer to the
non-negative integers, I write $\NN_{0}=\{0\}\cup\NN$.
\end{itemize}

\begin{theorem}[{Dummit and Foote~\cite[Cor.~4.14]{dummit-foote}}]
  If $H\subgroup G$ is any subgroup and $g\in G$, then
  $H$ and $gHg^{-1}$ are isomorphic. Conjugate elements have the same
  order. Conjugate subgroups have the same order.
\end{theorem}

\begin{theorem}\label{thm:pure-math:iso-subgroups-have-same-order}
  Let $\varphi\in\Aut(G)$ be any automorphism of a group $G$.
  If $H\subgroup G$ is a subgroup, then the order of $H$ and
  $\varphi(H)$ are the same.
\end{theorem}

\begin{proof}[Proof sketch]
This follows from $\varphi$ being a bijection.
\end{proof}

\begin{theorem}[{Gorenstein~\cite[{Th.1.5} of {ch.2}\,\S1]{gorenstein1980finite}}]
  If $H$ is a minimal normal subgroup of $G$ (i.e., $H\normalSubgroup G$
  and there is no nontrivial proper subgroup of $H$ which is normal in
  $G$),
  then \emph{either} $H$ is an elementary Abelian $p$-group (for some
  prime $p\in\NN$) \emph{or} $H$ is the direct product of isomorphic
  non-Abelian simple groups.
\end{theorem}

\section{Commutators}

\begin{definition}
  Let $G$ be a group and let $x$, $y\in G$ be arbitrary.
  Then the \define{Commutator} of $x$ with $y$ is
  \begin{equation*}
    [x,y] = x^{-1}y^{-1}xy.
  \end{equation*}
  We abuse notation and write $[x,y,z] = [{[x,y]},z]$,
  $[w,x,y,z] = [{[w,x,y]},z]$, and so on.
\end{definition}

\begin{remark}
This convention, while used in Mizar (see definition 2 in \verb#GROUP_5#,
lines 329--334) and finite group theory, clashes with the convention
with Lie groups where they have $[X,Y]_{\text{Lie}}=XYX^{-1}Y^{-1}$ to
have the Lie bracket for the corresponding Lie algebra coincide with the
commutator $[x,y] = xy - yx$. However, one advantage of the finite group
version is that $[g,h] = g^{-1}g^{h}$ where $g^{h}=h^{-1}gh$ is
conjugation of $g$ by $h$. \emph{Extreme care must be taken when working
with finite groups of Lie type!}
\end{remark}

\begin{definition}
Let $G$ be a group. The \define{Derived Subgroup} of $G$ is the subgroup
denoteed $G'$ or $[G,G]$ generated by commutators of elements of $G$, where
for generic subsets $X,Y\subset G$ we denote
\begin{equation}
  [X,Y] = \langle [x,y] : x\in X,y\in Y\rangle.
\end{equation}
We also use the notation $[X,Y,Z] = [\,{[X,Y]},Z]$, $[W,X,Y,Z] = [\,{[W,X,Y]},Z]$,
and so on. In particular, $[X,Y,Z] = \langle [a,z] : a\in[X,Y], z\in Z\rangle$.
\end{definition}


\begin{theorem}[Hall's Three Subgroup Lemma]
  Let $H$, $K$, $L$ be subgroups of $G$.
  If $[H,K,L]=\trivialSubgroup$ and $[K,L,H]=\trivialSubgroup$,
  then $[L,H,K]=\trivialSubgroup$.
\end{theorem}

\section{Characteristic Subgroups}

\begin{definition}[{Dummit and Foote~\cite[\S4.4]{dummit-foote}}]
A subgroup $H$ of $G$ is called \define{Characteristic} in $G$, usually
denoted $H~\mathrm{char}~G$, if every Automorphism of $G$ maps $H$ to
itself; i.e., $\sigma(H)=H$ for all $\sigma\in\aut(G)$.
\end{definition}

\begin{theorem}
For any group $G$, its center $Z(G)$ is a characteristic subgroup.
\end{theorem}

\begin{theorem}
If $K\subgroup H$ is characteristic, and if $H\normalSubgroup G$ is normal,
then $K\normalSubgroup G$ is normal.
\end{theorem}

\begin{theorem}
  If $H$ is the unique subgroup of a given order in a group $G$,
  then $H$ is characteristic in $G$.
\end{theorem}

\begin{proof}
Let $H$ be a subgroup of $G$. Assume there are no other subgroups of
order $|H|$. Then for any $\varphi\in\Aut(G)$, we'd have $\varphi(H)=H$
since $\varphi(H)$ has the same order as $H$ by
Theorem~\ref{thm:pure-math:iso-subgroups-have-same-order}, but we
assumed there is only one (namely, $H$).
\end{proof}

\begin{theorem}[{Gorenstein~\cite[Th.1.3 of ch.2\S1]{gorenstein1980finite}}]
Let $G$ be a group. If a normal subgroup $H\normalSubgroup G$ whose
order and index are coprime
\begin{equation*}
\gcd(|H|, [G:H])=1,
\end{equation*}
then $H$ is a characteristic subgroup of $G$.
\end{theorem}

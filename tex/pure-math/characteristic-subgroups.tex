\section{Characteristic Subgroups}

\begin{definition}[{Dummit and Foote~\cite[\S4.4]{dummit-foote}}]\index{Subgroup!Characteristic}\index{Characteristic!Subgroup}
A subgroup $H$ of $G$ is called \define{Characteristic} in $G$, usually
denoted $H~\mathrm{char}~G$, if every Automorphism of $G$ maps $H$ to
itself; i.e., $\sigma(H)=H$ for all $\sigma\in\aut(G)$.
\end{definition}

\begin{theorem}[{Gorenstein~\cite[Th.2.1.3]{gorenstein1980finite}}]
Let $G$ be a group. If a normal subgroup $H\normalSubgroup G$ whose
order and index are coprime
\begin{equation*}
\gcd(|H|, [G:H])=1,
\end{equation*}
then $H$ is a characteristic subgroup of $G$.
\end{theorem}

\begin{theorem}[{Gorenstein~\cite[Th.2.1.3(iv)]{gorenstein1980finite}}]
If  $H\subgroup K$ are both subgroups of $G$ such that $H$ is
a characteristic subgroup of $G$, and if $K/H$ is characteristic in
$G/H$, \emph{then} $K$ is characteristic in $G$.
\end{theorem}

\begin{lemma}
Let $G$ be a finite group, $H$ be a Sylow $p$-subgroup of $G$. Then for
any automorphism $\varphi\in\Aut(G)$, $\varphi(H)$ is a Sylow
$p$-subgroup of $G$.
\end{lemma}

\begin{theorem}
The $p$-core for any group $G$ is a characteristic subgroup.
\end{theorem}

\begin{definition}\label{defn:pure-math:X-residual}
Let $G$ be a finite group, let $\mathcal{X}$ be a class of finite
subgroups of $G$ which is closed under isomorphisms (in particular,
automorphisms of $G$), quotients, subgroups, and finite direct products.
Then the \define{$\mathcal{X}$-Residual} of $G$ is the subgroup
\begin{equation}
  O^{\mathcal{X}}(G) := \bigcap\{N\normalSubgroup G\mid G/N\in\mathcal{X}\}.
\end{equation}
\end{definition}

\begin{def-remark}
This definition seems to be folklore. Indeed, I only discovered it by
accident from the internet.\footnote{I am indebted to Jack Schmidt's
post about it here: \url{https://math.stackexchange.com/a/216961/31693}}
\end{def-remark}

\begin{theorem}
The $\mathcal{X}$-Residual is the unique normal subgroup of $G$ such
that $G/N\in\mathcal{X}$ if and only if $O^{\mathcal{X}}(G)\subgroup N$.
\end{theorem}

\begin{theorem}
For any group $G$ and family of subgroups $\mathcal{X}\neq\emptyset$,
the $\mathcal{X}$-residual of $G$ is a characteristic subgroup of $G$.
\end{theorem}

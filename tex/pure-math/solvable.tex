\section{Solvable Groups}

This is the formal proof sketches of \S9 of Aschbacher~\cite{aschbacher2000finite}.

\begin{definition}[{\cite[\S8]{aschbacher2000finite}}]
We define the $n^{\text{th}}$-derived subgroup (for any $n\in\NN$) as:
\begin{enumerate}
\item $G^{(0)} = G$
\item $G^{(1)} = [G, G]$ 
\item $G^{(n+1)} = [G^{(n)}, G^{(n)}]$.
\end{enumerate}
\end{definition}

\begin{theorem}[{\cite[(9.1)]{aschbacher2000finite}}]
A group $G$ is solvable if and only if there exists some positive
$n\in\NN$ such that $G^{(n)}=\trivialGroup$.
\end{theorem}

\begin{theorem}[{\cite[(9.2)]{aschbacher2000finite}}]
A finite group is solvable if and only if all its composition factors
are of prime order.
\end{theorem}

\begin{theorem}
  \begin{enumerate}
  \item If $G$ is a solvable group and $\varphi\colon G\to H$ is a group
morphism, then $\varphi(G)$ is a solvable group.
\item If $G$ is solvable and $H\subgroup G$, then $H$ is solvable.
\item Let $H\normalSubgroup G$. If $H$ is solvable and $G/H$ is
  solvable, then $G$ is solvable.
  \end{enumerate}
\end{theorem}

\begin{thm-remark}
This first claim is proven in Theorem~\mml[Th16]{grsolv1},
the second claim is proven in Theorem~\mml[Th5]{grsolv1}.
\end{thm-remark}

\begin{definition}
Let $G$ be a group. We call a subgroup $H$ a \define{Minimal Normal}
subgroup if $H\normalSubgroup G$ and for any $N\normalSubgroup G$
such that $N\subgroup H$ we necessarily have $N=\trivialGroup$.
\end{definition}

\begin{mizar}
definition
  let G be Group;
  let IT be Subgroup of G;
  attr IT is minimal_normal means
  IT is normal & for N being strict normal Subgroup of G
  st N is Subgroup of IT holds N=(1).G;
end;

registration
  let G be Group;
  cluster minimal_normal for Subgroup of G;
  existence; :: take (1).G;
end;

registration
  let G be Group;
  cluster minimal_normal -> normal for Subgroup of G;
  correctness;
end;
\end{mizar}

\begin{definition}
Let $p\in\NN$ be prime.
A \define{Elementary Abelian} $p$-group is a group whose non-identity
elements are order $p$.
\end{definition}

\begin{mizar}
definition
  let p be Nat;
  let IT be p-group Group;
  attr IT is elementary_abelian means
  for g being Element of IT st g <> 1_IT holds ord g = p;
end;

registration
  let p be Prime;
  cluster elementary_abelian for p-group Group;
  existence; :: take INT.Group(p)
end;

registration
  let p be Prime;
  cluster elementary_abelian for solvable p-group Group;
  existence;
end;
\end{mizar}

\begin{theorem}[{\cite[(9.4)]{aschbacher2000finite}}]
Solvable minimal normal subgroups of finite groups are elementary
abelian $p$-groups.
\end{theorem}

\begin{mizar}
theorem
  for G being finite Group
  for H being solvable minimal_normal Subgroup of G
  ex p being Prime st H is elementary_abelian p-group Group;
\end{mizar}

\begin{definition}
Let $G$ be a group. The \define{Lower Central Series} is a family of
groups defined recursively as:
\begin{enumerate}
\item $L_{1}(G) = G$
\item $L_{n+1}(G) = [L_{n}(G), G]$.
\end{enumerate}
\end{definition}

\begin{mizar}
definition
  let G be Group;
  func lower_central_series G -> Function means
  dom it = pos Nat & it.1 = G
  & for n being pos Nat holds it.(n+1) = [. (it.n), (Omega).G .];
  existence;
  uniqueness;
end;

definition
  let G be Group;
  synonym L(G) for lower_centra_series G;
end;
\end{mizar}


\begin{theorem} Let $G$ be a group.
\begin{enumerate}
\item $L_{n}(G)$ is a characteristic subgroup of $G$ for each $n\in\NN$
\item $L_{n+1}(G)\subgroup L_{n}(G)$ for each $n\in\NN$
\item $L_{n}(G)/L_{n+1}(G)\subgroup Z(G/L_{n+1}(G))$ for each $n\in\NN$.
\end{enumerate}
\end{theorem}

\begin{mizar}
reserve G for Group;

theorem
  for n being pos Nat
  holds L(G).n is strict characteristic Subgroup of G;

definition
  let G be Group;
  let n be positive Nat;
  redefine (lower_central_series G).n -> strict
  characteristic Subgroup of G;
  correctness;
end;

theorem
  for n being positive Nat
  holds L(G).(n+1) is Subgroup of L(G).n;

theorem
  for n being positive Nat
  for Gn being strict normal Subgroup of L(G).(n+1)
  st Gn = (lower_central_series G).n
  holds ((lower_central_series G).(n+1)) ./. Gn
         = center (G ./. (lower_central_series G).n);
\end{mizar}

\begin{theorem}
Let $G_{1}$, $G_{2}$ be groups, let $\varphi\colon G_{1}\to G_{2}$ be a
group morphism, and let $H_{2}\subgroup G_{2}$ be a subgroup.
Then the pre-image $\varphi^{-1}(H_{2})$ forms a subgroup of $G_{1}$.
\end{theorem}

\begin{mizar}
theorem
  for G1,G2 being Group
  for H2 being Subgroup of G2
  for phi being Homomorphism of G1,G2
  ex H1 being strict Subgroup of G1
  st carr H1 = phi " carr H2;

definition
  let G1,G2 be Group;
  let H2 be Subgroup of G2;
  let phi be Homomorphism of G1,G2;
  func Preimage(phi,H) -> strict Subgroup of G1 means
  carr it = phi " carr H2;
  existence;
  uniqueness;
end;

registration
  let G1,G2 be Group;
  let N be normal Subgroup of G2;
  let phi be Homomorphism of G1,G2;
  cluster Preimage(phi, N) -> normal;
  correctness;
end;
\end{mizar}

\begin{definition}
Let $G$ be a group. We define the \define{Upper Central Series} for $G$
to be a function on $\NN$ defined recursively by:
\begin{enumerate}
\item $Z_{0}(G)=\trivialSubgroup$
\item $Z_{n}$ is the pre-image in $G$ of $Z(G/Z_{n-1}(G))$.
\end{enumerate}
Observe $Z_{1}=Z(G)$ and that each $Z_{n}(G)$ is a characteristic
Subgroup of $G$.
\end{definition}

\begin{def-remark}
Compare this to Definition~\mml[def2]{grnilp1}
\end{def-remark}

\begin{mizar}
definition
  let G be Group;
  func upper_central_series G -> Function of Nat, the_normal_subgroups_of G
  means
  it.0 = (1).G
  & for n being pos Nat
    holds it.(n+1) = Preimage(nat_hom it.n, G./.(it/.n));
  existence;
  uniqueness;
end;
\end{mizar}

\begin{proposition}
For each $n\in\NN$, we see $Z_{n}(G)$ is a characteristic Subgroup of $G$.
\end{proposition}

\begin{mizar}
definition
  let G be Group;
  let n be Nat;
  redefine (upper_central_series G).n -> strict characteristic
  Subgroup of G;
  
  correctness;
end;
\end{mizar}

\begin{definition}
Let $G$ be a nilpotent group. We define the \define{Nilpotence Class}
of $G$ to be the smallest $n$ for which there exists a defining subgroup
series for it.
\end{definition}

\begin{def-remark}
I am trying to be consistent with how Mizar defines a nilpotent group,
which is in terms of the upper central series (but without using that
name). This means we have to define it in terms of the upper central
series.
\end{def-remark}

\begin{mizar}
definition
  let G be nilpotent Group;
  func nilpotence_class G -> pos Nat means
  for F being FinSequence of the_normal_subgroups_of G st
  (len F > 0 & F.1 = (Omega).G & F.(len F) = (1).G &
   (for i being Element of NAT st i in dom F & i+1 in dom F
    for G1, G2 being strict normal Subgroup of G
    st G1 = F.i & G2 = F.(i + 1)
    holds (G2 is Subgroup of G1 &
           G1./.((G1,G2)`*`) is Subgroup of
             center (G./.G2))))
  for n being Nat st n <= len F
  holds n <= it & it <= len F;
end; 
\end{mizar}


\begin{theorem}[{\cite[(9.7)]{aschbacher2000finite}}]
Let $G$ be a nontrivial group.
Then $G$ is nilpotent of class $m$ if and only if $G/Z(G)$ is nilpotent
of class $m-1$.
\end{theorem}

\begin{mizar}
theorem ThNilpotentIffQuotientByCenter:
  G is nilpotent iff G./.(center G) is nilpotent;

registration
  let G be nilpotent Group;
  cluster G./.(center G) -> nilpotent;
  correctness;
end;

theorem
  for G being nilpotent Group
  holds nilpotence_class (G./.(center G)) = nilpotence_class G - 1;
\end{mizar}

\begin{theorem}[{\cite[(5.16)]{aschbacher2000finite}}]
If $G$ is a nontrivial $p$-group, then $Z(G)\neq\trivialSubgroup$.
\end{theorem}

\begin{mizar}
theorem
  for p being Prime
  for G being non trivial p-group Group
  holds center G is non trivial;
\end{mizar}

\begin{theorem}[{\cite[(9.8)]{aschbacher2000finite}}]
Let $p\in\NN$ be prime. Then $p$-groups are nilpotent.
\end{theorem}

\begin{mizar}
theorem
  for p being Prime
  for G being p-group Group
  holds G is nilpotent
proof
  let p be Prime;
  let G be p-group Group;
  per cases;
  suppose G is trivial;
    hence thesis;
  end;
  suppose A1: G is non trivial;
    defpred P[Group] means $1 is p-group;
    defpred Q[Group] means $1 is nilpotent;
    (not Q[G] &
    for H being Group st P[H] & card H < card G holds Q[H])
    implies contradiction
    proof
      assume A2: not Q[G];
      assume A3: for H being Group
                 st P[H] & card H < card G
                 holds Q[H];
      P[center G];
      P[G./.(center G)] & card (G./.(center G)) < card G;
      then Q[G./.(center G)] by A3;
      then Q[G] by ThNilpotentIffQuotientByCenter;
      hence contradiction by A2;
    end;
    hence thesis; :: by minimal counter-example
  end;
end;
  
registration
  let p be Prime;
  cluster p-group -> nilpotent for Group;
  correctness;
end;
\end{mizar}

\begin{thm-remark}
We should take a moment to think about the structure of the proof here,
the ``argument by minimal counter-example'' appears to be a regular
induction proof. Recall well-founded induction looks like: if $x$ is an
element of $X$ and $P[y]$ is true for all $y$ such that $y~R~x$, then
$P[x]$ must also be true.

We are trying to prove for any group $G$ that $P[G]\implies Q[G]$.
We start by proving for any group $H$ such that $H~R~G$ (i.e., $|H|<|G|$)
that $P[H]\implies Q[H]$ holds must imply $P[G]\implies Q[G]$ holds.
This is logically equivalent to $P[G]\land(P[H]\implies Q[H])\implies Q[G]$ 
by Currying. We have assumed $P[G]$. We can Curry again, writing
$Q[G]=(\neg Q[G])\implies\bot$ to make the proof obligation become
$P[G]\land(P[H]\implies Q[H])\land\neg Q[G]\implies\bot$ by Currying,
which is what we have proven.

For this to work in Mizar, I believe we need to define a lattice of
finite groups. We then need to use the scheme ``\lstinline{WFInduction}''
from~\mml{wellfnd1}.
\end{thm-remark}

\begin{mizar}
scheme :: WELLFND1:sch 3
WFInduction{ F1() -> non empty RelStr , P1[ set ] } :
  for x being Element of F1() holds P1[x]
provided
  A1: for x being Element of F1()
      st (for y being Element of F1()
          st y <> x & [y,x] in the InternalRel of F1()
          holds P1[y])
      holds P1[x] and
  A2: F1() is well_founded;
\end{mizar}

\begin{theorem}
If $G$ is a nilpotent group of class $m$, then subgroups and homomorphic
images of $G$ are nilpotent of class at most $m$.
\end{theorem}

\begin{theorem}
If $G$ is nilpotent and $H\subgroup G$ is a subgroup,
then $H\properSubgroup N_{G}(H)$ is a proper subgroup of its normalizer
in $G$.
\end{theorem}

\begin{mizar}
theorem
  for G being nilpotent Group
  for H being Subgroup of G
  holds H is proper Subgroup of Normalizer H;
\end{mizar}

\begin{theorem}
A finite group is nilpotent if and only if it is the [internal] direct
product of its Sylow subgroups.
\end{theorem}

\begin{mizar}
definition
  let G be Group;
  func Sylow_subgroups G -> Subset of Subgroups G means
  for P being strict Subgroup of G
  holds P in it
        iff ex p being Prime
            st P is_Sylow_p-subgroup_of_prime p;
  existence;
  uniqueness;
end;

theorem ThProdOfNilpotents:
  for G being Group
  for I being set
  for F being normal Subgroup-Family of G,I
  st (for i being Element of I holds F.i is nilpotent)
   & G is_internal_product_of F
  holds G is nilpotent;

theorem ThUniqueSylowSubgroups:
  for G being finite Group
  st (for p being Prime st p divides card G
      ex P being strict Subgroup of G
      st the_sylow_p-subgroups_of_prime (p,G) = {P})
  holds G is_internal_product_of Sylow_subgroups G;
  
theorem
  for G being finite Group
  holds G is_internal_product_of Sylow_subgroups G
        iff G is nilpotent
proof
  thus C1: G is_internal_product_of Sylow_subgroups G
           implies G is nilpotent
  proof
    assume A1: G is_internal_product_of Sylow_subgroups G;
    reconsider I=Sylow_subgroups of G as set,
    F=id (Sylow_subgroups of G) as Subgroup-Family of I,G;
    A2: G is_internal_product_of F by A1;
    for i being Element of I holds F.i is nilpotent;
    hence thesis by A2, ThProdOfNilpotents;
  end;
  thus G is nilpotent
       implies G is_internal_product_of Sylow_subgroups G
  proof
    assume A1: G is nilpotent;
    for p being Prime st p divides card G
    for P being Subgroup of G
    st P is_Sylow_p-subgroup_of_prime p
    holds the_sylow_p-subgroups_of_prime (p,G) = {P};
    then G is_internal_product_of Sylow_subgroups G
    by ThUniqueSylowSubgroups;
    hence thesis by C1;
  end;
end;
\end{mizar}

\subsection{More about ``Proof by Minimal Counter-example''}
We want to be able to perform proofs by minimal counter-example
easily. Towards that end, we need to construct the infrastructure to
facilitate such proofs. We need something like

\begin{mizar}
definition
  func finite_groups -> RelStr means
  (for G being object
   holds G is strict finite Group
         iff G in the carrier of it)
  & (for x,y being strict finite Group
     holds [x,y] in the InternalRel of it
           iff card x < card y);
  existence;
  uniqueness;
end;

registration
  cluster finite_groups -> non empty;
  correctness; :: the trivial finite Group in finite_groups
end;
\end{mizar}

We need to prove that ``\lstinline{finite_groups}'' is well-founded,
then we can construct a scheme which basically amounts to a wrapper
around the well-founded induction scheme.

However, it is unclear to me whether we should restrict ourselves to
\emph{strict} finite groups, or whether we should permit \emph{all}
finite groups. I think all finite groups would no longer be
well-founded, which could cause problems: our induction scheme would
only work for proving properties about \emph{strict} finite groups.
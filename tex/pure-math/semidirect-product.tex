\section{Semidirect Product}

This is the proof sketch for formalizing \S10 of
Aschbacher~\cite{aschbacher2000finite}.

\begin{definition}
Let $G$ be a group, let $N\normalSubgroup G$, let $H\subgroup G$.
We call $H$ a \define{Complement to} $N$ in $G$ if $G=HN$ and $H\cap N=\trivialSubgroup$.
\end{definition}

\begin{mizar}
definition
  let G be Group;
  let H, K be Subgroup of G;
  attr H is_complement_to K means
  (H is normal or K is normal)
  & H "\/" K = the multMagma of G
  & H /\ K = (1).G;
  symmetry
  proof
    thus for A,B being Subgroup of G
    st ((A is normal or B is normal)
        & (A "\/" B = the multMagma of G)
        & (A "/\" B = (1).G))
    holds ((B is normal or A is normal)
        & (B "\/" A = the multMagma of G)
        & (B "/\" A = (1).G));
  end;
end;  
\end{mizar}

\begin{definition}
Let $N$, $H$ be groups, let $\varphi\colon H\to\Aut(N)$ be a group morphism.
We define the \define{Semidirect Product} of $N$ by $H$ along $\varphi$
as the group $S(H,N,\varphi)=N\rtimes_{\varphi} H$ consisting of ordered pairs $(h,n)\in H\times N$
with binary operator
\begin{equation}
(h_{1},n_{1})(h_{2},n_{2}) = (h_{1}h_{2}, \varphi_{h_{2}}(n_{1})n_{2})
\end{equation}
where $\varphi_{h_{2}}\in\Aut(N)$.
\end{definition}

\begin{def-remark}
This is a rather massive definition. It's best to attack this
piece-meal: we define the semidirect product to be a ``\lstinline{multMagma}'',
then register it is associative, unital, and group-like (thus proving it
is a Group).
\end{def-remark}

\begin{def-remark}
We choose to define the semidirect product as a \emph{strict} magma,
because we want the definition to reduce to the direct product when
$\varphi$ is the trivial morphism (and Definition~\mml[def 2]{group7}
defines the product group as a \emph{strict} magma).
\end{def-remark}

\begin{mizar}
definition
  let H, N be Group;
  let phi be Homomorphism of H,AutGroup(N);
  func semi_product(H,N,phi) -> strict multMagma means
  the carrier of it = the carrier of product <* H, N *>
  & for h1,h2 being Element of H
    for n1,n2 being Element of N
    holds (the multF of it).(<*h1,n1*>, <*h2,n2*>)
    = <*(h1*h2), ((phi.h2).n1)*n2 *>;
  existence;
  uniqueness;
end;
\end{mizar}

\begin{registration}
The semidirect product is non empty.
\end{registration}

\begin{remark}
It's rather urgent to register this fact, otherwise all future proofs
concerning the semidirect product will require proving it.
\end{remark}

\begin{mizar}
registration
  let H, N be Group;
  let phi be Homomorphism of H,AutGroup(N);
  cluster semi_product(H,N,phi) -> non empty;
  coherence
  proof
    <* 1_H, 1_N *> in semi_product(H, N, phi);
    hence semi_product(H,N,phi) is non empty;
  end;
end;
\end{mizar}

\begin{definition}
Let $H$, $N$ be groups, let $\varphi\colon H\to\Aut(N)$ be a group morphism.
We use the notation $\langle[h,n]\rangle$ for an element of the
semidirect product $S(H,N,\varphi)=N\rtimes_{\varphi} H$.
\end{definition}

\begin{def-remark}
This definition is just to avoid ambiguity when referring to elements of
$S(H,N,\varphi)$ as opposed to $H\times N$ (which will matter when
working with, e.g., the multiplication of elements).
\end{def-remark}

\begin{mizar}
definition
  let H, N be Group;
  let phi be Homomorphism of H,AutGroup(N);
  let h be Element of H;
  let n be Element of N;
  func <[* h, n *]> -> Element of (semi_product(H,N,phi))
  equals <* h, n *>;
  coherence;
  :: goal: <* h, n *> is Element of semi_product(H, N, phi)
end;
\end{mizar}

\begin{theorem}
For any element of the semidirect product $x\in S(H, N, \varphi)$,
there exists elements $h\in H$ and $n\in N$ such that $x=\langle[h,n]\rangle$.
\end{theorem}

\begin{thm-remark}
See also Theorem (8.8) in Suzuki~\cite{suzuki1982group}, chapter 1.
There Suzuki proves if $G=HN$ and $H\cap N=\trivialSubgroup$, then any
$g\in G$ may be written uniquely as $g=hn$ for some $h\in H$ and $n\in N$.
\end{thm-remark}

\begin{mizar}
theorem
  for x being Element of semi_product(H, N, phi)
  ex h being Element of H, n being Element of N
  st x = <[* h, n *]>;
\end{mizar}

\begin{registration}
The semidirect product is associative.
\end{registration}

\begin{mizar}
registration
  let H, N be Group;
  let phi be Homomorphism of H,AutGroup(N);
  cluster semi_product(H,N,phi) -> non empty associative;
  coherence
  proof
    for x, y, z being Element of semi_product(H,N,phi)
    holds (x * y) * z = x * (y * z);
    hence semi_product(H,N,phi) is associative by GROUP_1:def 3;
  end;
end;
\end{mizar}

\begin{registration}
The semidirect product is group-like.
\end{registration}

\begin{remark}
This will complete the proof that the semidirect product is a group, and
Mizar will automatically think of the semidirect product of groups
\emph{as a group} from now on.
\end{remark}

\begin{mizar}
registration
  let H, N be Group;
  let phi be Homomorphism of H,AutGroup(N);
  cluster semi_product(H,N,phi) -> non empty Group-like associative;
  coherence
  proof
    ex e being Element of semi_product(H,N,phi) st
    for x being Element of semi_product(H,N,phi)
    holds (x * e = x & e * x = x
           & ex y being Element of semi_product(H,N,phi)
             st x * y = e & y * x = e);
    hence semi_product(H,N,phi) is Group-like by GROUP_1:def 2;
  end;
end;
\end{mizar}

\begin{theorem}
When $\varphi$ is the trivial homomorphism, then the semidirect product
coincides with the direct product.
\end{theorem}

\begin{mizar}
theorem
  semi_product(H, N, 1:(H, AutGroup(N))) = product <* H, N *>;
\end{mizar}

\begin{theorem}[{\cite[(10.1.2)]{aschbacher2000finite}}]
The maps $\sigma_{H}\colon H\to S(H, N,\varphi)$ 
sending $\sigma_{H}(h)=\langle[h,1]\rangle$, 
and $\sigma_{N}\colon N\to S(H, N, \varphi)$
sending $\sigma_{N}(n)=\langle[1,n]\rangle$,
are both injective group morphisms.
\end{theorem}

\begin{theorem}[{\cite[(10.1.3)]{aschbacher2000finite}}]
The image $\sigma_{N}(N)\normalSubgroup S(H, N, \varphi)$ is a normal
subgroup of the semidirect product.
\end{theorem}

\begin{theorem}[{\cite[(10.1.3)]{aschbacher2000finite}}]
The image $\sigma_{N}(N)$ is a complement of $\sigma_{H}(H)$ in the
semidirect product $\normalSubgroup S(H, N, \varphi)$.
\end{theorem}

\begin{theorem}[{\cite[(10.1.4)]{aschbacher2000finite}}]
For any $n\in N$ and $h\in H$ we have
$\langle[1, n]\rangle^{\langle[h,1]\rangle} = \langle[1, \varphi_{h}(n)]\rangle$ 
in the semidirect product.
\end{theorem}

\begin{theorem}[{\cite[(10.2)]{aschbacher2000finite}}]
Let $G$ be a group, let $N\normalSubgroup G$, and let $H\subgroup G$ be a
complement to $N$ in $G$.
Let $\alpha\colon H\to\Aut(N)$ be the conjugation map (so
$\alpha(h)\colon n\mapsto n^{h}$).
Let $\beta\colon S(H, N, \alpha)\to G$ be the group morphism defined by
$\beta(h,n)=hn$. Then $\beta$ is an isomorphism.
\end{theorem}

\begin{thm-remark}
See also Theorem (8.7) in Suzuki~\cite{suzuki1982group}, chapter 1.
\end{thm-remark}

\begin{theorem}[{\cite[(10.3)]{aschbacher2000finite}}]
Let $S_{i}=S(H_{i}, N_{i}, \varphi_{i})$ for $i=1,2$ be semidirect products.
Then there exists an isomorphism $\psi\colon S_{1}\to S_{2}$ with
$(\psi\circ\sigma_{H_{1}})(H_{1})=\sigma_{H_{2}}(H_{2})$ and
$(\psi\circ\sigma_{N_{1}})(N_{1})=\sigma_{N_{2}}(N_{2})$ if and only if
$\varphi_{1}$ is quasi-equivalent to $\varphi_{2}$ (i.e., if there
exists an isomorphism $\beta\colon H_{2}\to H_{1}$ and an isomorphism
$\alpha\colon N_{1}\to N_{2}$ such that the ``obvious'' induced isomorphism
$\alpha^{*}\colon\Aut(N_{1})\to\Aut(N_{2})$
satisfies $\varphi_{2}=\alpha^{*}\circ\varphi_{1}\circ\beta$).
\end{theorem}

\begin{thm-remark}
Aschbacher uses a highly idiosyncratic notion of ``quasiequivalence'',
which appears only in his book on finite group theory.
Basically, if we have two representations $\rho_{i}\colon G\to\Aut(X_{i})$
for $i=1,2$, if there exists an isomorphism $\alpha\colon X_{1}\to X_{2}$,
then there is an isomorphism $\alpha^{*}\colon\Aut(X_{1})\to\Aut(X_{2})$
such that $\rho_{2}=\alpha^{*}\circ\rho_{1}$. 

More generally, in any category, if $\alpha\colon A\to B$ is an
isomorphism, we can define $\alpha^{*}\colon\End(A)\to\End(B)$ by
$\alpha^{*}(\beta)=\alpha\circ\beta\circ\alpha^{-1}$.

But when there are two representations of two groups
$\rho_{i}\colon G_{i}\to\Aut(X_{i})$ for $i=1,2$, then they are called
\define{Quasiequivalent} if there exists an isomorphism of groups
$\beta\colon G_{2}\to G_{1}$ and an isomorphism $\alpha\colon X_{1}\to X_{2}$
such that $\rho_{2}=\alpha^{*}\circ\rho_{1}\circ\beta$.
This makes sense since $\rho_{1}\circ\beta\colon G_{2}\to\Aut(X_{1})$,
so for any $g\in G_{2}$ we would have
$(\alpha^{*}\circ\rho_{1}\circ\beta)(g)=\alpha\circ\bigl((\rho_{1}\circ\beta)(g)\bigr)\circ\alpha^{-1}$.
\end{thm-remark}


\begin{theorem}[{\cite[(I.8.9)]{suzuki1982group}}]
Let $G=S(H,N,\varphi)$ be the semidirect product of $H$ and $N$ along $\varphi$.
Let $\widetilde{H}=\sigma_{H}(H)\subgroup G$ and
$\widetilde{N}=\sigma_{N}(N)\normalSubgroup G$.
Then:
\begin{enumerate}
\item $N_{G}(\widetilde{H})\cap\widetilde{N}$ commutes elementwise with $\widetilde{H}$,
\item $N_{G}(\widetilde{H})\cap\widetilde{N} = C_{\widetilde{N}}(\widetilde{H})$,
and
\item $N_{G}(\widetilde{H}) = \widetilde{H}C_{\widetilde{N}}(\widetilde{H})$.
\end{enumerate}
\end{theorem}

\begin{theorem}[{G\"aschutz~\cite[(10.4)]{aschbacher2000finite}}]
Let $p$ be a prime, $V$ a commutative normal $p$-subgroup of a finite
group $G$, let $P\in\SylowSubgroups{p}{G}$. 
Then $G$ splits over $V$ if and only if $P$ splits over $V$.
\end{theorem}

\begin{thm-remark}
See also Higman~\cite{higman1954remarks} and Rose~\cite{rose1966} for
more about this particular theorem. 
\end{thm-remark}
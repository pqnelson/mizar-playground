\section{Finite Simple Groups}

\begin{proposition}\label{prop:pure-math:cfsg}
There are four types of finite simple groups:
\begin{enumerate}
\item the finite cyclic groups of prime order;
\item the alternating groups $A_{n}$ for $n\geq 5$;
\item the finite groups of Lie type; and
\item the Sporadic groups (i.e., ``the 26 whacky groups'')
\end{enumerate}
Moreover, any finite simple group must be one of these four types, with
the only repetitions being:
\begin{subequations}
  \begin{align}
    \PSL_{2}(4)&\iso A_{5}\\
    \PSL_{2}(5)&\iso A_{5}\\
    \PSL_{2}(7)&\iso\PSL_{3}(2)\\
    \PSL_{2}(9)&\iso A_{6}
  \end{align}
\end{subequations}\vskip-2\belowdisplayskip\vskip-2\abovedisplayskip
  \begin{align}
    \PSL_{4}(2)&\iso A_{8}\\
    \PSU_{4}(2)&\iso\PSp_{4}(3).
  \end{align}
\end{proposition}

\begin{definition}
We say a group $G$ is a \define{$\mathcal{K}$-Group} if it is one of the
simple groups as listed in Proposition~\ref{prop:pure-math:cfsg}.
\end{definition}

\subsection{Sporadic Groups}
These are the simple groups which are not finite groups of Lie type,
cyclic, or alternating. Their construction is non-uniform, since they
are symmetries of random exotic mathematical objects.

\begin{definition}
  We call a finite group $G$ \define{Sporadic} if
  \begin{enumerate}
  \item $G$ is a simple group,
  \item $G$ is not cyclic,
  \item $G$ is not of Lie type,
  \item $G$ is not an alternating group.
  \end{enumerate}
\end{definition}

\begin{def-remark}[References]
For actually \emph{constructing} sporadic groups,
Aschbacher~\cite{aschbacher1994sporadic} provides a concise series of
constructions, Wilson~\cite{wilson2009finite} provides elegant
constructions. Griess~\cite{griess1998twelve} reviews 12 sporadic
groups.
\end{def-remark}

\begin{def-remark}[Organization of Sporadics]
There are three ``generations'' used to organize sporadic groups plus
the six ``pariahs'':
\begin{enumerate}
\item First Generation (the Mathieu groups): $\Mathieu{11}$,
  $\Mathieu{12}$, $\Mathieu{22}$, $\Mathieu{23}$, $\Mathieu{24}$;
\item Second Generation (the Leech Lattice groups): the Conway groups
  $\Conway{1}$, $\Conway{2}$, $\Conway{3}$; the McLaughlin group
  $\McLaughlin$; the Higman--Sims group $\HigmanSims$; and the Janko
  group $\Janko{2}$;
\item Third Generation (other Subgroups of the Monster $\Monster$): the most
  relevant one for quasithin classification being the Held group
  $\Held$;
\item Pariahs: three Janko groups $\Janko{1}$, $\Janko{3}$, $\Janko{4}$;
  the O'Nan group $\ONan$; the Rudvalis group $\Rudvalis$; and Lyons
  group $\Lyons$.
\end{enumerate}
They are very exciting and beautiful, but I am not going to investigate
\emph{all} of them further.
\end{def-remark}

\begin{def-remark}[Quasithin Sporadics]
According to Aschbacher and Smith~\cite{aschbacher2004classification1,aschbacher2004classification2}, the
sporadic groups which are quasithin include:
$\Mathieu{11}$, $\Mathieu{12}$, $\Mathieu{22}$,
$\Mathieu{23}$, $\Mathieu{24}$, $\Janko{2}$, $\Janko{3}$, $\Janko{4}$,
$\HigmanSims$, $\Held$, $\Rudvalis$. Clustered by generation:
\begin{enumerate}
\item all five Mathieu groups ($\Mathieu{11}$, $\Mathieu{12}$, $\Mathieu{22}$, $\Mathieu{23}$,
  $\Mathieu{24}$);
\item a couple Leech Lattice groups ($\HigmanSims$, $\Janko{2}$);
\item from the Monster-related sporadics, only the Held group $\Held$; and
\item $\Janko{3}$, $\Janko{4}$, $\Rudvalis$.
\end{enumerate}
Also note: unlike the Mathieu groups, the Conway groups, or the Fischer
groups, the Janko groups do not form a series or ``family'' --- they are
just a few ``random'' groups constructed by the same mathematician.
\end{def-remark}

\begin{proposition}
Finite non-Abelian simple quasithin $\mathcal{K}$-groups of even
characteristic consist of:
\begin{enumerate}
\item (Generic case) Groups of Lie type of characteristic 2 and Lie rank at most 2, but
  $U_{5}(q)$ only for $q=4$
\item (Certain groups of rank 3 or 4) $L_{4}(2)$, $L_{5}(2)$, $\Sp_{6}(2)$
\item (Alternating groups) $A_{5}$, $A_{6}$, $A_{8}$, $A_{9}$
\item (Lie type of odd characteristic) $L_{2}(p)$ when $p$ is a Mersenne
  or Fermat prime, $L_{3}^{\epsilon}(3)$, $L_{4}^{\epsilon}(3)$, $G_{2}(3)$
\item (Sporadic) $\Mathieu{11}$, $\Mathieu{12}$, $\Mathieu{22}$,
$\Mathieu{23}$, $\Mathieu{24}$, $\Janko{2}$, $\Janko{3}$, $\Janko{4}$,
$\HigmanSims$, $\Held$, $\Rudvalis$
\end{enumerate}
\end{proposition}

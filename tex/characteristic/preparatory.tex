\nwfilename{nw/characteristic/preparatory.nw}\nwbegindocs{0}\subsection{Preparatory results}% ===> this file was generated automatically by noweave --- better not edit it

\M
We begin by registering the identity function as being surjective and
bijective. This should have been done in {\Tt{}FUNCT{\_}2\nwendquote}, but hey, what can
you do?

\nwenddocs{}\nwbegincode{1}\sublabel{NW18nMBU-LuhGu-1}\nwmargintag{{\nwtagstyle{}\subpageref{NW18nMBU-LuhGu-1}}}\moddef{Helper lemmas and registrations for \code{}tmp.miz\edoc{}~{\nwtagstyle{}\subpageref{NW18nMBU-LuhGu-1}}}\endmoddef\nwstartdeflinemarkup\nwprevnextdefs{\relax}{NW18nMBU-LuhGu-2}\nwenddeflinemarkup
begin :: preparatory work
reserve X for set;

registration
  let X;
  cluster id X -> onto;
  coherence;
end;

registration
  let X;
  cluster id X -> bijective;
  coherence;
end;

\nwalsodefined{\\{NW18nMBU-LuhGu-2}\\{NW18nMBU-LuhGu-3}}\nwnotused{Helper lemmas and registrations for [[tmp.miz]]}\nwendcode{}\nwbegindocs{2}\nwdocspar

\M
We also need to prove that, if we have a subgroup $H\subgroup G$ and
$h\in G$ is an arbitrary element of the underlying set, then $h\in H$
implies the typing judgement ``{\Tt{}h\ is\ Element\ of\ H\nwendquote}''.

\nwenddocs{}\nwbegincode{3}\sublabel{NW18nMBU-LuhGu-2}\nwmargintag{{\nwtagstyle{}\subpageref{NW18nMBU-LuhGu-2}}}\moddef{Helper lemmas and registrations for \code{}tmp.miz\edoc{}~{\nwtagstyle{}\subpageref{NW18nMBU-LuhGu-1}}}\plusendmoddef\nwstartdeflinemarkup\nwprevnextdefs{NW18nMBU-LuhGu-1}{NW18nMBU-LuhGu-3}\nwenddeflinemarkup
LmSubgroupElt:
for G being Group
for H being Subgroup of G
for h being Element of G st h in H holds h is Element of H
proof
  let G be Group,
      H be Subgroup of G,
      h be Element of G;
  assume h in H;
  then h is Element of the carrier of H by STRUCT_0:def 5;
  then h is Element of H;
  hence thesis;
end;

\nwendcode{}\nwbegindocs{4}\nwdocspar

\N{Restriction of Group Morphisms to Subgroups}
Mizar proves that, if $f\colon X\to Y$ is a set theoretic function and
$A\subset X$ is an arbitrary subset, then $\forall a\in X$ we have
$a\in A\implies f|_{A}(a)=f(a)$. But this doesn't generalize to
morphisms, sadly, because a group is like a chocolate-covered set.

So we just prove for any pair of groups $G_{1}$ and $G_{2}$,
for any subgroup $H\subgroup G_{1}$,
for any group morphism $f\colon G_{1}\to G_{2}$, and for arbitrary $h\in G_{1}$,
we have $h\in H\implies f|_{H}(h) = f(h)$. The reasoning is that we can
always look at the set-theoretic function $U(f)$ underlying $f$, then
piggie-back off earlier results establishing the desired claim
(Theorem~49 of {\Tt{}FUNCT{\_}1\nwendquote}, to be precise).

\nwenddocs{}\nwbegincode{5}\sublabel{NW18nMBU-LuhGu-3}\nwmargintag{{\nwtagstyle{}\subpageref{NW18nMBU-LuhGu-3}}}\moddef{Helper lemmas and registrations for \code{}tmp.miz\edoc{}~{\nwtagstyle{}\subpageref{NW18nMBU-LuhGu-1}}}\plusendmoddef\nwstartdeflinemarkup\nwprevnextdefs{NW18nMBU-LuhGu-2}{\relax}\nwenddeflinemarkup
Lm3: for G1,G2 being Group
for H being Subgroup of G1
for f being Homomorphism of G1,G2
for h being Element of G1
st h in H
holds (f|H).h = f.h
proof
  let G1,G2 be Group;
  let H be Subgroup of G1;
  let f be Homomorphism of G1,G2;
  let h be Element of G1;
  assume h in H;
  then h is Element of H by LmSubgroupElt;
  then (f|(the carrier of H)).h = f.h by FUNCT_1:49;
  then (f|H).h = f.h by GRSOLV_1:def 2;
  hence thesis;
end;

\nwendcode{}\nwbegindocs{6}\nwdocspar

\M
Let $H_{1}\subgroup G$ and $H_{2}\subgroup G$.
If $H_{1}\subgroup H_{2}$ and $H_{2}\subgroup H_{1}$, then
$H_{1}=H_{2}$. Pretty simple, right? \dots I'm not sure I'll need this,
but here it is if I do.

\nwenddocs{}\nwbegincode{7}\sublabel{NW18nMBU-G9U50-1}\nwmargintag{{\nwtagstyle{}\subpageref{NW18nMBU-G9U50-1}}}\moddef{Sandwich Subgroup Equality Theorem~{\nwtagstyle{}\subpageref{NW18nMBU-G9U50-1}}}\endmoddef\nwstartdeflinemarkup\nwenddeflinemarkup
LmSandwich: for H1 being strict Subgroup of G
for H2 being strict Subgroup of G
st (H1 is Subgroup of H2) & (H2 is Subgroup of H1)
holds H1 = H2
proof
  let H1 be strict Subgroup of G;
  let H2 be strict Subgroup of G;
  assume H1 is Subgroup of H2 & H2 is Subgroup of H1;
  then the multMagma of H1 = the multMagma of H2 by GROUP_2:55;
  hence thesis;
end;

\nwnotused{Sandwich Subgroup Equality Theorem}\nwendcode{}\nwbegindocs{8}\nwdocspar

\begin{remark}[On ``strict'' mathematical gadgets]\index{strict@\texttt{strict}|textbf}
Mizar has a notion of a ``{\Tt{}strict\nwendquote}'' attribute. What does it mean?
Well, sometimes a mathematical gadget is ``composite''; like a vector
space over a field is an Abelian group with vector addition, but it also
has other stuff (like scalar multiplication). This ``other stuff''
prevents it from being a ``bald-faced Abelian group''. Just as black
holes have no hair, strict groups have no hair: it's just a
{\Tt{}multMagma\nwendquote} satisfying the group properties (in Mizar, at least).
\end{remark}

\M Let $H\subgroup G$ be such that $\forall a\in G, a^{-1}Ha=H$.
Then $H\normalSubgroup G$ is a normal subgroup. The current theorems in
{\Tt{}GROUP{\_}3\nwendquote} require $H$ to be a \emph{strict} subgroup, but this
strictness condition is not necessary.

\nwenddocs{}\nwbegincode{9}\sublabel{NW18nMBU-2karRI-1}\nwmargintag{{\nwtagstyle{}\subpageref{NW18nMBU-2karRI-1}}}\moddef{Subgroups invariant under conjugation are normal~{\nwtagstyle{}\subpageref{NW18nMBU-2karRI-1}}}\endmoddef\nwstartdeflinemarkup\nwenddeflinemarkup
LmNormalSubgroup:
for G being Group
for H being Subgroup of G
st (for a being Element of G holds H |^ a = H)
holds H is normal Subgroup of G
proof
  let G be Group;
  let H be Subgroup of G;
  assume A0: for a being Element of G holds H |^ a = H;
  then for a being Element of G holds the carrier of H |^ a = the carrier of H;
  then for a being Element of G holds the multMagma of H = H |^ a by GROUP_2:59;
  hence H is normal Subgroup of G by GROUP_3:def 13;
end;

\nwnotused{Subgroups invariant under conjugation are normal}\nwendcode{}\nwbegindocs{10}\nwdocspar
\nwenddocs{}

\nwixlogsorted{c}{{Helper lemmas and registrations for \code{}tmp.miz\edoc{}}{NW18nMBU-LuhGu-1}{\nwixd{NW18nMBU-LuhGu-1}\nwixd{NW18nMBU-LuhGu-2}\nwixd{NW18nMBU-LuhGu-3}}}%
\nwixlogsorted{c}{{Sandwich Subgroup Equality Theorem}{NW18nMBU-G9U50-1}{\nwixd{NW18nMBU-G9U50-1}}}%
\nwixlogsorted{c}{{Subgroups invariant under conjugation are normal}{NW18nMBU-2karRI-1}{\nwixd{NW18nMBU-2karRI-1}}}%


\nwfilename{nw/characteristic/automorphism.nw}\nwbegindocs{0}\subsection{Automorphisms}% ===> this file was generated automatically by noweave --- better not edit it

\M Now, we have a section for defining inner and outer group
automorphisms. A \emph{group automorphism} is just a bijective endomorphism on
a group, and an \emph{endomorphism} is a group morphism whose codomain
is its domain.

Remember (\S\ref{par:characteristic:vocabularies}), although we are
defining new terms {\Tt{}Endomorphism\nwendquote} and {\Tt{}Automorphism\nwendquote}, we do not need
to add them to our {\Tt{}DICT/TMP.VOC\nwendquote} file. Why not? Because the tokens
are already included from {\Tt{}MOD{\_}4\nwendquote}.

\nwenddocs{}\nwbegincode{1}\sublabel{NW1Nnu7l-32pmQO-1}\nwmargintag{{\nwtagstyle{}\subpageref{NW1Nnu7l-32pmQO-1}}}\moddef{Inner and outer automorphisms~{\nwtagstyle{}\subpageref{NW1Nnu7l-32pmQO-1}}}\endmoddef\nwstartdeflinemarkup\nwenddeflinemarkup
begin

\LA{}Reserve symbols for inner and outer automorphisms~{\nwtagstyle{}\subpageref{NW1Nnu7l-1LL7AC-1}}\RA{}
\LA{}Define \code{}Endomorphism\edoc{}~{\nwtagstyle{}\subpageref{NW1Nnu7l-17cKlQ-1}}\RA{}
\LA{}Register \code{}bijective\edoc{} for \code{}Endoomorphism\edoc{}~{\nwtagstyle{}\subpageref{NW1Nnu7l-1BJmAm-1}}\RA{}
\LA{}Define \code{}Automorphism\edoc{}~{\nwtagstyle{}\subpageref{NW1Nnu7l-4WeeL5-1}}\RA{}
\LA{}Endomorphisms preserve the trivial subgroup~{\nwtagstyle{}\subpageref{NW1Nnu7l-26jvGu-1}}\RA{}
\LA{}Automorphisms map trivial subgroups to themselves~{\nwtagstyle{}\subpageref{NW1Nnu7l-3J82fO-1}}\RA{}
\LA{}Define $\Id_{G}$~{\nwtagstyle{}\subpageref{NW1Nnu7l-27i33n-1}}\RA{}
\LA{}\code{}Id{\_}G\edoc{} is the same as \code{}id\ the\ carrier\ of\ G\edoc{}~{\nwtagstyle{}\subpageref{NW1Nnu7l-4KG2Xs-1}}\RA{}
\LA{}Register \code{}Id{\_}G\edoc{} is bijective~{\nwtagstyle{}\subpageref{NW1Nnu7l-37g7e5-1}}\RA{}
\LA{}Define \code{}inner\edoc{} for Automorphism~{\nwtagstyle{}\subpageref{NW1Nnu7l-13QmRM-1}}\RA{}
\LA{}\code{}Id{\_}G\edoc{} is effectively inner~{\nwtagstyle{}\subpageref{NW1Nnu7l-3VEX6S-1}}\RA{}
\LA{}Register \code{}inner\edoc{} for \code{}Automorphism\edoc{}~{\nwtagstyle{}\subpageref{NW1Nnu7l-4IjmGJ-1}}\RA{}
\LA{}Relate \code{}Automorphism\ of\ G\edoc{} to elements of \code{}Aut\ G\edoc{}~{\nwtagstyle{}\subpageref{NW1Nnu7l-3c36xj-1}}\RA{}
\LA{}Theorem: $f$ in \code{}InnAut\ G\edoc{} iff $f$ is \code{}inner\ Automorphism\ of\ G\edoc{}~{\nwtagstyle{}\subpageref{NW1Nnu7l-2gUxot-1}}\RA{}
\LA{}Theorem: inner automorphism acting on subgroup is conjugate of argument~{\nwtagstyle{}\subpageref{NW1Nnu7l-2zspfg-1}}\RA{}
\LA{}Theorem: conjugation of given element is an inner automorphism~{\nwtagstyle{}\subpageref{NW1Nnu7l-4NcpAX-1}}\RA{}

\nwnotused{Inner and outer automorphisms}\nwendcode{}\nwbegindocs{2}\nwdocspar

\M Note we just used the symbol {\Tt{}G\nwendquote} for a generic group, so we add to
the reserved symbols this association.

\nwenddocs{}\nwbegincode{3}\sublabel{NW1Nnu7l-1LL7AC-1}\nwmargintag{{\nwtagstyle{}\subpageref{NW1Nnu7l-1LL7AC-1}}}\moddef{Reserve symbols for inner and outer automorphisms~{\nwtagstyle{}\subpageref{NW1Nnu7l-1LL7AC-1}}}\endmoddef\nwstartdeflinemarkup\nwusesondefline{\\{NW1Nnu7l-32pmQO-1}}\nwprevnextdefs{\relax}{NW1Nnu7l-1LL7AC-2}\nwenddeflinemarkup
reserve G for Group;

\nwalsodefined{\\{NW1Nnu7l-1LL7AC-2}}\nwused{\\{NW1Nnu7l-32pmQO-1}}\nwendcode{}\nwbegindocs{4}\nwdocspar

\M A group endomorphism of $G$ is just a homomorphism $f\colon G\to G$.

\nwenddocs{}\nwbegincode{5}\sublabel{NW1Nnu7l-17cKlQ-1}\nwmargintag{{\nwtagstyle{}\subpageref{NW1Nnu7l-17cKlQ-1}}}\moddef{Define \code{}Endomorphism\edoc{}~{\nwtagstyle{}\subpageref{NW1Nnu7l-17cKlQ-1}}}\endmoddef\nwstartdeflinemarkup\nwusesondefline{\\{NW1Nnu7l-32pmQO-1}}\nwenddeflinemarkup
definition :: Def1
  let G;
  mode Endomorphism of G is Homomorphism of G,G;
end;

\nwused{\\{NW1Nnu7l-32pmQO-1}}\nwendcode{}\nwbegindocs{6}\nwdocspar

\M\label{par:characteristic:register-bijective-for-endomorphism}
We begin by registering the attribute {\Tt{}bijective\nwendquote} for group
endomorphisms. This will effectively create a subtype of
{\Tt{}Endomorphism\ of\ G\nwendquote}, the aptly named {\Tt{}bijective\ Endomorphism\ of\ G\nwendquote}.
Most of our work has been done in article {\Tt{}GROUP{\_}6\nwendquote} by
Theorem~38 (which effectively states the function on the underlying set
{\Tt{}id\ (the\ carrier\ of\ G)\nwendquote} is a multiplicative function and so nearly a
group morphism that we can reconsider it as an {\Tt{}Endomorphism\nwendquote}), and
the fact that {\Tt{}id\ X\nwendquote} is bijective.

\nwenddocs{}\nwbegincode{7}\sublabel{NW1Nnu7l-1BJmAm-1}\nwmargintag{{\nwtagstyle{}\subpageref{NW1Nnu7l-1BJmAm-1}}}\moddef{Register \code{}bijective\edoc{} for \code{}Endoomorphism\edoc{}~{\nwtagstyle{}\subpageref{NW1Nnu7l-1BJmAm-1}}}\endmoddef\nwstartdeflinemarkup\nwusesondefline{\\{NW1Nnu7l-32pmQO-1}}\nwenddeflinemarkup
registration
  let G;
  cluster bijective for Homomorphism of G,G;
  existence
  proof
    reconsider i = id the carrier of G as Homomorphism of G,G by GROUP_6:38;
    i is bijective;
    hence thesis;
  end;
end;

\nwused{\\{NW1Nnu7l-32pmQO-1}}\nwendcode{}\nwbegindocs{8}\nwdocspar

\N{Definition (Automorphism)}
Now we have an {\Tt{}Automorphism\ of\ G\nwendquote} be just a bijective endomorphism.

\nwenddocs{}\nwbegincode{9}\sublabel{NW1Nnu7l-4WeeL5-1}\nwmargintag{{\nwtagstyle{}\subpageref{NW1Nnu7l-4WeeL5-1}}}\moddef{Define \code{}Automorphism\edoc{}~{\nwtagstyle{}\subpageref{NW1Nnu7l-4WeeL5-1}}}\endmoddef\nwstartdeflinemarkup\nwusesondefline{\\{NW1Nnu7l-32pmQO-1}}\nwenddeflinemarkup
definition :: Def2
  let G;
  mode Automorphism of G is bijective Endomorphism of G;
end;

\nwused{\\{NW1Nnu7l-32pmQO-1}}\nwendcode{}\nwbegindocs{10}\nwdocspar

\M For any group $G$ and endomorphism $f\in\End(G)$, we have the trivial
subgroup $\trivialSubgroup\subgroup G$ be preserved under $f$; i.e.,
$f(\trivialSubgroup)=\trivialSubgroup$.

\nwenddocs{}\nwbegincode{11}\sublabel{NW1Nnu7l-26jvGu-1}\nwmargintag{{\nwtagstyle{}\subpageref{NW1Nnu7l-26jvGu-1}}}\moddef{Endomorphisms preserve the trivial subgroup~{\nwtagstyle{}\subpageref{NW1Nnu7l-26jvGu-1}}}\endmoddef\nwstartdeflinemarkup\nwusesondefline{\\{NW1Nnu7l-32pmQO-1}}\nwenddeflinemarkup
theorem Th1:
  Image (f|(1).G) = (1).G
proof
  Image(f|(1).G) = f .: ((1).G) by GRSOLV_1:def 3
  .= (1).G by GRSOLV_1:11;
  hence thesis;
end;

\nwused{\\{NW1Nnu7l-32pmQO-1}}\nwendcode{}\nwbegindocs{12}\nwdocspar

\M Now we need to tell Mizar that $f$ is an endomorphism of $G$.

\nwenddocs{}\nwbegincode{13}\sublabel{NW1Nnu7l-1LL7AC-2}\nwmargintag{{\nwtagstyle{}\subpageref{NW1Nnu7l-1LL7AC-2}}}\moddef{Reserve symbols for inner and outer automorphisms~{\nwtagstyle{}\subpageref{NW1Nnu7l-1LL7AC-1}}}\plusendmoddef\nwstartdeflinemarkup\nwusesondefline{\\{NW1Nnu7l-32pmQO-1}}\nwprevnextdefs{NW1Nnu7l-1LL7AC-1}{\relax}\nwenddeflinemarkup
reserve f for Endomorphism of G;

\nwused{\\{NW1Nnu7l-32pmQO-1}}\nwendcode{}\nwbegindocs{14}\nwdocspar

\M If $\phi$ is an automorphism of a group $G$, then
the image of the trivial subgroup under $\phi$ is a subgroup of itself
$\phi(\trivialSubgroup)\subgroup\trivialSubgroup$. We have, from
Theorem~{\Tt{}Th1\nwendquote}, that $\phi(\trivialSubgroup)=\trivialSubgroup$ since
$\phi$ (being an automorphism) is also an endomorphism. And {\Tt{}GROUP{\_}2\nwendquote}
proves that $G$ is a subgroup of itself in Theorem~{\Tt{}Th54\nwendquote}.

\nwenddocs{}\nwbegincode{15}\sublabel{NW1Nnu7l-3J82fO-1}\nwmargintag{{\nwtagstyle{}\subpageref{NW1Nnu7l-3J82fO-1}}}\moddef{Automorphisms map trivial subgroups to themselves~{\nwtagstyle{}\subpageref{NW1Nnu7l-3J82fO-1}}}\endmoddef\nwstartdeflinemarkup\nwusesondefline{\\{NW1Nnu7l-32pmQO-1}}\nwenddeflinemarkup
:: In particular, the trivial proper subgroup (1).G of G is invariant
:: under inner automorphisms, and thus is a characteristic subgroup.
theorem Th2:
  for phi being Automorphism of G
  holds Image(phi|(1).G) is Subgroup of (1).G
proof
  let phi be Automorphism of G;
  (1).G is Subgroup of (1).G by GROUP_2:54;
  hence Image(phi|(1).G) is Subgroup of (1).G by Th1;
end;

\nwused{\\{NW1Nnu7l-32pmQO-1}}\nwendcode{}\nwbegindocs{16}\nwdocspar

\N{Identity Group Endomorphism}
I found it tedious to write {\Tt{}id\ (the\ carrier\ of\ G)\nwendquote} over and over
again, so I wrote a shortcut {\Tt{}Id{\_}G\nwendquote} for $\Id_{G}$.

\nwenddocs{}\nwbegincode{17}\sublabel{NW1Nnu7l-27i33n-1}\nwmargintag{{\nwtagstyle{}\subpageref{NW1Nnu7l-27i33n-1}}}\moddef{Define $\Id_{G}$~{\nwtagstyle{}\subpageref{NW1Nnu7l-27i33n-1}}}\endmoddef\nwstartdeflinemarkup\nwusesondefline{\\{NW1Nnu7l-32pmQO-1}}\nwenddeflinemarkup
definition :: Def3
  let G;
  func Id_G -> Endomorphism of G means
  :Def3:
  for x being Element of G holds it.x=x;
  existence
  \LA{}Proof that $\Id_{G}$ exists~{\nwtagstyle{}\subpageref{NW1Nnu7l-2hCvdU-1}}\RA{}
  uniqueness
  \LA{}Proof $\Id_{G}$ is unique~{\nwtagstyle{}\subpageref{NW1Nnu7l-q6iKQ-1}}\RA{}
end;

\nwused{\\{NW1Nnu7l-32pmQO-1}}\nwendcode{}\nwbegindocs{18}\nwdocspar

\M We need to add {\Tt{}Id{\_}\nwendquote} to our vocabulary.

\nwenddocs{}\nwbegincode{19}\sublabel{NW1Nnu7l-9L8EQ-1}\nwmargintag{{\nwtagstyle{}\subpageref{NW1Nnu7l-9L8EQ-1}}}\moddef{DICT/TMP.VOC~{\nwtagstyle{}\subpageref{NW1Nnu7l-9L8EQ-1}}}\endmoddef\nwstartdeflinemarkup\nwprevnextdefs{\relax}{NW1Nnu7l-9L8EQ-2}\nwenddeflinemarkup
OId_

\nwalsodefined{\\{NW1Nnu7l-9L8EQ-2}}\nwnotused{DICT/TMP.VOC}\nwendcode{}\nwbegindocs{20}\nwdocspar

\N{Proof ($\Id_{G}$ exists)}
Similar to the proof (\S\ref{par:characteristic:register-bijective-for-endomorphism})
that there exists a bijective endomorphism, the proof that $\Id_{G}$
exists amounts to reconsidering {\Tt{}id\nwendquote} as an endomorphism. Earlier work
in {\Tt{}FUNCT{\_}1\nwendquote} proved (in Theorem~17) the identity function satisfies
$\forall x\in X, \id_{X}(x)=x$.

\nwenddocs{}\nwbegincode{21}\sublabel{NW1Nnu7l-2hCvdU-1}\nwmargintag{{\nwtagstyle{}\subpageref{NW1Nnu7l-2hCvdU-1}}}\moddef{Proof that $\Id_{G}$ exists~{\nwtagstyle{}\subpageref{NW1Nnu7l-2hCvdU-1}}}\endmoddef\nwstartdeflinemarkup\nwusesondefline{\\{NW1Nnu7l-27i33n-1}}\nwenddeflinemarkup
proof
  reconsider i = id (the carrier of G) as Homomorphism of G,G by GROUP_6:38;
  i.x = x by FUNCT_1:17;
  hence thesis;
end;

\nwused{\\{NW1Nnu7l-27i33n-1}}\nwendcode{}\nwbegindocs{22}\nwdocspar

\N{Proof (Uniqueness of $\Id_{G}$)}
The usual strategy is to consider two arbitrary endomorphisms
$\Id^{(1)}_{G}$ and $\Id^{(2)}_{G}$ satisfying the definition of
{\Tt{}Id{\_}\nwendquote}, then prove $\Id^{(1)}_{G} = \Id^{(2)}_{G}$.

If we wanted to be completely pedantic, we could cite Theorem~12 from
{\Tt{}FUNCT{\_}2\nwendquote} which states for any functions of sets
$f_{1}$, $f_{2}\colon X\to Y$ we have $\forall x\in X, f_{1}(x)=f_{2}(x)$
implies $f_{1}=f_{2}$.

\nwenddocs{}\nwbegincode{23}\sublabel{NW1Nnu7l-q6iKQ-1}\nwmargintag{{\nwtagstyle{}\subpageref{NW1Nnu7l-q6iKQ-1}}}\moddef{Proof $\Id_{G}$ is unique~{\nwtagstyle{}\subpageref{NW1Nnu7l-q6iKQ-1}}}\endmoddef\nwstartdeflinemarkup\nwusesondefline{\\{NW1Nnu7l-27i33n-1}}\nwenddeflinemarkup
proof
  let Id1, Id2 be Endomorphism of G such that
  A1: for x being Element of G holds Id1.x=x and
  A2: for x being Element of G holds Id2.x=x;
  let x be Element of G;
  thus Id1.x = x by A1
            .= Id2.x by A2;
  thus thesis;
end;

\nwused{\\{NW1Nnu7l-27i33n-1}}\nwendcode{}\nwbegindocs{24}\nwdocspar

\M\label{thm:characteristic:identity-endomorphism-is-identity-function}
We should now have some sanity check that {\Tt{}Id{\_}G\nwendquote} really is
the identity group morphism $\Id_{G}$ that \emph{we} think it is.

\nwenddocs{}\nwbegincode{25}\sublabel{NW1Nnu7l-4KG2Xs-1}\nwmargintag{{\nwtagstyle{}\subpageref{NW1Nnu7l-4KG2Xs-1}}}\moddef{\code{}Id{\_}G\edoc{} is the same as \code{}id\ the\ carrier\ of\ G\edoc{}~{\nwtagstyle{}\subpageref{NW1Nnu7l-4KG2Xs-1}}}\endmoddef\nwstartdeflinemarkup\nwusesondefline{\\{NW1Nnu7l-32pmQO-1}}\nwenddeflinemarkup
theorem Th3:
  Id_G = id (the carrier of G)
proof
  let x be Element of G;
  thus A4: (Id_G).x = x by Def3
                   .= (id (the carrier of G)).x;
  thus thesis;
end;

\nwused{\\{NW1Nnu7l-32pmQO-1}}\nwendcode{}\nwbegindocs{26}\nwdocspar

\M
Now we should automatically associate $\Id_{G}$ is bijective. This
requires proving a couple of helper lemmas establishing injectivity and
surjectivity.

\nwenddocs{}\nwbegincode{27}\sublabel{NW1Nnu7l-37g7e5-1}\nwmargintag{{\nwtagstyle{}\subpageref{NW1Nnu7l-37g7e5-1}}}\moddef{Register \code{}Id{\_}G\edoc{} is bijective~{\nwtagstyle{}\subpageref{NW1Nnu7l-37g7e5-1}}}\endmoddef\nwstartdeflinemarkup\nwusesondefline{\\{NW1Nnu7l-32pmQO-1}}\nwenddeflinemarkup
\LA{}\code{}Id{\_}G\edoc{} is injective~{\nwtagstyle{}\subpageref{NW1Nnu7l-17utZy-1}}\RA{}
\LA{}\code{}Id{\_}G\edoc{} is surjective~{\nwtagstyle{}\subpageref{NW1Nnu7l-2Qowg2-1}}\RA{}

registration let G;
  cluster Id_G -> bijective;
  coherence by Lm1,Lm2;
end;

\nwused{\\{NW1Nnu7l-32pmQO-1}}\nwendcode{}\nwbegindocs{28}\nwdocspar

\N{Identity is injective}
The proof is straightforward since we've established (\S\ref{thm:characteristic:identity-endomorphism-is-identity-function}) the group
endomorphism $\Id_{G}$ coincides with the set-theoretic function $\id_{U(G)}$
on the underlying set $U(G)$ of the group $G$.

\nwenddocs{}\nwbegincode{29}\sublabel{NW1Nnu7l-17utZy-1}\nwmargintag{{\nwtagstyle{}\subpageref{NW1Nnu7l-17utZy-1}}}\moddef{\code{}Id{\_}G\edoc{} is injective~{\nwtagstyle{}\subpageref{NW1Nnu7l-17utZy-1}}}\endmoddef\nwstartdeflinemarkup\nwusesondefline{\\{NW1Nnu7l-37g7e5-1}}\nwenddeflinemarkup
Lm1: Id_G is one-to-one
proof
  id (the carrier of G) is one-to-one;
  hence thesis by Th3;
end;

\nwused{\\{NW1Nnu7l-37g7e5-1}}\nwendcode{}\nwbegindocs{30}\nwdocspar

\N{Identity is surjective}
As with establishing injectivity, it's straightforward.

\nwenddocs{}\nwbegincode{31}\sublabel{NW1Nnu7l-2Qowg2-1}\nwmargintag{{\nwtagstyle{}\subpageref{NW1Nnu7l-2Qowg2-1}}}\moddef{\code{}Id{\_}G\edoc{} is surjective~{\nwtagstyle{}\subpageref{NW1Nnu7l-2Qowg2-1}}}\endmoddef\nwstartdeflinemarkup\nwusesondefline{\\{NW1Nnu7l-37g7e5-1}}\nwenddeflinemarkup
Lm2: Id_G is onto
proof
  id (the carrier of G) is onto;
  hence thesis by Th3;
end;

\nwused{\\{NW1Nnu7l-37g7e5-1}}\nwendcode{}\nwbegindocs{32}\nwdocspar

\subsubsection{Inner Automorphisms}

\N{Definition: Inner Automorphism}
We call a group automorphism $f\in\Aut(G)$ \define{inner} if there is a
group element $g\in G$ such that for all $x\in G$ we have $f(x) = x^{g} = g^{-1}xg$.
That is, $f$ is just conjugation by a fixed group element.

\nwenddocs{}\nwbegincode{33}\sublabel{NW1Nnu7l-13QmRM-1}\nwmargintag{{\nwtagstyle{}\subpageref{NW1Nnu7l-13QmRM-1}}}\moddef{Define \code{}inner\edoc{} for Automorphism~{\nwtagstyle{}\subpageref{NW1Nnu7l-13QmRM-1}}}\endmoddef\nwstartdeflinemarkup\nwusesondefline{\\{NW1Nnu7l-32pmQO-1}}\nwenddeflinemarkup
definition :: Def5
  let G;
  let IT be Automorphism of G;
  attr IT is inner means
  :Def5:
  ex a being Element of G st
  for x being Element of G holds IT.x = x |^ a;
end;
\LA{}Outer as antonym of inner~{\nwtagstyle{}\subpageref{NW1Nnu7l-RgKDM-1}}\RA{}

\nwused{\\{NW1Nnu7l-32pmQO-1}}\nwendcode{}\nwbegindocs{34}\nwdocspar

\M We also recall that an automorphism is called \define{Outer} if it is
not inner. Mizar let's us do this with the {\Tt{}antonym\nwendquote} construct within
a {\Tt{}notation\nwendquote} block.

\nwenddocs{}\nwbegincode{35}\sublabel{NW1Nnu7l-RgKDM-1}\nwmargintag{{\nwtagstyle{}\subpageref{NW1Nnu7l-RgKDM-1}}}\moddef{Outer as antonym of inner~{\nwtagstyle{}\subpageref{NW1Nnu7l-RgKDM-1}}}\endmoddef\nwstartdeflinemarkup\nwusesondefline{\\{NW1Nnu7l-13QmRM-1}}\nwenddeflinemarkup
notation
  let G be Group, f be Automorphism of G;
  antonym f is outer for f is inner;
end;

\nwused{\\{NW1Nnu7l-13QmRM-1}}\nwendcode{}\nwbegindocs{36}\nwdocspar

\N{Update our vocabular file}
Before rushing off to prove properties concerning inner and outer
automorphisms, we should add the attributes to our vocabulary file.


\nwenddocs{}\nwbegincode{37}\sublabel{NW1Nnu7l-9L8EQ-2}\nwmargintag{{\nwtagstyle{}\subpageref{NW1Nnu7l-9L8EQ-2}}}\moddef{DICT/TMP.VOC~{\nwtagstyle{}\subpageref{NW1Nnu7l-9L8EQ-1}}}\plusendmoddef\nwstartdeflinemarkup\nwprevnextdefs{NW1Nnu7l-9L8EQ-1}{\relax}\nwenddeflinemarkup
Vinner
Vouter

\nwendcode{}\nwbegindocs{38}\nwdocspar

\N{Theorem: $\Id_{G}$ is effectively inner}\label{thm:characteristic:id-inner}
We will be registering ``inner'' as an attribute for ``Automorphism
of $G$''. This will require proving that there exists an inner
Automorphism of $G$. I've found the trivial examples are often best for
establishing the existence of such things, so we will prove {\Tt{}Id\ G\nwendquote} is
an inner Automorphism. This uses the fact, if $e\in G$ is the identity
element, then for any $g\in G$ we have conjugation $g^{e} = e^{-1}ge=g$
(proven in Theorem~19 of {\Tt{}GROUP{\_}3\nwendquote}).

\nwenddocs{}\nwbegincode{39}\sublabel{NW1Nnu7l-3VEX6S-1}\nwmargintag{{\nwtagstyle{}\subpageref{NW1Nnu7l-3VEX6S-1}}}\moddef{\code{}Id{\_}G\edoc{} is effectively inner~{\nwtagstyle{}\subpageref{NW1Nnu7l-3VEX6S-1}}}\endmoddef\nwstartdeflinemarkup\nwusesondefline{\\{NW1Nnu7l-32pmQO-1}}\nwenddeflinemarkup
theorem Th5:
  for x being Element of G holds (Id_G).x = x |^ 1_G
proof
  let x;
  (Id_G).x = x by Def3
          .= x |^ 1_G by GROUP_3:19;
  hence thesis;
end;

\nwused{\\{NW1Nnu7l-32pmQO-1}}\nwendcode{}\nwbegindocs{40}\nwdocspar

\M Now registering {\Tt{}inner\nwendquote} for {\Tt{}Automorphism\nwendquote}.

\nwenddocs{}\nwbegincode{41}\sublabel{NW1Nnu7l-4IjmGJ-1}\nwmargintag{{\nwtagstyle{}\subpageref{NW1Nnu7l-4IjmGJ-1}}}\moddef{Register \code{}inner\edoc{} for \code{}Automorphism\edoc{}~{\nwtagstyle{}\subpageref{NW1Nnu7l-4IjmGJ-1}}}\endmoddef\nwstartdeflinemarkup\nwusesondefline{\\{NW1Nnu7l-32pmQO-1}}\nwenddeflinemarkup
registration
  let G;
  cluster inner for Automorphism of G;
  existence
  \LA{}Proof of existence of an inner Automorphism~{\nwtagstyle{}\subpageref{NW1Nnu7l-9idpO-1}}\RA{}
end;

\nwused{\\{NW1Nnu7l-32pmQO-1}}\nwendcode{}\nwbegindocs{42}\nwdocspar

\N{Proof} The proof is a two punch knock-out. We take {\Tt{}Id{\_}G\nwendquote} to be the
morphism, {\Tt{}1{\_}g\nwendquote} the group's identity element to be the element
{\Tt{}Id{\_}G\nwendquote} conjugates by, then from earlier (\S\ref{thm:characteristic:id-inner})
we have {\Tt{}Id{\_}G\nwendquote} be inner.

\nwenddocs{}\nwbegincode{43}\sublabel{NW1Nnu7l-9idpO-1}\nwmargintag{{\nwtagstyle{}\subpageref{NW1Nnu7l-9idpO-1}}}\moddef{Proof of existence of an inner Automorphism~{\nwtagstyle{}\subpageref{NW1Nnu7l-9idpO-1}}}\endmoddef\nwstartdeflinemarkup\nwusesondefline{\\{NW1Nnu7l-4IjmGJ-1}}\nwenddeflinemarkup
proof
  take Id_G;
  take 1_G;
  thus thesis by Th5;
end;

\nwused{\\{NW1Nnu7l-4IjmGJ-1}}\nwendcode{}\nwbegindocs{44}\nwdocspar

\N{Theorem ($f\in\aut(G)\iff f$ is {\Tt{}Automorphism\ of\ G\nwendquote})}
Mizar has {\Tt{}AUTGROUP\nwendquote}, an article which defines {\Tt{}Aut\ G\nwendquote} the
collection of functions on the underlying set $U(G)$ of a group $G$. We
can prove that $f\in\aut(G)$ if and only if $f$ is {\Tt{}Automorphism\ of\ G\nwendquote}.

\nwenddocs{}\nwbegincode{45}\sublabel{NW1Nnu7l-3c36xj-1}\nwmargintag{{\nwtagstyle{}\subpageref{NW1Nnu7l-3c36xj-1}}}\moddef{Relate \code{}Automorphism\ of\ G\edoc{} to elements of \code{}Aut\ G\edoc{}~{\nwtagstyle{}\subpageref{NW1Nnu7l-3c36xj-1}}}\endmoddef\nwstartdeflinemarkup\nwusesondefline{\\{NW1Nnu7l-32pmQO-1}}\nwenddeflinemarkup
theorem Th8:
  for G being strict Group, f being object
  holds (f in Aut G) iff (f is Automorphism of G)
proof
  let G be strict Group;
  let f be object;
  thus f in Aut G implies f is Automorphism of G
  \LA{}Proof $f\in\aut(G)\implies f$ is \code{}Automorphism\ of\ G\edoc{}~{\nwtagstyle{}\subpageref{NW1Nnu7l-wlMQo-1}}\RA{}
  thus f is Automorphism of G implies f in Aut G
  \LA{}Proof $f\in\aut(G)\impliedby f$ is \code{}Automorphism\ of\ G\edoc{}~{\nwtagstyle{}\subpageref{NW1Nnu7l-21iyLo-1}}\RA{}
  thus thesis;
end;

\nwused{\\{NW1Nnu7l-32pmQO-1}}\nwendcode{}\nwbegindocs{46}\nwdocspar

\N{Proof forwards direction} The forward direction is
straightforward. The only subtlety is, since we didn't assume anything
about $f$, we should establish it's an endomorphism of $G$ along the way,

\nwenddocs{}\nwbegincode{47}\sublabel{NW1Nnu7l-wlMQo-1}\nwmargintag{{\nwtagstyle{}\subpageref{NW1Nnu7l-wlMQo-1}}}\moddef{Proof $f\in\aut(G)\implies f$ is \code{}Automorphism\ of\ G\edoc{}~{\nwtagstyle{}\subpageref{NW1Nnu7l-wlMQo-1}}}\endmoddef\nwstartdeflinemarkup\nwusesondefline{\\{NW1Nnu7l-3c36xj-1}}\nwenddeflinemarkup
proof
  assume A0: f in Aut G;
  then reconsider f as Endomorphism of G by AUTGROUP:def 1;
  f is bijective by A0,AUTGROUP:def 1;
  then f is Automorphism of G;
  hence thesis;
end;

\nwused{\\{NW1Nnu7l-3c36xj-1}}\nwendcode{}\nwbegindocs{48}\nwdocspar

\N{Proof backwards direction} The backwards direction is nearly
identical to the forwards direction proof.

\nwenddocs{}\nwbegincode{49}\sublabel{NW1Nnu7l-21iyLo-1}\nwmargintag{{\nwtagstyle{}\subpageref{NW1Nnu7l-21iyLo-1}}}\moddef{Proof $f\in\aut(G)\impliedby f$ is \code{}Automorphism\ of\ G\edoc{}~{\nwtagstyle{}\subpageref{NW1Nnu7l-21iyLo-1}}}\endmoddef\nwstartdeflinemarkup\nwusesondefline{\\{NW1Nnu7l-3c36xj-1}}\nwenddeflinemarkup
proof
  assume f is Automorphism of G;
  then reconsider f as Automorphism of G;
  f is bijective;
  then f in Aut G by AUTGROUP:def 1;
  hence thesis;
end;

\nwused{\\{NW1Nnu7l-3c36xj-1}}\nwendcode{}\nwbegindocs{50}\nwdocspar

\N{Inner automorphisms are inner automorphisms}
We can relate the notion of an {\Tt{}inner\ Automorphism\ of\ G\nwendquote} with elements
of {\Tt{}InnAut\ G\nwendquote} from {\Tt{}AUTGROUP\nwendquote}. The only peculiarity is that
{\Tt{}AUTGROUP\nwendquote} requires $G$ to be a \emph{strict} group.

\nwenddocs{}\nwbegincode{51}\sublabel{NW1Nnu7l-2gUxot-1}\nwmargintag{{\nwtagstyle{}\subpageref{NW1Nnu7l-2gUxot-1}}}\moddef{Theorem: $f$ in \code{}InnAut\ G\edoc{} iff $f$ is \code{}inner\ Automorphism\ of\ G\edoc{}~{\nwtagstyle{}\subpageref{NW1Nnu7l-2gUxot-1}}}\endmoddef\nwstartdeflinemarkup\nwusesondefline{\\{NW1Nnu7l-32pmQO-1}}\nwenddeflinemarkup
\LA{}Lemma: Elements of \code{}InnAut\ G\edoc{} are automorphisms~{\nwtagstyle{}\subpageref{NW1Nnu7l-3eJW7Q-1}}\RA{}

theorem Th9:
  for G being strict Group
  for f being object
  holds (f in InnAut G) iff (f is inner Automorphism of G)
proof
  let G be strict Group;
  let f be object;
  thus (f in InnAut G) implies (f is inner Automorphism of G)
  \LA{}Proof $f$ is in \code{}InnAut\ G\edoc{} $\implies$ ($f$ is inner automorphism)~{\nwtagstyle{}\subpageref{NW1Nnu7l-18aBqf-1}}\RA{}
  thus (f is inner Automorphism of G) implies (f in InnAut G)
  \LA{}Proof ($f$ is inner automorphism) $\implies$ $f$ is in \code{}InnAut\ G\edoc{}~{\nwtagstyle{}\subpageref{NW1Nnu7l-2maG0l-1}}\RA{}
  thus thesis;
end;

\nwused{\\{NW1Nnu7l-32pmQO-1}}\nwendcode{}\nwbegindocs{52}\nwdocspar

\N{Proof in forwards direction} The proof amounts to unwinding
definitions, but the subtlety is in first reconsidering $f$ as an
Automorphism of $G$ thanks to our handy-dandy lemma.

\nwenddocs{}\nwbegincode{53}\sublabel{NW1Nnu7l-18aBqf-1}\nwmargintag{{\nwtagstyle{}\subpageref{NW1Nnu7l-18aBqf-1}}}\moddef{Proof $f$ is in \code{}InnAut\ G\edoc{} $\implies$ ($f$ is inner automorphism)~{\nwtagstyle{}\subpageref{NW1Nnu7l-18aBqf-1}}}\endmoddef\nwstartdeflinemarkup\nwusesondefline{\\{NW1Nnu7l-2gUxot-1}}\nwenddeflinemarkup
proof
  assume Z0: f in InnAut G;
  then f is Automorphism of G by LmInnAut;
  then reconsider f as Automorphism of G;
  f is Element of Funcs (the carrier of G, the carrier of G) by FUNCT_2:9;
  then consider a being Element of G such that
  A2: for x being Element of G holds f.x = x |^ a
  by Z0,AUTGROUP:def 4;
  f is inner Automorphism of G by Def5,A2;
  hence thesis;
end;

\nwused{\\{NW1Nnu7l-2gUxot-1}}\nwendcode{}\nwbegindocs{54}\nwdocspar

\N{Proof in backwards direction}
This is again unwinding the definitions. The same subtlety lurks here,
requiring us to reconsider $f$ as an inner automorphism of $G$.

\nwenddocs{}\nwbegincode{55}\sublabel{NW1Nnu7l-2maG0l-1}\nwmargintag{{\nwtagstyle{}\subpageref{NW1Nnu7l-2maG0l-1}}}\moddef{Proof ($f$ is inner automorphism) $\implies$ $f$ is in \code{}InnAut\ G\edoc{}~{\nwtagstyle{}\subpageref{NW1Nnu7l-2maG0l-1}}}\endmoddef\nwstartdeflinemarkup\nwusesondefline{\\{NW1Nnu7l-2gUxot-1}}\nwenddeflinemarkup
proof
  assume Z1: f is inner Automorphism of G;
  reconsider f as inner Automorphism of G by Z1;
  f is Element of Aut G by Z1,Th8;
  then consider a being Element of G such that
  B1: for x being Element of G holds f.x = x |^ a
  by Z1,Def5;
  f is Element of Funcs (the carrier of G, the carrier of G) by FUNCT_2:9;
  then f in InnAut G by B1,AUTGROUP:def 4;
  hence thesis;
end;

\nwused{\\{NW1Nnu7l-2gUxot-1}}\nwendcode{}\nwbegindocs{56}\nwdocspar

\M
It's relatively straightforward to show that if $f$ is an element of
{\Tt{}InnAut\ G\nwendquote}, then $f$ is an {\Tt{}Automorphism\ of\ G\nwendquote}. We just unwind the
definitions.

\nwenddocs{}\nwbegincode{57}\sublabel{NW1Nnu7l-3eJW7Q-1}\nwmargintag{{\nwtagstyle{}\subpageref{NW1Nnu7l-3eJW7Q-1}}}\moddef{Lemma: Elements of \code{}InnAut\ G\edoc{} are automorphisms~{\nwtagstyle{}\subpageref{NW1Nnu7l-3eJW7Q-1}}}\endmoddef\nwstartdeflinemarkup\nwusesondefline{\\{NW1Nnu7l-2gUxot-1}}\nwenddeflinemarkup
LmInnAut:
  for G being strict Group
  for f being Element of InnAut G
  holds f is Automorphism of G
proof
  let G be strict Group;
  let f be Element of InnAut G;
  f is Element of Aut G by AUTGROUP:12;
  then f in Aut G;
  hence f is Automorphism of G by Th8;
end;

\nwused{\\{NW1Nnu7l-2gUxot-1}}\nwendcode{}\nwbegindocs{58}\nwdocspar

\N{Theorem}
Given any element $a\in G$, and any inner automorphism $f$ of $G$ such
that $\forall x\in G, f(x) = x^{a} = a^{-1}xa$, it follows that the
image of a subgroup $f(H) = H^{a}$ is the conjugate of that subgroup.

\nwenddocs{}\nwbegincode{59}\sublabel{NW1Nnu7l-2zspfg-1}\nwmargintag{{\nwtagstyle{}\subpageref{NW1Nnu7l-2zspfg-1}}}\moddef{Theorem: inner automorphism acting on subgroup is conjugate of argument~{\nwtagstyle{}\subpageref{NW1Nnu7l-2zspfg-1}}}\endmoddef\nwstartdeflinemarkup\nwusesondefline{\\{NW1Nnu7l-32pmQO-1}}\nwenddeflinemarkup
theorem Th7:
  for a being Element of G
  for f being inner Automorphism of G
  st (for x being Element of G holds f.x = x |^ a)
  holds Image(f|H) = H |^ a
proof
  let a be Element of G,
      f be inner Automorphism of G;
  assume
A0: for x being Element of G holds f.x = x |^ a;
C1: for h being Element of G st h in H holds (f|H).h = h |^ a
  proof
    let h be Element of G;
    assume h in H;
    then Z1: f.h = (f|H).h by Lm3;
    f.h = h |^ a by A0;
    hence (f|H).h = h |^ a by Z1;
  end;

C2: for y being Element of G st y in Image(f|H) holds y in H |^ a
  \LA{}Proof $\forall y\in G, y\in f(H)\implies y\in H^{a}$~{\nwtagstyle{}\subpageref{NW1Nnu7l-1BjobC-1}}\RA{}
C3: for y being Element of G st y in H |^ a holds y in Image(f|H)
  \LA{}Proof $\forall y\in G, y \in H^{a}\implies y\in f(H)$~{\nwtagstyle{}\subpageref{NW1Nnu7l-AZduT-1}}\RA{}
  for y being Element of G holds y in (H |^ a) iff y in Image(f|H) by C2,C3;
  then (H |^ a) = Image(f|H) by GROUP_2:def 6;
  hence thesis;
end;
  
\nwused{\\{NW1Nnu7l-32pmQO-1}}\nwendcode{}\nwbegindocs{60}\nwdocspar

\nwenddocs{}\nwbegincode{61}\sublabel{NW1Nnu7l-1BjobC-1}\nwmargintag{{\nwtagstyle{}\subpageref{NW1Nnu7l-1BjobC-1}}}\moddef{Proof $\forall y\in G, y\in f(H)\implies y\in H^{a}$~{\nwtagstyle{}\subpageref{NW1Nnu7l-1BjobC-1}}}\endmoddef\nwstartdeflinemarkup\nwusesondefline{\\{NW1Nnu7l-2zspfg-1}}\nwenddeflinemarkup
proof
  let y be Element of G;
  assume y in Image(f|H);
  then consider h being Element of H such that
  AA1: (f|H).h = y by GROUP_6:45;
  h is Element of G by GROUP_2:42;
  then reconsider h as Element of G;
  AA2: h in H by STRUCT_0:def 5;
  then (f|H).h = h |^ a by C1;
  then y = h |^ a by AA1;
  then y = h |^ a & h in H by AA2; 
  then y in H |^ a by GROUP_3:58;
  hence thesis;
end;
    
\nwused{\\{NW1Nnu7l-2zspfg-1}}\nwendcode{}\nwbegindocs{62}\nwdocspar


\nwenddocs{}\nwbegincode{63}\sublabel{NW1Nnu7l-AZduT-1}\nwmargintag{{\nwtagstyle{}\subpageref{NW1Nnu7l-AZduT-1}}}\moddef{Proof $\forall y\in G, y \in H^{a}\implies y\in f(H)$~{\nwtagstyle{}\subpageref{NW1Nnu7l-AZduT-1}}}\endmoddef\nwstartdeflinemarkup\nwusesondefline{\\{NW1Nnu7l-2zspfg-1}}\nwenddeflinemarkup
proof
  let y be Element of G;
  assume y in H |^ a;
  then ex g being Element of G st y = g |^ a & g in H by GROUP_3:58;
  then consider g being Element of G such that
  AB1:    y=g|^a and
  AB2:    g in H;
    
AB3:  (f|H).g = f.g by Lm3,AB2
      .= g |^ a by A0
      .= y by AB1;
  reconsider g as Element of H by AB2,LmSubgroupElt;
  reconsider fH = (f|H) as Homomorphism of H,G;
  fH.g = y by AB3;
  hence y in Image(f|H) by GROUP_6:45;
end;

\nwused{\\{NW1Nnu7l-2zspfg-1}}\nwendcode{}\nwbegindocs{64}\nwdocspar

\N{Theorem} Given a group element $a\in G$, we can always construct an
inner automorphism $f\in\Inn(G)$ defined by
$\forall x\in G,f(x)=x^{a}=a^{-1}xa$ conjugation by $a$.

\nwenddocs{}\nwbegincode{65}\sublabel{NW1Nnu7l-4NcpAX-1}\nwmargintag{{\nwtagstyle{}\subpageref{NW1Nnu7l-4NcpAX-1}}}\moddef{Theorem: conjugation of given element is an inner automorphism~{\nwtagstyle{}\subpageref{NW1Nnu7l-4NcpAX-1}}}\endmoddef\nwstartdeflinemarkup\nwusesondefline{\\{NW1Nnu7l-32pmQO-1}}\nwenddeflinemarkup
theorem Th10:
  for G being strict Group
  for a being Element of G
  holds ex f being inner Automorphism of G st (for x being Element of G
  holds f.x = x |^ a)
proof
  let G be strict Group;
  let a be Element of G;
  reconsider f = (Conjugate a) as inner Automorphism of G
  by AUTGROUP:def 6, Th9;
  take f;
  let x be Element of G;
  thus f.x = x |^ a by AUTGROUP:def 6;
end;

\nwused{\\{NW1Nnu7l-32pmQO-1}}\nwendcode{}\nwbegindocs{66}\nwdocspar

\nwenddocs{}

\nwixlogsorted{c}{{\code{}Id{\_}G\edoc{} is effectively inner}{NW1Nnu7l-3VEX6S-1}{\nwixu{NW1Nnu7l-32pmQO-1}\nwixd{NW1Nnu7l-3VEX6S-1}}}%
\nwixlogsorted{c}{{\code{}Id{\_}G\edoc{} is injective}{NW1Nnu7l-17utZy-1}{\nwixu{NW1Nnu7l-37g7e5-1}\nwixd{NW1Nnu7l-17utZy-1}}}%
\nwixlogsorted{c}{{\code{}Id{\_}G\edoc{} is surjective}{NW1Nnu7l-2Qowg2-1}{\nwixu{NW1Nnu7l-37g7e5-1}\nwixd{NW1Nnu7l-2Qowg2-1}}}%
\nwixlogsorted{c}{{\code{}Id{\_}G\edoc{} is the same as \code{}id\ the\ carrier\ of\ G\edoc{}}{NW1Nnu7l-4KG2Xs-1}{\nwixu{NW1Nnu7l-32pmQO-1}\nwixd{NW1Nnu7l-4KG2Xs-1}}}%
\nwixlogsorted{c}{{Automorphisms map trivial subgroups to themselves}{NW1Nnu7l-3J82fO-1}{\nwixu{NW1Nnu7l-32pmQO-1}\nwixd{NW1Nnu7l-3J82fO-1}}}%
\nwixlogsorted{c}{{Define $\Id_{G}$}{NW1Nnu7l-27i33n-1}{\nwixu{NW1Nnu7l-32pmQO-1}\nwixd{NW1Nnu7l-27i33n-1}}}%
\nwixlogsorted{c}{{Define \code{}Automorphism\edoc{}}{NW1Nnu7l-4WeeL5-1}{\nwixu{NW1Nnu7l-32pmQO-1}\nwixd{NW1Nnu7l-4WeeL5-1}}}%
\nwixlogsorted{c}{{Define \code{}Endomorphism\edoc{}}{NW1Nnu7l-17cKlQ-1}{\nwixu{NW1Nnu7l-32pmQO-1}\nwixd{NW1Nnu7l-17cKlQ-1}}}%
\nwixlogsorted{c}{{Define \code{}inner\edoc{} for Automorphism}{NW1Nnu7l-13QmRM-1}{\nwixu{NW1Nnu7l-32pmQO-1}\nwixd{NW1Nnu7l-13QmRM-1}}}%
\nwixlogsorted{c}{{DICT/TMP.VOC}{NW1Nnu7l-9L8EQ-1}{\nwixd{NW1Nnu7l-9L8EQ-1}\nwixd{NW1Nnu7l-9L8EQ-2}}}%
\nwixlogsorted{c}{{Endomorphisms preserve the trivial subgroup}{NW1Nnu7l-26jvGu-1}{\nwixu{NW1Nnu7l-32pmQO-1}\nwixd{NW1Nnu7l-26jvGu-1}}}%
\nwixlogsorted{c}{{Inner and outer automorphisms}{NW1Nnu7l-32pmQO-1}{\nwixd{NW1Nnu7l-32pmQO-1}}}%
\nwixlogsorted{c}{{Lemma: Elements of \code{}InnAut\ G\edoc{} are automorphisms}{NW1Nnu7l-3eJW7Q-1}{\nwixu{NW1Nnu7l-2gUxot-1}\nwixd{NW1Nnu7l-3eJW7Q-1}}}%
\nwixlogsorted{c}{{Outer as antonym of inner}{NW1Nnu7l-RgKDM-1}{\nwixu{NW1Nnu7l-13QmRM-1}\nwixd{NW1Nnu7l-RgKDM-1}}}%
\nwixlogsorted{c}{{Proof $\forall y\in G, y \in H^{a}\implies y\in f(H)$}{NW1Nnu7l-AZduT-1}{\nwixu{NW1Nnu7l-2zspfg-1}\nwixd{NW1Nnu7l-AZduT-1}}}%
\nwixlogsorted{c}{{Proof $\forall y\in G, y\in f(H)\implies y\in H^{a}$}{NW1Nnu7l-1BjobC-1}{\nwixu{NW1Nnu7l-2zspfg-1}\nwixd{NW1Nnu7l-1BjobC-1}}}%
\nwixlogsorted{c}{{Proof $\Id_{G}$ is unique}{NW1Nnu7l-q6iKQ-1}{\nwixu{NW1Nnu7l-27i33n-1}\nwixd{NW1Nnu7l-q6iKQ-1}}}%
\nwixlogsorted{c}{{Proof $f$ is in \code{}InnAut\ G\edoc{} $\implies$ ($f$ is inner automorphism)}{NW1Nnu7l-18aBqf-1}{\nwixu{NW1Nnu7l-2gUxot-1}\nwixd{NW1Nnu7l-18aBqf-1}}}%
\nwixlogsorted{c}{{Proof $f\in\aut(G)\impliedby f$ is \code{}Automorphism\ of\ G\edoc{}}{NW1Nnu7l-21iyLo-1}{\nwixu{NW1Nnu7l-3c36xj-1}\nwixd{NW1Nnu7l-21iyLo-1}}}%
\nwixlogsorted{c}{{Proof $f\in\aut(G)\implies f$ is \code{}Automorphism\ of\ G\edoc{}}{NW1Nnu7l-wlMQo-1}{\nwixu{NW1Nnu7l-3c36xj-1}\nwixd{NW1Nnu7l-wlMQo-1}}}%
\nwixlogsorted{c}{{Proof ($f$ is inner automorphism) $\implies$ $f$ is in \code{}InnAut\ G\edoc{}}{NW1Nnu7l-2maG0l-1}{\nwixu{NW1Nnu7l-2gUxot-1}\nwixd{NW1Nnu7l-2maG0l-1}}}%
\nwixlogsorted{c}{{Proof of existence of an inner Automorphism}{NW1Nnu7l-9idpO-1}{\nwixu{NW1Nnu7l-4IjmGJ-1}\nwixd{NW1Nnu7l-9idpO-1}}}%
\nwixlogsorted{c}{{Proof that $\Id_{G}$ exists}{NW1Nnu7l-2hCvdU-1}{\nwixu{NW1Nnu7l-27i33n-1}\nwixd{NW1Nnu7l-2hCvdU-1}}}%
\nwixlogsorted{c}{{Register \code{}bijective\edoc{} for \code{}Endoomorphism\edoc{}}{NW1Nnu7l-1BJmAm-1}{\nwixu{NW1Nnu7l-32pmQO-1}\nwixd{NW1Nnu7l-1BJmAm-1}}}%
\nwixlogsorted{c}{{Register \code{}Id{\_}G\edoc{} is bijective}{NW1Nnu7l-37g7e5-1}{\nwixu{NW1Nnu7l-32pmQO-1}\nwixd{NW1Nnu7l-37g7e5-1}}}%
\nwixlogsorted{c}{{Register \code{}inner\edoc{} for \code{}Automorphism\edoc{}}{NW1Nnu7l-4IjmGJ-1}{\nwixu{NW1Nnu7l-32pmQO-1}\nwixd{NW1Nnu7l-4IjmGJ-1}}}%
\nwixlogsorted{c}{{Relate \code{}Automorphism\ of\ G\edoc{} to elements of \code{}Aut\ G\edoc{}}{NW1Nnu7l-3c36xj-1}{\nwixu{NW1Nnu7l-32pmQO-1}\nwixd{NW1Nnu7l-3c36xj-1}}}%
\nwixlogsorted{c}{{Reserve symbols for inner and outer automorphisms}{NW1Nnu7l-1LL7AC-1}{\nwixu{NW1Nnu7l-32pmQO-1}\nwixd{NW1Nnu7l-1LL7AC-1}\nwixd{NW1Nnu7l-1LL7AC-2}}}%
\nwixlogsorted{c}{{Theorem: $f$ in \code{}InnAut\ G\edoc{} iff $f$ is \code{}inner\ Automorphism\ of\ G\edoc{}}{NW1Nnu7l-2gUxot-1}{\nwixu{NW1Nnu7l-32pmQO-1}\nwixd{NW1Nnu7l-2gUxot-1}}}%
\nwixlogsorted{c}{{Theorem: conjugation of given element is an inner automorphism}{NW1Nnu7l-4NcpAX-1}{\nwixu{NW1Nnu7l-32pmQO-1}\nwixd{NW1Nnu7l-4NcpAX-1}}}%
\nwixlogsorted{c}{{Theorem: inner automorphism acting on subgroup is conjugate of argument}{NW1Nnu7l-2zspfg-1}{\nwixu{NW1Nnu7l-32pmQO-1}\nwixd{NW1Nnu7l-2zspfg-1}}}%


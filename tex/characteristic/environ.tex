\nwfilename{nw/characteristic/environ.nw}\nwbegindocs{0}\section{Environment}% ===> this file was generated automatically by noweave --- better not edit it

\M
The header, or ``environment part'', tells Mizar what mathematics needs
to be imported from existing Mizar articles found in the MML.

\nwenddocs{}\nwbegincode{1}\sublabel{NW4bbsBH-G7ezF-1}\nwmargintag{{\nwtagstyle{}\subpageref{NW4bbsBH-G7ezF-1}}}\moddef{Environment for \code{}tmp.miz\edoc{}~{\nwtagstyle{}\subpageref{NW4bbsBH-G7ezF-1}}}\endmoddef\nwstartdeflinemarkup\nwenddeflinemarkup
environ
 \LA{}\code{}tmp.miz\edoc{} vocabularies~{\nwtagstyle{}\subpageref{NW4bbsBH-h1cmx-1}}\RA{};
 \LA{}\code{}tmp.miz\edoc{} constructors~{\nwtagstyle{}\subpageref{NW4bbsBH-1JJknr-1}}\RA{};
 \LA{}\code{}tmp.miz\edoc{} notations~{\nwtagstyle{}\subpageref{NW4bbsBH-j1Lze-1}}\RA{};
 \LA{}\code{}tmp.miz\edoc{} registrations~{\nwtagstyle{}\subpageref{NW4bbsBH-1OlNUs-1}}\RA{};
 \LA{}\code{}tmp.miz\edoc{} requirements~{\nwtagstyle{}\subpageref{NW4bbsBH-2HdRVn-1}}\RA{};
 \LA{}\code{}tmp.miz\edoc{} definitions~{\nwtagstyle{}\subpageref{NW4bbsBH-1XYRUf-1}}\RA{};
 \LA{}\code{}tmp.miz\edoc{} equalities~{\nwtagstyle{}\subpageref{NW4bbsBH-1Aylxu-1}}\RA{};
 \LA{}\code{}tmp.miz\edoc{} expansions~{\nwtagstyle{}\subpageref{NW4bbsBH-10tSeT-1}}\RA{};
 \LA{}\code{}tmp.miz\edoc{} theorems~{\nwtagstyle{}\subpageref{NW4bbsBH-4Fk6bh-1}}\RA{};
 \LA{}\code{}tmp.miz\edoc{} schemes~{\nwtagstyle{}\subpageref{NW4bbsBH-285BP4-1}}\RA{};
 \LA{}\code{}tmp.miz\edoc{} requirements~{\nwtagstyle{}\subpageref{NW4bbsBH-2HdRVn-1}}\RA{};

\nwnotused{Environment for [[tmp.miz]]}\nwendcode{}\nwbegindocs{2}\nwdocspar

\subsection{Vocabularies, Notations, Constructors}

\N{Vocabularies}\label{par:characteristic:vocabularies}
The {\Tt{}vocabularies\nwendquote} refers to the identifiers defined. For example,
{\Tt{}Isomorphism\nwendquote} may be found in {\Tt{}RING{\_}3\nwendquote}. So even if I wanted to use
{\Tt{}Isomorphism\nwendquote} as a token, I need to add {\Tt{}RING{\_}3\nwendquote} to
the vocabularies list. Similarly, {\Tt{}MOD{\_}4\nwendquote} introduces the tokens
{\Tt{}Endomorphism\nwendquote} and {\Tt{}Automorphism\nwendquote}, which I want to use, so I add
them, too.

\nwenddocs{}\nwbegincode{3}\sublabel{NW4bbsBH-h1cmx-1}\nwmargintag{{\nwtagstyle{}\subpageref{NW4bbsBH-h1cmx-1}}}\moddef{\code{}tmp.miz\edoc{} vocabularies~{\nwtagstyle{}\subpageref{NW4bbsBH-h1cmx-1}}}\endmoddef\nwstartdeflinemarkup\nwusesondefline{\\{NW4bbsBH-G7ezF-1}}\nwenddeflinemarkup
vocabularies RING_3, MOD_4, TMP,
  \LA{}Functions and subset tokens~{\nwtagstyle{}\subpageref{NW4bbsBH-1ATTd1-1}}\RA{},
  \LA{}Group and subgroups tokens~{\nwtagstyle{}\subpageref{NW4bbsBH-5jMsk-1}}\RA{},
  \LA{}Group conjugation and normal subgroups tokens~{\nwtagstyle{}\subpageref{NW4bbsBH-40Iyod-1}}\RA{}

\nwused{\\{NW4bbsBH-G7ezF-1}}\nwendcode{}\nwbegindocs{4}\nwdocspar

\M
We need to recognize the tokens found in rudimentary set theory, so we
begin with importing the usual suspects.

\nwenddocs{}\nwbegincode{5}\sublabel{NW4bbsBH-1ATTd1-1}\nwmargintag{{\nwtagstyle{}\subpageref{NW4bbsBH-1ATTd1-1}}}\moddef{Functions and subset tokens~{\nwtagstyle{}\subpageref{NW4bbsBH-1ATTd1-1}}}\endmoddef\nwstartdeflinemarkup\nwusesondefline{\\{NW4bbsBH-h1cmx-1}}\nwenddeflinemarkup
MSSUBFAM, RELAT_1, TARSKI, FUNCT_1, ZFMISC_1, NUMBERS, WELLORD1,
SUBSET_1, XBOOLE_0

\nwused{\\{NW4bbsBH-h1cmx-1}}\nwendcode{}\nwbegindocs{6}\nwdocspar

\M
Characteristic subgroups requires recognizing tokens about\dots groups,
and subgroups.

\nwenddocs{}\nwbegincode{7}\sublabel{NW4bbsBH-5jMsk-1}\nwmargintag{{\nwtagstyle{}\subpageref{NW4bbsBH-5jMsk-1}}}\moddef{Group and subgroups tokens~{\nwtagstyle{}\subpageref{NW4bbsBH-5jMsk-1}}}\endmoddef\nwstartdeflinemarkup\nwusesondefline{\\{NW4bbsBH-h1cmx-1}}\nwenddeflinemarkup
STRUCT_0, GROUP_1, GROUP_2, SUBSET_1, GROUP_4, GROUP_5,
MSSUBFAM, GROUP_6, BINOP_1, BINOP_2, ALGSTR_0, REALSET1,
AUTGROUP

\nwused{\\{NW4bbsBH-h1cmx-1}}\nwendcode{}\nwbegindocs{8}\nwdocspar

\M
{\Tt{}NEWTON\nwendquote} defines the token {\Tt{}|{\char94}\nwendquote}, used as infix operator {\Tt{}a\ |{\char94}\ b\nwendquote}
which is Mizar notation for $a^{b}$. Mizar follows group theorist
notation of writing $g^{h} = h^{-1}gh$ for conjugation. Also observe
that {\Tt{}normal\nwendquote} is introduced in {\Tt{}PRE{\_}TOPC\nwendquote}, so we need to include
that, as well.

\nwenddocs{}\nwbegincode{9}\sublabel{NW4bbsBH-40Iyod-1}\nwmargintag{{\nwtagstyle{}\subpageref{NW4bbsBH-40Iyod-1}}}\moddef{Group conjugation and normal subgroups tokens~{\nwtagstyle{}\subpageref{NW4bbsBH-40Iyod-1}}}\endmoddef\nwstartdeflinemarkup\nwusesondefline{\\{NW4bbsBH-h1cmx-1}}\nwenddeflinemarkup
NEWTON, PRE_TOPC, GROUP_3

\nwused{\\{NW4bbsBH-h1cmx-1}}\nwendcode{}\nwbegindocs{10}\nwdocspar

\N{Constructors}
But the vocabularies just permits Mizar's parser to \emph{recognize}
terms. For the \emph{meaning} of these terms, we need to import the
\emph{constructors}. But if a constructor uses \emph{another article}'s
constructors, we need to also import that other article as well.

Often we just copy the articles imported for the notations section, but
in my experience it's often a strict subset of the notations. I'm lazy,
so I'll just copy the constructor imports:

\nwenddocs{}\nwbegincode{11}\sublabel{NW4bbsBH-1JJknr-1}\nwmargintag{{\nwtagstyle{}\subpageref{NW4bbsBH-1JJknr-1}}}\moddef{\code{}tmp.miz\edoc{} constructors~{\nwtagstyle{}\subpageref{NW4bbsBH-1JJknr-1}}}\endmoddef\nwstartdeflinemarkup\nwusesondefline{\\{NW4bbsBH-G7ezF-1}}\nwenddeflinemarkup
constructors \LA{}set theoretic notation for \code{}tmp.miz\edoc{}~{\nwtagstyle{}\subpageref{NW4bbsBH-1usaUC-1}}\RA{},
  \LA{}group theoretic notation for \code{}tmp.miz\edoc{}~{\nwtagstyle{}\subpageref{NW4bbsBH-1E7MdW-1}}\RA{}

\nwused{\\{NW4bbsBH-G7ezF-1}}\nwendcode{}\nwbegindocs{12}\nwdocspar

\N{Notations}
Now we need to import the functor patterns to ``couple'' the definitions
and notations. Usually this is just the constructor list.

\nwenddocs{}\nwbegincode{13}\sublabel{NW4bbsBH-j1Lze-1}\nwmargintag{{\nwtagstyle{}\subpageref{NW4bbsBH-j1Lze-1}}}\moddef{\code{}tmp.miz\edoc{} notations~{\nwtagstyle{}\subpageref{NW4bbsBH-j1Lze-1}}}\endmoddef\nwstartdeflinemarkup\nwusesondefline{\\{NW4bbsBH-G7ezF-1}}\nwenddeflinemarkup
notations \LA{}set theoretic notation for \code{}tmp.miz\edoc{}~{\nwtagstyle{}\subpageref{NW4bbsBH-1usaUC-1}}\RA{},
  \LA{}group theoretic notation for \code{}tmp.miz\edoc{}~{\nwtagstyle{}\subpageref{NW4bbsBH-1E7MdW-1}}\RA{}

\nwused{\\{NW4bbsBH-G7ezF-1}}\nwendcode{}\nwbegindocs{14}\nwdocspar

\M The basics of Tarski--Grothendieck set theory may be found in
{\Tt{}TARSKI\nwendquote}. Partial functions are introduced in {\Tt{}PARTFUN1\nwendquote}. Binary
operations applied to functions {\Tt{}FUNCOP{\_}1\nwendquote} will be necessary later
on. And fancy functions from sets to sets, like {\Tt{}Permutation\nwendquote}, is
defined in {\Tt{}FUNCT{\_}2\nwendquote}. There are few random odds and ends, like
{\Tt{}NUMBERS\nwendquote} for subsets of complex numbers

\nwenddocs{}\nwbegincode{15}\sublabel{NW4bbsBH-1usaUC-1}\nwmargintag{{\nwtagstyle{}\subpageref{NW4bbsBH-1usaUC-1}}}\moddef{set theoretic notation for \code{}tmp.miz\edoc{}~{\nwtagstyle{}\subpageref{NW4bbsBH-1usaUC-1}}}\endmoddef\nwstartdeflinemarkup\nwusesondefline{\\{NW4bbsBH-1JJknr-1}\\{NW4bbsBH-j1Lze-1}}\nwenddeflinemarkup
TARSKI, XBOOLE_0, ZFMISC_1, SUBSET_1, RELAT_1, FUNCT_1,
RELSET_1, PARTFUN1, FUNCT_2, FUNCOP_1, NUMBERS

\nwused{\\{NW4bbsBH-1JJknr-1}\\{NW4bbsBH-j1Lze-1}}\nwendcode{}\nwbegindocs{16}\nwdocspar

\M
The group theoretic notions are a grab bag of binary operators
({\Tt{}BINOP{\_}1\nwendquote} and {\Tt{}BINOP{\_}2\nwendquote}), prerequisites for algebraic structures
({\Tt{}STRUCT{\_}0\nwendquote} and {\Tt{}ALGSTR{\_}0\nwendquote}), primordial group theoretic articles
({\Tt{}REALSET1\nwendquote}), and the relevant group theory articles.

\nwenddocs{}\nwbegincode{17}\sublabel{NW4bbsBH-1E7MdW-1}\nwmargintag{{\nwtagstyle{}\subpageref{NW4bbsBH-1E7MdW-1}}}\moddef{group theoretic notation for \code{}tmp.miz\edoc{}~{\nwtagstyle{}\subpageref{NW4bbsBH-1E7MdW-1}}}\endmoddef\nwstartdeflinemarkup\nwusesondefline{\\{NW4bbsBH-1JJknr-1}\\{NW4bbsBH-j1Lze-1}}\nwenddeflinemarkup
BINOP_1, BINOP_2, STRUCT_0, ALGSTR_0, REALSET1, GROUP_1, GROUP_2,
GROUP_3, GROUP_4, GROUP_5, GROUP_6, GRSOLV_1, AUTGROUP

\nwused{\\{NW4bbsBH-1JJknr-1}\\{NW4bbsBH-j1Lze-1}}\nwendcode{}\nwbegindocs{18}\nwdocspar

\subsection{Registrations, Definitions, Theorems, Schemes}

\N{Registrations}
We often cluster adjectives together with registrations, or have one
adjective imply another automatically (like how a characteristic
Subgroup is always normal). We import these using the registrations
portion of the environment. For our purposes, we may need basic facts
about relations ({\Tt{}RELAT{\_}1\nwendquote}), functions and partial functions ({\Tt{}FUNCT{\_}1\nwendquote},
{\Tt{}PARTFUN1\nwendquote}, {\Tt{}FUNCT{\_}2\nwendquote}), relations between sets ({\Tt{}RELSET{\_}1\nwendquote}).

\nwenddocs{}\nwbegincode{19}\sublabel{NW4bbsBH-1OlNUs-1}\nwmargintag{{\nwtagstyle{}\subpageref{NW4bbsBH-1OlNUs-1}}}\moddef{\code{}tmp.miz\edoc{} registrations~{\nwtagstyle{}\subpageref{NW4bbsBH-1OlNUs-1}}}\endmoddef\nwstartdeflinemarkup\nwusesondefline{\\{NW4bbsBH-G7ezF-1}}\nwenddeflinemarkup
registrations \LA{}Register set theoretic clusters~{\nwtagstyle{}\subpageref{NW4bbsBH-1d38DI-1}}\RA{},
  \LA{}Register group theoretic clusters~{\nwtagstyle{}\subpageref{NW4bbsBH-3IQfHr-1}}\RA{}

\nwused{\\{NW4bbsBH-G7ezF-1}}\nwendcode{}\nwbegindocs{20}\nwdocspar

\M
The clusters we want to use from set theory are defined in the ``same''
scattering of places.

\nwenddocs{}\nwbegincode{21}\sublabel{NW4bbsBH-1d38DI-1}\nwmargintag{{\nwtagstyle{}\subpageref{NW4bbsBH-1d38DI-1}}}\moddef{Register set theoretic clusters~{\nwtagstyle{}\subpageref{NW4bbsBH-1d38DI-1}}}\endmoddef\nwstartdeflinemarkup\nwusesondefline{\\{NW4bbsBH-1OlNUs-1}}\nwenddeflinemarkup
XBOOLE_0, RELAT_1, FUNCT_1, PARTFUN1, RELSET_1, FUNCT_2

\nwused{\\{NW4bbsBH-1OlNUs-1}}\nwendcode{}\nwbegindocs{22}\nwdocspar

\M
We also need to register adjectives germane to group theory.

\nwenddocs{}\nwbegincode{23}\sublabel{NW4bbsBH-3IQfHr-1}\nwmargintag{{\nwtagstyle{}\subpageref{NW4bbsBH-3IQfHr-1}}}\moddef{Register group theoretic clusters~{\nwtagstyle{}\subpageref{NW4bbsBH-3IQfHr-1}}}\endmoddef\nwstartdeflinemarkup\nwusesondefline{\\{NW4bbsBH-1OlNUs-1}}\nwenddeflinemarkup
STRUCT_0, GROUP_1, GROUP_2, GROUP_3, GROUP_6

\nwused{\\{NW4bbsBH-1OlNUs-1}}\nwendcode{}\nwbegindocs{24}\nwdocspar

\N{Definitions} When using a definition $f := M$, we need to cite it in
a proof. Specifically, automatically unfolding predicates from specific
articles. If we want this to be automated, we can cite the article in
the {\Tt{}definitions\nwendquote} portion of the {\Tt{}environ\nwendquote}.

\nwenddocs{}\nwbegincode{25}\sublabel{NW4bbsBH-1XYRUf-1}\nwmargintag{{\nwtagstyle{}\subpageref{NW4bbsBH-1XYRUf-1}}}\moddef{\code{}tmp.miz\edoc{} definitions~{\nwtagstyle{}\subpageref{NW4bbsBH-1XYRUf-1}}}\endmoddef\nwstartdeflinemarkup\nwusesondefline{\\{NW4bbsBH-G7ezF-1}}\nwenddeflinemarkup
definitions \LA{}Include set theoretic definitions~{\nwtagstyle{}\subpageref{NW4bbsBH-4e7fxs-1}}\RA{},
  \LA{}Include group theoretic definitions~{\nwtagstyle{}\subpageref{NW4bbsBH-40sjrh-1}}\RA{}

\nwused{\\{NW4bbsBH-G7ezF-1}}\nwendcode{}\nwbegindocs{26}\nwdocspar

\begin{remark}
  Kornilowicz~\cite[see \S{5.1}]{kornilowicz2015definitional} that:
  ``Environment directive {\Tt{}definitions\nwendquote} is used for importing two different kinds of information from the database: definitional expansions used by REASONER and expansions of terms defined by equals used by EQUALIZER.''
\end{remark}

\M Arguably, we want to be using basic predicates concerning subsets
({\Tt{}SUBSET{\_}1\nwendquote}), functions ({\Tt{}FUNCT{\_}1\nwendquote} and {\Tt{}FUNCT{\_}2\nwendquote}), and set theory
({\Tt{}TARSKI\nwendquote}), so let's just add them.

\nwenddocs{}\nwbegincode{27}\sublabel{NW4bbsBH-4e7fxs-1}\nwmargintag{{\nwtagstyle{}\subpageref{NW4bbsBH-4e7fxs-1}}}\moddef{Include set theoretic definitions~{\nwtagstyle{}\subpageref{NW4bbsBH-4e7fxs-1}}}\endmoddef\nwstartdeflinemarkup\nwusesondefline{\\{NW4bbsBH-1XYRUf-1}}\nwenddeflinemarkup
SUBSET_1, FUNCT_1, TARSKI, FUNCT_2

\nwused{\\{NW4bbsBH-1XYRUf-1}}\nwendcode{}\nwbegindocs{28}\nwdocspar

\M
But we also want to use facts concerning normal subgroups ({\Tt{}GROUP{\_}3\nwendquote})
and the automorphism group $\aut(G)$ ({\Tt{}AUTGROUP\nwendquote}).

\nwenddocs{}\nwbegincode{29}\sublabel{NW4bbsBH-40sjrh-1}\nwmargintag{{\nwtagstyle{}\subpageref{NW4bbsBH-40sjrh-1}}}\moddef{Include group theoretic definitions~{\nwtagstyle{}\subpageref{NW4bbsBH-40sjrh-1}}}\endmoddef\nwstartdeflinemarkup\nwusesondefline{\\{NW4bbsBH-1XYRUf-1}}\nwenddeflinemarkup
GROUP_3, AUTGROUP

\nwused{\\{NW4bbsBH-1XYRUf-1}}\nwendcode{}\nwbegindocs{30}\nwdocspar

\N{Theorems}
The {\Tt{}vocabularies\nwendquote} allows Mizar's scanner and parser to
\emph{recognize} terms. The {\Tt{}constructors\nwendquote} and {\Tt{}notations\nwendquote} allows us
to use the patterns and constructors for terms. But if we want to cite
theorems and definitions in proofs (i.e., if we want to use the
\emph{results} of previous articles), then we need to add those cited
articles to the {\Tt{}theorems\nwendquote} environment.

\nwenddocs{}\nwbegincode{31}\sublabel{NW4bbsBH-4Fk6bh-1}\nwmargintag{{\nwtagstyle{}\subpageref{NW4bbsBH-4Fk6bh-1}}}\moddef{\code{}tmp.miz\edoc{} theorems~{\nwtagstyle{}\subpageref{NW4bbsBH-4Fk6bh-1}}}\endmoddef\nwstartdeflinemarkup\nwusesondefline{\\{NW4bbsBH-G7ezF-1}}\nwenddeflinemarkup
theorems
  \LA{}Import set-theoretic theorems~{\nwtagstyle{}\subpageref{NW4bbsBH-2m40FU-1}}\RA{},
  \LA{}Import group-theoretic theorems~{\nwtagstyle{}\subpageref{NW4bbsBH-cTLti-1}}\RA{}

\nwused{\\{NW4bbsBH-G7ezF-1}}\nwendcode{}\nwbegindocs{32}\nwdocspar

\M We have the usual cast of set theoretic characters.
\nwenddocs{}\nwbegincode{33}\sublabel{NW4bbsBH-2m40FU-1}\nwmargintag{{\nwtagstyle{}\subpageref{NW4bbsBH-2m40FU-1}}}\moddef{Import set-theoretic theorems~{\nwtagstyle{}\subpageref{NW4bbsBH-2m40FU-1}}}\endmoddef\nwstartdeflinemarkup\nwusesondefline{\\{NW4bbsBH-4Fk6bh-1}}\nwenddeflinemarkup
TARSKI_0, TARSKI, SUBSET_1, RELSET_1, FUNCT_1, FUNCT_2, ZFMISC_1,
XBOOLE_0, RELAT_1

\nwused{\\{NW4bbsBH-4Fk6bh-1}}\nwendcode{}\nwbegindocs{34}\nwdocspar

\M Again, we import the usual group theoretic theorems.
\nwenddocs{}\nwbegincode{35}\sublabel{NW4bbsBH-cTLti-1}\nwmargintag{{\nwtagstyle{}\subpageref{NW4bbsBH-cTLti-1}}}\moddef{Import group-theoretic theorems~{\nwtagstyle{}\subpageref{NW4bbsBH-cTLti-1}}}\endmoddef\nwstartdeflinemarkup\nwusesondefline{\\{NW4bbsBH-4Fk6bh-1}}\nwenddeflinemarkup
GROUP_1, GROUP_2, GROUP_3, GROUP_5, GROUP_6, REALSET1, STRUCT_0,
GRSOLV_1, AUTGROUP

\nwused{\\{NW4bbsBH-4Fk6bh-1}}\nwendcode{}\nwbegindocs{36}\nwdocspar

\N{Schemes}
If we want to cite and use a scheme defined elsewhere, then we need the
article's name cited in the {\Tt{}schemes\nwendquote} portion of the environment.

\nwenddocs{}\nwbegincode{37}\sublabel{NW4bbsBH-285BP4-1}\nwmargintag{{\nwtagstyle{}\subpageref{NW4bbsBH-285BP4-1}}}\moddef{\code{}tmp.miz\edoc{} schemes~{\nwtagstyle{}\subpageref{NW4bbsBH-285BP4-1}}}\endmoddef\nwstartdeflinemarkup\nwusesondefline{\\{NW4bbsBH-G7ezF-1}}\nwenddeflinemarkup
schemes BINOP_1, FUNCT_2

\nwused{\\{NW4bbsBH-G7ezF-1}}\nwendcode{}\nwbegindocs{38}\nwdocspar

\subsection{\dots and the rest}

\N{Equalities}
This seems to be introduced around 2015, the only documentation I could
find was in Kornilowics~\cite{kornilowicz2015definitional}.
Expansions of terms defined by {\Tt{}equals\nwendquote} are imported by a new
{\Tt{}environ\nwendquote} directive {\Tt{}equalities\nwendquote}.

\nwenddocs{}\nwbegincode{39}\sublabel{NW4bbsBH-1Aylxu-1}\nwmargintag{{\nwtagstyle{}\subpageref{NW4bbsBH-1Aylxu-1}}}\moddef{\code{}tmp.miz\edoc{} equalities~{\nwtagstyle{}\subpageref{NW4bbsBH-1Aylxu-1}}}\endmoddef\nwstartdeflinemarkup\nwusesondefline{\\{NW4bbsBH-G7ezF-1}}\nwenddeflinemarkup
equalities BINOP_1, REALSET1, GROUP_2, GROUP_3, ALGSTR_0

\nwused{\\{NW4bbsBH-G7ezF-1}}\nwendcode{}\nwbegindocs{40}\nwdocspar

\N{Expansions}
Import expansions of predicates and adjectives from the specified
articles.

\nwenddocs{}\nwbegincode{41}\sublabel{NW4bbsBH-10tSeT-1}\nwmargintag{{\nwtagstyle{}\subpageref{NW4bbsBH-10tSeT-1}}}\moddef{\code{}tmp.miz\edoc{} expansions~{\nwtagstyle{}\subpageref{NW4bbsBH-10tSeT-1}}}\endmoddef\nwstartdeflinemarkup\nwusesondefline{\\{NW4bbsBH-G7ezF-1}}\nwenddeflinemarkup
expansions BINOP_1, FUNCT_2

\nwused{\\{NW4bbsBH-G7ezF-1}}\nwendcode{}\nwbegindocs{42}\nwdocspar

\N{Requirements} Within mathematics, there's a lot of implicit
knowledge. Mizar automates some of that with {\Tt{}requirements\nwendquote}
inclusions. For example, if we want to show {\Tt{}x\ in\ X\nwendquote}
(Mizar for the primitive binary predicate $x\in X$) implies
the typing relation {\Tt{}x\ is\ Element\ of\ X\nwendquote}, well, that's ``obvious'' to
us humans, and we can make it obvious to Mizar as well using the proper
requirements.

\begin{remark}
As I understand it (from Wiedijk's excellent ``Writing a Mizar Article
in 9 easy steps''), the only possibilities for the {\Tt{}requirements\nwendquote} are:
{\Tt{}BOOLE\nwendquote}, {\Tt{}SUBSET\nwendquote}, {\Tt{}NUMERALS\nwendquote}, {\Tt{}ARITHM\nwendquote}, {\Tt{}REAL\nwendquote}.
\end{remark}

\nwenddocs{}\nwbegincode{43}\sublabel{NW4bbsBH-2HdRVn-1}\nwmargintag{{\nwtagstyle{}\subpageref{NW4bbsBH-2HdRVn-1}}}\moddef{\code{}tmp.miz\edoc{} requirements~{\nwtagstyle{}\subpageref{NW4bbsBH-2HdRVn-1}}}\endmoddef\nwstartdeflinemarkup\nwusesondefline{\\{NW4bbsBH-G7ezF-1}}\nwenddeflinemarkup
requirements BOOLE, SUBSET

\nwused{\\{NW4bbsBH-G7ezF-1}}\nwendcode{}\nwbegindocs{44}\nwdocspar

\nwenddocs{}

\nwixlogsorted{c}{{\code{}tmp.miz\edoc{} constructors}{NW4bbsBH-1JJknr-1}{\nwixu{NW4bbsBH-G7ezF-1}\nwixd{NW4bbsBH-1JJknr-1}}}%
\nwixlogsorted{c}{{\code{}tmp.miz\edoc{} definitions}{NW4bbsBH-1XYRUf-1}{\nwixu{NW4bbsBH-G7ezF-1}\nwixd{NW4bbsBH-1XYRUf-1}}}%
\nwixlogsorted{c}{{\code{}tmp.miz\edoc{} equalities}{NW4bbsBH-1Aylxu-1}{\nwixu{NW4bbsBH-G7ezF-1}\nwixd{NW4bbsBH-1Aylxu-1}}}%
\nwixlogsorted{c}{{\code{}tmp.miz\edoc{} expansions}{NW4bbsBH-10tSeT-1}{\nwixu{NW4bbsBH-G7ezF-1}\nwixd{NW4bbsBH-10tSeT-1}}}%
\nwixlogsorted{c}{{\code{}tmp.miz\edoc{} notations}{NW4bbsBH-j1Lze-1}{\nwixu{NW4bbsBH-G7ezF-1}\nwixd{NW4bbsBH-j1Lze-1}}}%
\nwixlogsorted{c}{{\code{}tmp.miz\edoc{} registrations}{NW4bbsBH-1OlNUs-1}{\nwixu{NW4bbsBH-G7ezF-1}\nwixd{NW4bbsBH-1OlNUs-1}}}%
\nwixlogsorted{c}{{\code{}tmp.miz\edoc{} requirements}{NW4bbsBH-2HdRVn-1}{\nwixu{NW4bbsBH-G7ezF-1}\nwixd{NW4bbsBH-2HdRVn-1}}}%
\nwixlogsorted{c}{{\code{}tmp.miz\edoc{} schemes}{NW4bbsBH-285BP4-1}{\nwixu{NW4bbsBH-G7ezF-1}\nwixd{NW4bbsBH-285BP4-1}}}%
\nwixlogsorted{c}{{\code{}tmp.miz\edoc{} theorems}{NW4bbsBH-4Fk6bh-1}{\nwixu{NW4bbsBH-G7ezF-1}\nwixd{NW4bbsBH-4Fk6bh-1}}}%
\nwixlogsorted{c}{{\code{}tmp.miz\edoc{} vocabularies}{NW4bbsBH-h1cmx-1}{\nwixu{NW4bbsBH-G7ezF-1}\nwixd{NW4bbsBH-h1cmx-1}}}%
\nwixlogsorted{c}{{Environment for \code{}tmp.miz\edoc{}}{NW4bbsBH-G7ezF-1}{\nwixd{NW4bbsBH-G7ezF-1}}}%
\nwixlogsorted{c}{{Functions and subset tokens}{NW4bbsBH-1ATTd1-1}{\nwixu{NW4bbsBH-h1cmx-1}\nwixd{NW4bbsBH-1ATTd1-1}}}%
\nwixlogsorted{c}{{Group and subgroups tokens}{NW4bbsBH-5jMsk-1}{\nwixu{NW4bbsBH-h1cmx-1}\nwixd{NW4bbsBH-5jMsk-1}}}%
\nwixlogsorted{c}{{Group conjugation and normal subgroups tokens}{NW4bbsBH-40Iyod-1}{\nwixu{NW4bbsBH-h1cmx-1}\nwixd{NW4bbsBH-40Iyod-1}}}%
\nwixlogsorted{c}{{group theoretic notation for \code{}tmp.miz\edoc{}}{NW4bbsBH-1E7MdW-1}{\nwixu{NW4bbsBH-1JJknr-1}\nwixu{NW4bbsBH-j1Lze-1}\nwixd{NW4bbsBH-1E7MdW-1}}}%
\nwixlogsorted{c}{{Import group-theoretic theorems}{NW4bbsBH-cTLti-1}{\nwixu{NW4bbsBH-4Fk6bh-1}\nwixd{NW4bbsBH-cTLti-1}}}%
\nwixlogsorted{c}{{Import set-theoretic theorems}{NW4bbsBH-2m40FU-1}{\nwixu{NW4bbsBH-4Fk6bh-1}\nwixd{NW4bbsBH-2m40FU-1}}}%
\nwixlogsorted{c}{{Include group theoretic definitions}{NW4bbsBH-40sjrh-1}{\nwixu{NW4bbsBH-1XYRUf-1}\nwixd{NW4bbsBH-40sjrh-1}}}%
\nwixlogsorted{c}{{Include set theoretic definitions}{NW4bbsBH-4e7fxs-1}{\nwixu{NW4bbsBH-1XYRUf-1}\nwixd{NW4bbsBH-4e7fxs-1}}}%
\nwixlogsorted{c}{{Register group theoretic clusters}{NW4bbsBH-3IQfHr-1}{\nwixu{NW4bbsBH-1OlNUs-1}\nwixd{NW4bbsBH-3IQfHr-1}}}%
\nwixlogsorted{c}{{Register set theoretic clusters}{NW4bbsBH-1d38DI-1}{\nwixu{NW4bbsBH-1OlNUs-1}\nwixd{NW4bbsBH-1d38DI-1}}}%
\nwixlogsorted{c}{{set theoretic notation for \code{}tmp.miz\edoc{}}{NW4bbsBH-1usaUC-1}{\nwixu{NW4bbsBH-1JJknr-1}\nwixu{NW4bbsBH-j1Lze-1}\nwixd{NW4bbsBH-1usaUC-1}}}%


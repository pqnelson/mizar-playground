\nwfilename{nw/characteristic/isomorphism.nw}\nwbegindocs{0}\subsection{Isomorphisms}% ===> this file was generated automatically by noweave --- better not edit it

\M
We also want to prove results concerning group isomorphisms. This
requires a preliminary notion for a group $G_{2}$ being a
$G_{1}$-isomorphic group.

\nwenddocs{}\nwbegincode{1}\sublabel{NW1mUXx1-2BsOkA-1}\nwmargintag{{\nwtagstyle{}\subpageref{NW1mUXx1-2BsOkA-1}}}\moddef{Group isomorphisms~{\nwtagstyle{}\subpageref{NW1mUXx1-2BsOkA-1}}}\endmoddef\nwstartdeflinemarkup\nwenddeflinemarkup
\LA{}Define when a group is \code{}G-isomorphic\edoc{}~{\nwtagstyle{}\subpageref{NW1mUXx1-34yWsG-1}}\RA{}
\LA{}Register \code{}G-isomorphic\edoc{} for groups~{\nwtagstyle{}\subpageref{NW1mUXx1-3gtzsG-1}}\RA{}
\LA{}Register \code{}bijective\edoc{} for \code{}Homomorphism\edoc{} of $G_{1}$, $G_{2}$~{\nwtagstyle{}\subpageref{NW1mUXx1-3ik4I8-1}}\RA{}
\LA{}Define group \code{}Isomorphism\edoc{}~{\nwtagstyle{}\subpageref{NW1mUXx1-2NbDsC-1}}\RA{}

\nwnotused{Group isomorphisms}\nwendcode{}\nwbegindocs{2}\nwdocspar


\M It's useful to introduce, not just a group $K$, but a $G$-isomorphic
group $K$.

\nwenddocs{}\nwbegincode{3}\sublabel{NW1mUXx1-34yWsG-1}\nwmargintag{{\nwtagstyle{}\subpageref{NW1mUXx1-34yWsG-1}}}\moddef{Define when a group is \code{}G-isomorphic\edoc{}~{\nwtagstyle{}\subpageref{NW1mUXx1-34yWsG-1}}}\endmoddef\nwstartdeflinemarkup\nwusesondefline{\\{NW1mUXx1-2BsOkA-1}}\nwenddeflinemarkup
definition :: Def6
  let G1,G2 be Group;
  attr G2 is G1-isomorphic means :Def6:
  G1,G2 are_isomorphic;
end;

\nwused{\\{NW1mUXx1-2BsOkA-1}}\nwendcode{}\nwbegindocs{4}\nwdocspar

\M We also wanter to register this attribute (``being $G$-isomorphic'')
as a perfectly fine adjective for groups. This requires proving that,
for any group $G_{1}$ we have a $G_{1}$-isomorphic group.

\nwenddocs{}\nwbegincode{5}\sublabel{NW1mUXx1-3gtzsG-1}\nwmargintag{{\nwtagstyle{}\subpageref{NW1mUXx1-3gtzsG-1}}}\moddef{Register \code{}G-isomorphic\edoc{} for groups~{\nwtagstyle{}\subpageref{NW1mUXx1-3gtzsG-1}}}\endmoddef\nwstartdeflinemarkup\nwusesondefline{\\{NW1mUXx1-2BsOkA-1}}\nwenddeflinemarkup
registration
  let G1 be Group;
  cluster G1-isomorphic for Group;
  existence
  proof
    take G1;
    Id_G1 is bijective;
    thus thesis;
  end;
end;

\nwused{\\{NW1mUXx1-2BsOkA-1}}\nwendcode{}\nwbegindocs{6}\nwdocspar

\N{Registering {\Tt{}bijective\nwendquote} for {\Tt{}Homomorphism\nwendquote}}
We had to take a detour, because if you just gave me any two random
groups $G_{1}$ and $G_{2}$, there is zero reason to believe they are
isomorphic. Just consider any two finite groups of different size. And
group isomorphisms are only well-defined between isomorphic groups. Now
that we have a notion of isomorphic-groups, we can define a notion of
{\Tt{}Isomorphism\nwendquote}. We will also need to prove their existence, which we
place in a lemma.

\nwenddocs{}\nwbegincode{7}\sublabel{NW1mUXx1-3ik4I8-1}\nwmargintag{{\nwtagstyle{}\subpageref{NW1mUXx1-3ik4I8-1}}}\moddef{Register \code{}bijective\edoc{} for \code{}Homomorphism\edoc{} of $G_{1}$, $G_{2}$~{\nwtagstyle{}\subpageref{NW1mUXx1-3ik4I8-1}}}\endmoddef\nwstartdeflinemarkup\nwusesondefline{\\{NW1mUXx1-2BsOkA-1}}\nwenddeflinemarkup
\LA{}Lemma: $G_{2}$ is $G_{1}$-isomorphic implies existence of isomorphism~{\nwtagstyle{}\subpageref{NW1mUXx1-gupLZ-1}}\RA{}

registration
  let G1 be Group,
      G2 be G1-isomorphic Group;
  cluster bijective for Homomorphism of G1,G2;
  existence by Lm5;
end;

\nwused{\\{NW1mUXx1-2BsOkA-1}}\nwendcode{}\nwbegindocs{8}\nwdocspar

\N{Lemma: Isomorphic groups have an isomorphism between them}
Let $G_{1}$ be a group. For any $G_{1}$-isomorphic group $G_{2}$,
there exists at least one isomorphism $G_{1}\to G_{2}$. It's just a
straightforward matter of unwinding the definitions.

\nwenddocs{}\nwbegincode{9}\sublabel{NW1mUXx1-gupLZ-1}\nwmargintag{{\nwtagstyle{}\subpageref{NW1mUXx1-gupLZ-1}}}\moddef{Lemma: $G_{2}$ is $G_{1}$-isomorphic implies existence of isomorphism~{\nwtagstyle{}\subpageref{NW1mUXx1-gupLZ-1}}}\endmoddef\nwstartdeflinemarkup\nwusesondefline{\\{NW1mUXx1-3ik4I8-1}}\nwenddeflinemarkup
Lm5:
  for G2 being G1-isomorphic Group
  holds (ex h being Homomorphism of G1,G2 st h is bijective)
proof
  let G2 be G1-isomorphic Group;
  G1,G2 are_isomorphic by Def6;
  then consider h being Homomorphism of G1,G2 such that
A1: h is bijective by GROUP_6:def 11;
  thus thesis by A1;
end;

\nwused{\\{NW1mUXx1-3ik4I8-1}}\nwendcode{}\nwbegindocs{10}\nwdocspar

\N{Definition} We can now define a notion of group {\Tt{}Isomorphism\nwendquote} in
Mizar. We can't do it ``willy-nilly'', of course: the notion of an
isomorphism only makes sense when it is from a group $G_{1}$ to a
$G_{1}$-isomorphic group. But given such a condition on $G_{2}$, we can
define an {\Tt{}Isomorphism\nwendquote} as just a bijective group morphism $G_{1}\to G_{2}$.

\nwenddocs{}\nwbegincode{11}\sublabel{NW1mUXx1-2NbDsC-1}\nwmargintag{{\nwtagstyle{}\subpageref{NW1mUXx1-2NbDsC-1}}}\moddef{Define group \code{}Isomorphism\edoc{}~{\nwtagstyle{}\subpageref{NW1mUXx1-2NbDsC-1}}}\endmoddef\nwstartdeflinemarkup\nwusesondefline{\\{NW1mUXx1-2BsOkA-1}}\nwenddeflinemarkup
definition :: Def7
  let G1 be Group,
      G2 be G1-isomorphic Group;
  mode Isomorphism of G1,G2 is bijective Homomorphism of G1,G2;
end;

\nwused{\\{NW1mUXx1-2BsOkA-1}}\nwendcode{}\nwbegindocs{12}\nwdocspar

\nwenddocs{}

\nwixlogsorted{c}{{Define group \code{}Isomorphism\edoc{}}{NW1mUXx1-2NbDsC-1}{\nwixu{NW1mUXx1-2BsOkA-1}\nwixd{NW1mUXx1-2NbDsC-1}}}%
\nwixlogsorted{c}{{Define when a group is \code{}G-isomorphic\edoc{}}{NW1mUXx1-34yWsG-1}{\nwixu{NW1mUXx1-2BsOkA-1}\nwixd{NW1mUXx1-34yWsG-1}}}%
\nwixlogsorted{c}{{Group isomorphisms}{NW1mUXx1-2BsOkA-1}{\nwixd{NW1mUXx1-2BsOkA-1}}}%
\nwixlogsorted{c}{{Lemma: $G_{2}$ is $G_{1}$-isomorphic implies existence of isomorphism}{NW1mUXx1-gupLZ-1}{\nwixu{NW1mUXx1-3ik4I8-1}\nwixd{NW1mUXx1-gupLZ-1}}}%
\nwixlogsorted{c}{{Register \code{}bijective\edoc{} for \code{}Homomorphism\edoc{} of $G_{1}$, $G_{2}$}{NW1mUXx1-3ik4I8-1}{\nwixu{NW1mUXx1-2BsOkA-1}\nwixd{NW1mUXx1-3ik4I8-1}}}%
\nwixlogsorted{c}{{Register \code{}G-isomorphic\edoc{} for groups}{NW1mUXx1-3gtzsG-1}{\nwixu{NW1mUXx1-2BsOkA-1}\nwixd{NW1mUXx1-3gtzsG-1}}}%


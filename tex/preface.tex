\preface

This is an attempt to learn Mizar while formalizing some results in
finite group theory. I am assuming familiarity with group theory, the
reader has at least an undergraduate understanding of mathematics.
Results are reviewed in Chapter~\ref{chapter:pure-math}.

\subsection*{Goal: Quasithin Groups}
One of the driving questions is how feasible is it to formalize the
classification theorem of quasithin groups? Aschbacher and Smith have
written a self-contained 1220-page
proof~\cite{aschbacher2004classification1,aschbacher2004classification2}.

This constitutes about a quarter of the classification theorem for
finite simple groups, the theorem with the world's longest proof. The
classification of quasithin groups constitutes a ``proof of concept''
whether formalizing the entire classification theorem is feasible ``by hand''.

\bigbreak
Can Mizar formalize the mathematics surrounding the classification of
quasithin groups?

The answer appears to be, ``If so, not at present.'' Mizar is lacking
some infrastructure. This document is an attempt to formalize
\emph{some} of that infrastructure, capture some of the thought process
behind formalizing the mathematics, and provide some insight into how
the mathematics works.

\bigbreak
As far as the classification of finite simple groups, there are two
proofs of it: the first generation proof spans approximately {15,000}
pages across decades of articles in many technical journals; the second
generation attempts to collate this in about a dozen books (volume 40 in
AMS's series ``Mathematical Surveys and Monographs''). Thus far, the
nine books of the second generation proof span 3994 pages; four more
books are likely to be published\footnote{According to Ron Solomon's
personal communication to Hugo de Garis, as reported on MathOverflow \url{https://mathoverflow.net/a/217397/22457}}, resulting in approximately 3041
pages\footnote{This is based on using a simple linear regression on the
number of pages in volume $X$ more than in volume one, against the
volume number. The adjusted $R^{2}$ for this model is $0.8806$, so it's
not terrible.}. The approximately 7035 page proof is a ``bridge too
far'', the quasithin classification is one-sixth this size: still big,
but at least plausibly formalizable.

% 137 + 200 + 400 + 331 + 471 + 518 + 340 + 483 + 514
% = 2880 + 514 = 3394
% vol 1 = 139 - 2
% vol 2 = 215 - 15
% vol 3 = 419 - 19
% vol 4 = 349 - 18
% vol 5 = 475 - 14
% vol 6 = 532 - 14
% vol 7 = 352 - 12
% vol 8 = 483
% vol 9 = 530 - 16

\subsection*{On Style}
The writing style is particularly idiosyncratic. Although Mizar is
designed in a declarative proof style, and highly readable, it is
difficult (for me) to keep track of what I've done and where I'm going.

Literate programming techniques are used to organize and present the
proof scripts and definitions. Cross-referencing occurs automatically,
so we can see where definitions and theorems are used (for the reader's
convenience). For me, as author, literate programming let's me track
intent and keep goals in mind.

The pure mathematician can read these notes like it's written in the
style of the great Arthur Besse\index{Besse, Arthur L.}. Or this document may be read, like
Edward Gibbon's\index{Gibbon, Edward}
\emph{Decline and Fall}, as written by two Alex Nelsons: one directing
his voice in code towards Mizar (formatted in \verb#teletype font#), the
other in \emph{sotto voce} towards the reader through ``footnotes'' (the
annotations in English which constitute roughly two-thirds of the text).

\bigbreak
\subsubsection*{On Proofs, Proof Outlines, Proof Sketches, Proof Steps}
Since this is a literate program (or collection of literate programs), I
give formal proofs in Mizar with some sketch or outline in English. The
terms I use vary, but I try to be consistent in their usage:
\begin{enumerate}
\item ``Proof sketch'' for ordinary proofs summarizing the corresponding
Mizar formal proof;
\item ``Proof outline'' for ``roadmaps'' of long Mizar proofs which we
have divided into several chunks or ``proof steps'';
\item ``Proof steps'' for each chunk in a ``proof outline''.
\end{enumerate}
Sometimes I prepend ``sub-'' to these terms and give the claim being
proven in parentheses, in an attempt to clearly identify the scope of
the proof chunk.

\subsection*{On Mizar}
Mizar is a proof assistant working in set theoretic foundations. It is
among the most sophisticated theorem provers around. Mizar's language is
human readable and resembles ``ordinary mathematics''. Unfortunately, it
lacks documentation in one central location. The best introductions on
the subject remain~\cite{grabowski2010mizar}
and~\cite{wiedijk2006mizman}.

I don't think its sophistication is adequately appreciated or understood
by its competitors. Wiedijk attempts to model its type
system~\cite{wiedijk2007mizar}, but type theoretic wags just sneer,
``Why not use a \emph{real} type system?'' Those wags have now probably been eaten
by bears, like the children who made mock of the prophet Elisha. But if
they have somehow survived, they may wish to confer de Bruijn's mathematical
vernacular (see section {3.6} of~\cite{de1994mathematical}) where soft
typing (what de Bruijn calls ``higher typing'') is first introduced in
formalizing mathematical texts. Soft type systems are characteristic of
serious attempts at formalizing mathematics \emph{in any foundation or framework.}

Josef Urban~\cite{urban2003translating} summarizes the constructor
system and its translation to ``vanilla'' first-order logic. What's a
``constructor'' in Mizar? Well, as Trybulec~\cite{trybulec1993some}
explains: a \emph{predicate} is a ``constructor'' of (atomic) formulas,
a \emph{mode} is a constructor of (atomic) types, a \emph{functor} is a
constructor of (atomic) terms. These are the constructors we will
encounter.

For convenience, I have collected an index for occurrances of Mizar code
(in ``Mizar References''). Any defined term which appears in this text
should appear there, as well as any definitions or theorems referenced
from the Mizar library.

\subsection*{Bibliography, MML References}
I have collected the cited books and articles in the Bibliography. These
are the books and articles published over the years which accumulate and
transmit knowledge. With the exception of ``common knowledge'',
definitions and major theorems will cite sources which are stored in the
bibliography.

There are also Mizar articles, which are published in the
journal \emph{Formalized Mathematics}. Those articles we use during the
process formalizing group theory appear in the chapter ``MML Bibliography''.

\subsection*{Apology: proposition numbering.} There are four parallel
``theorem counters'', one for:
\begin{enumerate}
\item definitions,
\item theorems (and, within this text, corollaries),
\item lemmas,
\item everything else (propositions, examples, etc.).
\end{enumerate}
This mirrors Mizar's numbering scheme, where definitions are numbered
sequentially independent of theorems, and ``lemmas'' are ``private''
propositions using an \emph{ad hoc} numbering scheme. My hands are tied,
and I thought it best to imitate this numbering scheme, despite my
personal objections to it. In my private writings, I avoided such a
confusing scheme. But in Mizar, it makes much more sense, and is far
more robust than my personal scheme. All I can do is say this: I am so,
so sorry.

One convention I adopt is to number remarks after whatever is being
discussed. For example, if I have ``definition 7'', a remark about this
definition will be numbered ``remark 7.1''. The next theorem, suppose it
was ``theorem 23'', a remark about theorem 23 would be ``remark 23.1''.

\subsection*{Outline}
In the first chapter, I introduce Mizar briefly, and then review group
theory with references to corresponding definitions and theorems in the
Mizar Mathematical Library.

The second chapter formalizes characteristic subgroups.

%% \M There are a few drawbacks to Mizar's sophisticated system. For
%% example, many of the sloppy conveniences which working mathematicians
%% implicitly use (like treating isomorphic groups as equal) no longer
%% hold.
%%
%% Another problem is that there are various synonyms which confuse
%% neophytes: ``\verb#NAT#'' is a synonym for $\omega$ (the smallest
%% infinite ordinal, see \verb#ORDINAL1:def 11#) as defined in \verb#NUMBERS#,
%% ``\verb#Nat#'' is defined in \verb#ORDINAL1# as
%% ``\verb#natural number#'' (where ``\verb#natural#'' is an adjective
%% specifying $\mathtt{it}\in\omega$ and ``\verb#number#'' is a synonym for
%% ``\verb#set#''). The other number systems have a similar ``two pronged''
%% definition strategy.

\aufm{Alex Nelson\\
December 14, 2021}

\begin{chapterquotes}
  For small erections may be finished by their first architects;
  grand ones, true ones, ever leave the copestone to posterity.
  God keep me from ever completing anything.
  This whole book is but a draught---nay,
  but the draught of a draught.
  Oh, Time, Strength,
  Cash, and Patience!
  \author Herman Melville, \title{Moby Dick} (1851)
\end{chapterquotes}
